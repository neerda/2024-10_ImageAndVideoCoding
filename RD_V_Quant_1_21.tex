
%===== main document class =====
%\ifdefined\slideModeHandout
\documentclass%
[%
  %handout,          % avoid unnecessary overlays
  aspectratio=169,  % aspect ratio fo 16:9 
  t,                % place content at top of frames
  10pt,             % use 10pt as standard font size (default size is 11pt)
  compress,         % compress things like navigation bars...
]{beamer}
%\else
%\documentclass%
%[%
%  aspectratio=169,  % aspect ratio fo 16:9 
%  t,                % place content at top of frames
%  10pt,             % use 10pt as standard font size (default size is 11pt)
%  compress,         % compress things like navigation bars...
%]{beamer}
%\fi

\mode<presentation>
%===============tabto==============
% tabto.sty
%
% version 1.4  (Dec 2018)
%
% Tabbing to fixed positions in a paragraph.
%
% Copyright 2006,2009,2012,2013,2018 by 
% Donald Arseneau,   Vancouver, Canada (asnd@triumf.ca)
% Permission to use, distribute and modify this software is granted
% under the conditions of the LaTeX Project Public License, either 
% version 1.3 or (at your option) any later version.  The license is
% found at http://www.latex-project.org/lppl.txt, and is part of all 
% recent distributions of LaTeX.
%
% This work has the LPPL maintenance status `maintained' (by author).
%
% Two new text positioning commands are defined: \tabto and \tab.
% 
% \tabto{<length>}
% Tab to a position relative to the left margin in a paragraph (any
% indentation due to a list or \leftskip is part of the `margin' in
% this context). If the text on the line already goes past the desired
% position, the tab starts a new line and moves to the requested
% horizontal position.
%
% \tabto*{<length>}
% Similar to \tabto, except it will perform backspacing, and over-
% print previous text on the line whenever that text is already
% longer than the specified length (i.e., no linebreak is produced).
% Line-breaks are suppressed immediately after \tabto or \tabto*.
%
% The length register "\CurrentLineWidth" will report the width
% of the existing text on the line, and it may be used in the
% <length> argument (using calc.sty, for example). Also, there
% is "\TabPrevPos" which gives the "\CurrentLineWidth" from the
% previous tab command (the position where the tab command occurred,
% not where it went to), and can be used to return to that position
% if no line breaks have occurred in between, or directly below it,
% if there were line breaks.
%
% \tab
% Tab to the next tab-stop chosen from a list of tab positions, in
% the traditional style of typewriters.  A \tab will always move
% to the next tab stop (or the next line), even if it is already
% exactly at a tab stop. Thus, "\tab" at the beginning of a line,
% or "\tab\tab" elsewhere skips a position. A linebreak is permitted 
% immediately following a \tab, in case the ensuing text does not 
% fit well in the remaining space.
%
% If you do not want to skip positions, use "\tabto{\NextTabStop}"
% instead of "\tab".  This is particularly useful when you want to
% use \tab in some other command, but do not want to skip a column
% for the first item.
%
% The tab-stop positions are declared using either \TabPositions
% or \NumTabs:
%
% \TabPositions{<length>, <length>,...<length>}
% Declares the tab stops as a comma-separated list of positions 
% relative to the left margin. A tab-stop at 0pt is implicit, and 
% need not be listed.
%
% \NumTabs{<number>}
% Declares a list of <number> equally-spaced tabs, starting at the
% left margin and spanning \linewidth.  For example \NumTabs{2} 
% declares tab-stops at 0pt and 0.5\linewidth, the same as
% \TabPositions{0pt, 0.5\linewidth} or \TabPositions{0.5\linewidth}
%
% After these declarations, the list of tab positions is saved in
% \TabStopList, and the next tab position, relative to the current 
% position, is given by \NextTabStop.  You do not normally need
% to access them, but they are available.
%
% Problems:
%
% Tall objects after a tab stop may overlap the line above, rather
% than forcing a greater separation between lines.

%\ProvidesPackage{tabto}[2018/12/28 \space v 1.4 \space 
%  Another tabbing mechanism]\relax

%%%%%%%%%Code Begin

%\newdimen\CurrentLineWidth
%\newdimen\TabPrevPos
%
%\newcommand\tabto[1]{%
% \leavevmode
% \begingroup
% \def\@tempa{*}\def\@tempb{#1}%
% \ifx\@tempa\@tempb % \tab* 
%   \endgroup
%   \TTo@overlaptrue % ... set a flag and re-issue \tabto to get argument
%   \expandafter\tabto
% \else
%   \ifinner % in a \hbox, so ignore
%   \else % unrestricted horizontal mode
%     \null% \predisplaysize will tell the position of this box (must be box)
%     \parfillskip\fill
%     \everydisplay{}\everymath{}%
%     \predisplaypenalty\@M \postdisplaypenalty\@M
%     $$% math display so we can test \predisplaysize
%      \lineskiplimit=-999pt % so we get pure \baselineskip
%      \abovedisplayskip=-\baselineskip \abovedisplayshortskip=-\baselineskip
%      \belowdisplayskip\z@skip \belowdisplayshortskip\z@skip
%      \halign{##\cr\noalign{%
%        % get the width of the line above
%        \ifdim\predisplaysize=\maxdimen %\message{Mixed R and L, so say the line is full. }%
%           \CurrentLineWidth\linewidth
%        \else
%           \ifdim\predisplaysize=-\maxdimen 
%             % \message{Not in a paragraph, so call the line empty. }%
%             \CurrentLineWidth\z@
%           \else
%             \ifnum\TTo@Direction<\z@
%               \CurrentLineWidth\linewidth \advance\CurrentLineWidth\predisplaysize
%             \else
%               \CurrentLineWidth\predisplaysize 
%             \fi
%             % Correct the 2em offset
%             \advance\CurrentLineWidth -2em
%             \advance\CurrentLineWidth -\displayindent
%             \advance\CurrentLineWidth -\leftskip
%        \fi\fi
%        \ifdim\CurrentLineWidth<\z@ \CurrentLineWidth\z@\fi
%        % Enshrine the tab-to position; #1 might reference \CurrentLineWidth
%        \setlength\@tempdimb{#1}% allow calc.sty
%        %\message{*** Tab to \the\@tempdimb, previous width is \the\CurrentLineWidth. ***}%
%        % Save width for possible return use
%        \global\TabPrevPos\CurrentLineWidth
%        % Build the action to perform
%        \protected@xdef\TTo@action{%
%           \vrule\@width\z@\@depth\the\prevdepth
%           \ifdim\CurrentLineWidth>\@tempdimb
%              \ifTTo@overlap\else
%                 \protect\newline \protect\null
%           \fi\fi
%           \protect\nobreak
%           \protect\hskip\the\@tempdimb\relax
%        }%
%        %\message{\string\TTo@action: \meaning \TTo@action. }%
%        % get back to the baseline, regardless of its depth.
%        \vskip-\prevdepth
%        \prevdepth-99\p@
%        \vskip\prevdepth
%      }}%
%      $$
%     % Don't count the display as lines in the paragraph
%     \count@\prevgraf \advance\count@-4 \prevgraf\count@
%     \TTo@action
%     %%   \penalty\@m % to allow a penalized line break
%   \fi
%   \endgroup
%   \TTo@overlapfalse
%   \ignorespaces
% \fi
%}
%
%% \tab -- to the next position
%% \hskip so \tab\tab moves two positions
%% Allow a (penalized but flexible) line-break right after the tab.
%%
%\newcommand\tab{\leavevmode\hskip2sp\tabto{\NextTabStop}%
%  \nobreak\hskip\z@\@plus 30\p@\penalty4000\hskip\z@\@plus-30\p@\relax}
%
%
%% Expandable macro to select the next tab position from the list
%
%\newcommand\NextTabStop{%
%  \expandafter \TTo@nexttabstop \TabStopList,\maxdimen,>%
%}
%
%\def\TTo@nexttabstop #1,{%
%    \ifdim#1<\CurrentLineWidth
%      \expandafter\TTo@nexttabstop
%    \else
%      \ifdim#1<0.9999\linewidth#1\else\z@\fi
%      \expandafter\strip@prefix
%    \fi
%}
%\def\TTo@foundtabstop#1>{}
%
%\newcommand\TabPositions[1]{\def\TabStopList{\z@,#1}}
%
%\newcommand\NumTabs[1]{%
%   \def\TabStopList{}%
%   \@tempdimb\linewidth 
%   \divide\@tempdimb by#1\relax
%   \advance\@tempdimb 1sp % counteract rounding-down by \divide
%   \CurrentLineWidth\z@
%   \@whiledim\CurrentLineWidth<\linewidth\do {%
%     \edef\TabStopList{\TabStopList\the\CurrentLineWidth,}%
%     \advance\CurrentLineWidth\@tempdimb
%   }%
%   \edef\TabStopList{\TabStopList\linewidth}%
%}
%
% %default setting of tab positions:
%\TabPositions{\parindent,.5\linewidth}
%
%%\newif\ifTTo@overlap \TTo@overlapfalse
%
%%\@ifundefined{predisplaydirection}{
%% \let\TTo@Direction\predisplaysize
%% \let\predisplaydirection\@undefined
%%}{
%% \let\TTo@Direction\predisplaydirection
%%}
%
%

% ===== include packages =====
\usepackage{lmodern}
\usepackage[utf8]{inputenc}      % Unicode UTF-8 encoding support
\usepackage[T2A,T1]{fontenc}         % T1 font encoding
\usepackage{etoolbox}            % Programming tools (used for \insertpartstartframe, \insertpartendframe)
\usepackage{ifthen}              % Simple conditional statements
\usepackage{amsmath}             % AMS math package
\usepackage{amssymb}             % Extended collection of math symbols
\usepackage{mathtools}           % More math stuff
\usepackage{nicefrac}            % Nice fractions
\usepackage{xcolor}              % Driver independent colors
\usepackage{colortbl}            % rowcolor for tables
\usepackage{array}               % tables and arrays with extended features (e.g., overlays)
\usepackage{makecell}            % Simple formatting of single table cells
\usepackage{multirow}            % Multi-rows in tabular environments
\usepackage{booktabs}            % More flexible lines for tabular environments
\usepackage{tabto}               % Easy way of specifying tabulators
\usepackage{adjustbox}           % Macros for adjusting boxed content (used in defining \fitbox)
\usepackage{relsize}             % Relative font sizes (\larger=\relsize{1}, \smaller=\relsize{-1})
\usepackage{graphicx}            % Inserting pictures
\usepackage{hyperref}            % Cross referencing
%\usepackage{media9}              % Include media objects
%\usepackage{animate}             % Animations
\usepackage{tikz}                % TikZ library for drawing
\usepackage{pgfplots}            % Plots
\usepackage{import}              % including stuff with relative paths
%\usepackage{listings}            % source code inclusion

%===== TikZ libraries =====
\usetikzlibrary{shapes}          % Additional shapes: Ellipse
\usetikzlibrary{shapes.symbols}  % Additional shapes: Symbols (e.g., "cloud")
\usetikzlibrary{shapes.arrows}   % Additional shapes: Arrows (e.g., "single arrow")
\usetikzlibrary{arrows}          % Arrow tips
\usetikzlibrary{arrows.meta}     % Adjustable arrow heads
\usetikzlibrary{positioning}     % Relative positioning of nodes

%===== pgfsetting =====
\pgfplotsset{compat=newest}      % Use newest version of pgf
\usepgfplotslibrary{fillbetween} % filling between curves

%===== nice tables =====
\setlength{\heavyrulewidth}{0.08em}
\setlength{\lightrulewidth}{0.08em}
\setlength{\cmidrulewidth}{0.04em}
\setlength{\aboverulesep}{0.4ex}
\setlength{\belowrulesep}{0.6ex}
\newcommand{\cmidbeg}{\addlinespace[0.40ex]}
\newcommand{\cmidend}{\addlinespace[0.10ex]}

\usepackage{caption}
\usepackage{subcaption}


%%%%%%%%%%%%%%%%%%%%%%%%%%%%%%%%%%%%%%%%%%%%%%%%%%%%%%%%%%%%
%%%%%%
%%%%%%     M A I N   D O C U M E N T   S W I T C H E S     
%%%%%%
%%%%%%%%%%%%%%%%%%%%%%%%%%%%%%%%%%%%%%%%%%%%%%%%%%%%%%%%%%%%

% define command for directly using switches
\newcommand{\usetoggle}[1]{\iftoggle{#1}{true}{false}}

% define switches
\newtoggle{useNavSymbols}               % display of navigation symbols
\newtoggle{useShadows}                  % use blocks with shadows
\newtoggle{useColorBlocks}              % use colored blocks
\newtoggle{useColorBlockTitles}         % use colored block titles
\newtoggle{useInverseBlockTitles}       % use colored block title background with white text
\newtoggle{altColors}                   % use alternative color theme
\newtoggle{addExercises}                % whether exercises are used
\newtoggle{specialHeiko}                % special stuff for Heiko
\newtoggle{specialThomas}               % special stuff for Thomas

\def\slideStyleThomas{..}


\ifdefined\slideStyleThomas

  %>>>>>>>>>> parameters for Thomas >>>>>>>>>>
  \title%
      [Image and Video Coding]%
      {Image and Video Coding I:\\[0.5ex] Fundamentals}
  \author%
      [T. Wiegand, J. Pfaff, J. Rasch]%
      {Thomas Wiegand, Jonathan Pfaff, Jennifer Rasch}
  \institute%
      [TU Berlin, Fraunhofer HHI]%
      {Technische Universität Berlin, Fraunhofer Heinrich Hertz Institute, Berlin}

  \newcommand{\slideOrganization}{}

  \settoggle{useNavSymbols}        {false}
  \settoggle{useShadows}           {true}
  \settoggle{useColorBlocks}       {true}
  \settoggle{useColorBlockTitles}  {true}
  \settoggle{useInverseBlockTitles}{true}
  \settoggle{altColors}            {false}
  \settoggle{addExercises}         {false}

  \settoggle{specialHeiko}         {false}
  \settoggle{specialThomas}        {true}
  %<<<<<<<<<< parameters for Thomas <<<<<<<<<<

\else

  %>>>>>>>>>> parameters for Heiko >>>>>>>>>>
  \title%
      {Data Compression}
  \author%
      [Heiko Schwarz]%
      {Heiko Schwarz}
  \institute%
      [Freie Universität Berlin]%
      {Freie Universität Berlin\\%
      Fachbereich Mathematik und Informatik}

\ifdefined\slideModeHandout
  \settoggle{useNavSymbols}        {false}
\else
  \settoggle{useNavSymbols}        {false}
\fi
  \settoggle{useShadows}           {true}
  \settoggle{useColorBlocks}       {true}
  \settoggle{useColorBlockTitles}  {true}
  \settoggle{useInverseBlockTitles}{true}
  \settoggle{altColors}            {false}
  \settoggle{addExercises}         {true}

  \settoggle{specialHeiko}         {true}
  \settoggle{specialThomas}        {false}
  %<<<<<<<<<< parameters for Heiko <<<<<<<<<<

\fi % end of conditional





%%%%%%%%%%%%%%%%%%%%%%%%%%%%%%%%%%%%%%%%%%%%%%%%%
%%%%%%
%%%%%%     C U S T O M I Z E   D E S I G N
%%%%%%
%%%%%%%%%%%%%%%%%%%%%%%%%%%%%%%%%%%%%%%%%%%%%%%%%

%===== spacing for lists and paragraphs (modified copy from beamerbaselocalstructure.sty) =====
\makeatletter
\setlength  \parskip         {2ex}
\setlength  \leftmargini     {2em}
\setlength  \leftmarginii    {2em}
\setlength  \leftmarginiii   {2em}
\setlength  \labelsep        {.5em}
\setlength  \labelwidth      {\leftmargini}
\addtolength\labelwidth      {-\labelsep}
\def\@listi  {\leftmargin\leftmargini
              \partopsep  \parskip
              \parskip    0.0\p@
              \parsep     0.0\p@
              \topsep     3.0\p@ \@plus1.0\p@ \@minus2.0\p@
              \itemsep    3.0\p@ \@plus1.0\p@ \@minus2.0\p@}
\let\@listI\@listi
\def\@listii {\leftmargin\leftmarginii
              \parsep     0.0\p@
              \topsep     3.0\p@ \@plus1.0\p@ \@minus2.0\p@
              \itemsep    3.0\p@ \@plus1.0\p@ \@minus2.0\p@}
\def\@listiii{\leftmargin\leftmarginiii
              \parsep     0.0\p@
              \topsep     3.0\p@ \@plus1.0\p@ \@minus2.0\p@
              \itemsep    3.0\p@ \@plus1.0\p@ \@minus2.0\p@}
\makeatother


%===== counter for exercises =====
\newcounter{exercise}


%===== enumerate symbols (modified copy from beamerbaseauxtemplates.sty) =====
\makeatletter

%--- define commands for changing enum style ---
\newcommand{\setenumstyledepth}[2]{% {enumdepth}{command for displaying counter}
  \ifthenelse{\equal{#1}{1}}%
     {\renewcommand*{\theenumi}{#2{enumi}}}%
     {\ifthenelse{\equal{#1}{2}}%
        {\renewcommand*{\theenumii}{#2{enumii}}}%
        {\renewcommand*{\theenumiii}{#2{enumiii}}}%
     }%
}
% Note: For the following command you can also use your own styles.
%       For example, a style "A." in a smaller font can be defined by
%         \newcommand{\AlphaDot}[1]{{\smaller\smaller\Alph{#1}.}}
\newcommand{\enumStyle}[1]{\setenumstyledepth{\the\@enumdepth}{#1}}
\newcommand{\enumStylesDefault}[3]{%
  \setenumstyledepth{1}{#1}%
  \setenumstyledepth{2}{#2}%
  \setenumstyledepth{3}{#3}%
}

%--- define commands for putting enum symbols ---
\newcommand{\putenumsquare}[1]{%
  \smaller%
  \usebeamercolor[bg]{\beameritemnestingprefix item projected}%
  \begin{pgfpicture}{-1ex}{-0.25ex}{1.1ex}{2.0ex}%
    \pgfpathrectangle{\pgfpoint{-1.2ex}{-0.6ex}}{\pgfpoint{2.4ex}{2.4ex}}%
    \pgfusepath{fill}%
    \pgftext[base,y=-0.15ex]{\color{fg}#1}%
  \end{pgfpicture}%
}
\newcommand{\putenumcircle}[1]{%
  \smaller%
  \usebeamercolor[bg]{\beameritemnestingprefix item projected}%
  \begin{pgfpicture}{-1ex}{-0.25ex}{1.1ex}{2.0ex}%
    \pgfpathcircle{\pgfpoint{0pt}{.6ex}}{1.3ex}%
    \pgfusepath{fill}%
    \pgftext[base,y=-0.15ex]{\color{fg}#1}%
  \end{pgfpicture}%
}
\newcommand{\putenumblank}[1]{%
  \smaller%
  \begin{pgfpicture}{-1ex}{-0.25ex}{1.1ex}{2.0ex}%
    \pgfpathrectangle{\pgfpoint{-1.2ex}{-0.6ex}}{\pgfpoint{2.4ex}{2.4ex}}%
    \pgftext[base,y=-0.15ex]{#1}%
  \end{pgfpicture}%
}
\newcommand{\putenumbracket}[1]{%
  \smaller%
  \begin{pgfpicture}{-1ex}{-0.25ex}{1.1ex}{2.0ex}%
    \pgfpathrectangle{\pgfpoint{-1.2ex}{-0.6ex}}{\pgfpoint{2.4ex}{2.4ex}}%
    \pgftext[base,y=-0.15ex]{$\boldsymbol{\langle}$#1$\boldsymbol{\rangle}$}%
  \end{pgfpicture}%
}
\newcommand{\putenumautoi}[1]{\putenumcircle{#1}}
\newcommand{\putenumautoii}[1]{\putenumcircle{#1}}
\newcommand{\putenumautoiii}[1]{\putenumcircle{#1}}
\newcommand{\putenumauto}[1]{%
  \ifthenelse{\equal{\the\@itemdepth}{1}}%
     {\putenumautoi{#1}}%
     {\ifthenelse{\equal{\the\@itemdepth}{2}}%
        {\putenumautoii{#1}}%
        {\putenumautoiii{#1}}%
     }%
}

%--- define beamer enum templates [square][circle][blank][bracket][auto] ---
\expandafter\let\csname beamer@@tmpop@enumerate item@square\endcsname\relax
\expandafter\let\csname beamer@@tmpop@enumerate subitem@square\endcsname\relax
\expandafter\let\csname beamer@@tmpop@enumerate subsubitem@square\endcsname\relax
\expandafter\let\csname beamer@@tmpop@enumerate mini template@square\endcsname\relax

\expandafter\let\csname beamer@@tmpop@enumerate item@circle\endcsname\relax
\expandafter\let\csname beamer@@tmpop@enumerate subitem@circle\endcsname\relax
\expandafter\let\csname beamer@@tmpop@enumerate subsubitem@circle\endcsname\relax
\expandafter\let\csname beamer@@tmpop@enumerate mini template@circle\endcsname\relax

\defbeamertemplate{enumerate item}{square}{\putenumsquare{\insertenumlabel}}
\defbeamertemplate{enumerate subitem}{square}{\putenumsquare{\insertsubenumlabel}}
\defbeamertemplate{enumerate subsubitem}{square}{\putenumsquare{\insertsubsubenumlabel}}
\defbeamertemplate{enumerate mini template}{square}{\putenumsquare{\insertenumlabel}}

\defbeamertemplate{enumerate item}{circle}{\putenumcircle{\insertenumlabel}}
\defbeamertemplate{enumerate subitem}{circle}{\putenumcircle{\insertsubenumlabel}}
\defbeamertemplate{enumerate subsubitem}{circle}{\putenumcircle{\insertsubsubenumlabel}}
\defbeamertemplate{enumerate mini template}{circle}{\putenumcircle{\insertenumlabel}}

\defbeamertemplate{enumerate item}{blank}{\putenumblank{\insertenumlabel}}
\defbeamertemplate{enumerate subitem}{blank}{\putenumblank{\insertsubenumlabel}}
\defbeamertemplate{enumerate subsubitem}{blank}{\putenumblank{\insertsubsubenumlabel}}
\defbeamertemplate{enumerate mini template}{blank}{\putenumblank{\insertenumlabel}}

\defbeamertemplate{enumerate item}{bracket}{\putenumbracket{\insertenumlabel}}
\defbeamertemplate{enumerate subitem}{bracket}{\putenumbracket{\insertsubenumlabel}}
\defbeamertemplate{enumerate subsubitem}{bracket}{\putenumbracket{\insertsubsubenumlabel}}
\defbeamertemplate{enumerate mini template}{bracket}{\putenumbracket{\insertenumlabel}}

\defbeamertemplate{enumerate item}{auto}{\putenumauto{\insertenumlabel}}
\defbeamertemplate{enumerate subitem}{auto}{\putenumauto{\insertsubenumlabel}}
\defbeamertemplate{enumerate subsubitem}{auto}{\putenumauto{\insertsubsubenumlabel}}
\defbeamertemplate{enumerate mini template}{auto}{\putenumauto{\insertenumlabel}}

%--- define commands for easily changing enum modes ---
% The outcommented simple version has a problem with enum nested in itemize (wrong level)
%   \newcommand{\enumMode}[1]{\setbeamertemplate{enumerate \beameritemnestingprefix item}[#1]} 
\newcommand{\enumMode}[1]{%
  \ifthenelse{\equal{\the\@enumdepth}{1}}%
     {\setbeamertemplate{enumerate item}[#1]}%
     {\ifthenelse{\equal{\the\@enumdepth}{2}}%
        {\setbeamertemplate{enumerate subitem}[#1]}%
        {\setbeamertemplate{enumerate subsubitem}[#1]}%
     }%
}
\newcommand{\enumAutoDefault}[3]{%
  \renewcommand*{\putenumautoi}[1]{\csname putenum#1\endcsname{##1}}%
  \renewcommand*{\putenumautoii}[1]{\csname putenum#2\endcsname{##1}}%
  \renewcommand*{\putenumautoiii}[1]{\csname putenum#3\endcsname{##1}}%
}

\makeatother



%===== itemize symbols (modified copy from beamerbaseauxtemplates.sty) =====
\makeatletter

%--- new item symbols ---
\newcommand{\isquare}{%
  \begin{pgfpicture}%
    \pgfpathrectangle{\pgfpointorigin}{\pgfpoint{1.0ex}{1.0ex}}%
    \pgfusepath{fill}%
    \pgfsetbaseline{-0.2ex}%
  \end{pgfpicture}%
}
\newcommand{\icircle}{%
  \begin{pgfpicture}%
    \pgfpathcircle{\pgfpoint{0pt}{.5ex}}{0.5ex}%
    \pgfusepath{fill}%
    \pgfsetbaseline{-0.2ex}%
  \end{pgfpicture}%
}
\newcommand{\itriangle}{%
  \begin{pgfpicture}%
    \pgfpathmoveto{\pgfpointorigin}
    \pgfpathlineto{\pgfpoint{0.0ex}{1.0ex}}%
    \pgfpathlineto{\pgfpoint{1.0ex}{0.5ex}}%
    \pgfusepath{fill}%
    \pgfsetbaseline{-0.2ex}%
  \end{pgfpicture}%
}
\newcommand{\idash}{%
  \begin{pgfpicture}%
    \pgfpathrectangle{\pgfpoint{0.0ex}{0.4ex}}{\pgfpoint{1.0ex}{0.2ex}}%
    \pgfusepath{fill}%
    \pgfsetbaseline{-0.2ex}%
  \end{pgfpicture}%
}
\newcommand{\iarrow}{%
  \begin{pgfpicture}%
    \pgfpathmoveto{\pgfpointorigin}
    \pgfpathlineto{\pgfpoint{-0.80ex}{ 0.75ex}}%    (-hl)( ht)  % tl =      total length
    \pgfpathlineto{\pgfpoint{-0.80ex}{ 0.25ex}}%    (-hl)( tt)  % tt = half total thickness
    \pgfpathlineto{\pgfpoint{-2.00ex}{ 0.25ex}}%    (-tl)( tt)  % hl =      head length
    \pgfpathlineto{\pgfpoint{-2.00ex}{-0.25ex}}%    (-tl)(-tt)  % ht = half head thickness
    \pgfpathlineto{\pgfpoint{-0.80ex}{-0.25ex}}%    (-hl)(-tt)
    \pgfpathlineto{\pgfpoint{-0.80ex}{-0.75ex}}%    (-hl)(-ht)
    \pgfusepath{fill}%
  \end{pgfpicture}%
}
\newcommand{\idarrow}{%
  \begin{pgfpicture}%
    \pgfpathmoveto{\pgfpointorigin}
    \pgfpathlineto{\pgfpoint{-0.80ex}{ 0.75ex}}%    (-hl)( ht)  % tl =      total length
    \pgfpathlineto{\pgfpoint{-0.80ex}{ 0.25ex}}%    (-hl)( tt)  % tt = half total thickness
    \pgfpathlineto{\pgfpoint{-2.00ex}{ 0.25ex}}%    (-tl)( tt)  % hl =      head length
    \pgfpathlineto{\pgfpoint{-2.00ex}{ 0.75ex}}%
    \pgfpathlineto{\pgfpoint{-2.80ex}{ 0.00ex}}%
    \pgfpathlineto{\pgfpoint{-2.00ex}{-0.75ex}}%
    \pgfpathlineto{\pgfpoint{-2.00ex}{-0.25ex}}%    (-tl)(-tt)  % ht = half head thickness
    \pgfpathlineto{\pgfpoint{-0.80ex}{-0.25ex}}%    (-hl)(-tt)
    \pgfpathlineto{\pgfpoint{-0.80ex}{-0.75ex}}%    (-hl)(-ht)
    \pgfusepath{fill}%
  \end{pgfpicture}%
}

%--- define beamer item templates [square][circle][triangle][dash][arrow] ---
\expandafter\let\csname beamer@@tmpop@itemize item@square\endcsname\relax
\expandafter\let\csname beamer@@tmpop@itemize subitem@square\endcsname\relax
\expandafter\let\csname beamer@@tmpop@itemize subsubitem@square\endcsname\relax

\expandafter\let\csname beamer@@tmpop@itemize item@circle\endcsname\relax
\expandafter\let\csname beamer@@tmpop@itemize subitem@circle\endcsname\relax
\expandafter\let\csname beamer@@tmpop@itemize subsubitem@circle\endcsname\relax

\expandafter\let\csname beamer@@tmpop@itemize item@triangle\endcsname\relax
\expandafter\let\csname beamer@@tmpop@itemize subitem@triangle\endcsname\relax
\expandafter\let\csname beamer@@tmpop@itemize subsubitem@triangle\endcsname\relax

\defbeamertemplate{itemize item}{square}{\isquare}
\defbeamertemplate{itemize subitem}{square}{\isquare}
\defbeamertemplate{itemize subsubitem}{square}{\isquare}

\defbeamertemplate{itemize item}{circle}{\icircle}
\defbeamertemplate{itemize subitem}{circle}{\icircle}
\defbeamertemplate{itemize subsubitem}{circle}{\icircle}

\defbeamertemplate{itemize item}{triangle}{\itriangle}
\defbeamertemplate{itemize subitem}{triangle}{\itriangle}
\defbeamertemplate{itemize subsubitem}{triangle}{\itriangle}

\defbeamertemplate{itemize item}{dash}{\idash}
\defbeamertemplate{itemize subitem}{dash}{\idash}
\defbeamertemplate{itemize subsubitem}{dash}{\idash}

\defbeamertemplate{itemize item}{arrow}{\iarrow}
\defbeamertemplate{itemize subitem}{arrow}{\iarrow}
\defbeamertemplate{itemize subsubitem}{arrow}{\iarrow}

%--- define command for easily changing item styles ---
\newcommand{\itemMode}[1]{\setbeamertemplate{itemize \beameritemnestingprefix item}[#1]}

\makeatother



%===== commands for text highlighting  =====
% helping command
\newcommand{\setfontrm} {\fontshape{\rmdefault}\selectfont}
\newcommand{\setfontit} {\fontshape{\itdefault}\selectfont}
\newcommand{\setfontrs} {\fontseries{\mddefault}\selectfont}
\newcommand{\setfontbs} {\fontseries{\bfdefault}\selectfont}
\newcommand{\setfontrrm}{\fontseries{\mddefault}\fontshape{\rmdefault}\selectfont}
\newcommand{\setfontrit}{\fontseries{\mddefault}\fontshape{\itdefault}\selectfont}
\newcommand{\setfontbrm}{\fontseries{\bfdefault}\fontshape{\rmdefault}\selectfont}
\newcommand{\setfontbit}{\fontseries{\bfdefault}\fontshape{\itdefault}\selectfont}
% normal text attributes:
  % setting series
  \newcommand<>{\regu}  [1]{{\only#2{\setfontrs}#1}}
  \newcommand<>{\bold}  [1]{{\only#2{\setfontbs}#1}}
  % setting shape
  \newcommand<>{\norm}  [1]{{\only#2{\setfontrm}#1}}
  \newcommand<>{\ital}  [1]{{\only#2{\setfontit}#1}}
  % setting series and shape
  \newcommand<>{\rnorm} [1]{{\only#2{\setfontrrm}#1}}
  \newcommand<>{\rital} [1]{{\only#2{\setfontrit}#1}}
  \newcommand<>{\bnorm} [1]{{\only#2{\setfontbrm}#1}}
  \newcommand<>{\bital} [1]{{\only#2{\setfontbit}#1}}
% special text highlighting:
  % changing color only
\renewcommand<>{\alert} [1]{{\only#2{\usebeamercolor{alerted text}\color{fg}}#1}}
  \newcommand<>{\struc} [1]{{\only#2{\usebeamercolor{structure}\color{fg}}#1}}
  \newcommand<>{\examp} [1]{{\only#2{\usebeamercolor{example text}\color{fg}}#1}}
  % changing color and series
  \newcommand<>{\Alert} [1]{{\only#2{\setfontrs\usebeamercolor{alerted text}\color{fg}}#1}}
  \newcommand<>{\Struc} [1]{{\only#2{\setfontrs\usebeamercolor{structure}\color{fg}}#1}}
  \newcommand<>{\Examp} [1]{{\only#2{\setfontrs\usebeamercolor{example text}\color{fg}}#1}}
  \newcommand<>{\ALERT} [1]{{\only#2{\setfontbs\usebeamercolor{alerted text}\color{fg}}#1}}
  \newcommand<>{\STRUC} [1]{{\only#2{\setfontbs\usebeamercolor{structure}\color{fg}}#1}}
  \newcommand<>{\EXAMP} [1]{{\only#2{\setfontbs\usebeamercolor{example text}\color{fg}}#1}}
  % changing color and shape
  \newcommand<>{\ralert}[1]{{\only#2{\setfontrm\usebeamercolor{alerted text}\color{fg}}#1}}
  \newcommand<>{\rstruc}[1]{{\only#2{\setfontrm\usebeamercolor{structure}\color{fg}}#1}}
  \newcommand<>{\rexamp}[1]{{\only#2{\setfontrm\usebeamercolor{example text}\color{fg}}#1}}
  \newcommand<>{\ialert}[1]{{\only#2{\setfontit\usebeamercolor{alerted text}\color{fg}}#1}}
  \newcommand<>{\istruc}[1]{{\only#2{\setfontit\usebeamercolor{structure}\color{fg}}#1}}
  \newcommand<>{\iexamp}[1]{{\only#2{\setfontit\usebeamercolor{example text}\color{fg}}#1}}
  % changing color, shape, and series
  \newcommand<>{\rAlert}[1]{{\only#2{\setfontrrm\usebeamercolor{alerted text}\color{fg}}#1}}
  \newcommand<>{\rStruc}[1]{{\only#2{\setfontrrm\usebeamercolor{structure}\color{fg}}#1}}
  \newcommand<>{\rExamp}[1]{{\only#2{\setfontrrm\usebeamercolor{example text}\color{fg}}#1}}
  \newcommand<>{\rALERT}[1]{{\only#2{\setfontbrm\usebeamercolor{alerted text}\color{fg}}#1}}
  \newcommand<>{\rSTRUC}[1]{{\only#2{\setfontbrm\usebeamercolor{structure}\color{fg}}#1}}
  \newcommand<>{\rEXAMP}[1]{{\only#2{\setfontbrm\usebeamercolor{example text}\color{fg}}#1}}
  \newcommand<>{\iAlert}[1]{{\only#2{\setfontrit\usebeamercolor{alerted text}\color{fg}}#1}}
  \newcommand<>{\iStruc}[1]{{\only#2{\setfontrit\usebeamercolor{structure}\color{fg}}#1}}
  \newcommand<>{\iExamp}[1]{{\only#2{\setfontrit\usebeamercolor{example text}\color{fg}}#1}}
  \newcommand<>{\iALERT}[1]{{\only#2{\setfontbit\usebeamercolor{alerted text}\color{fg}}#1}}
  \newcommand<>{\iSTRUC}[1]{{\only#2{\setfontbit\usebeamercolor{structure}\color{fg}}#1}}
  \newcommand<>{\iEXAMP}[1]{{\only#2{\setfontbit\usebeamercolor{example text}\color{fg}}#1}}
% specials: Emphasizing and names [May want to redefine in actually used style]
\renewcommand<>{\emph}  [1]{\alert#2{#1}}
  \newcommand<>{\Emph}  [1]{\ALERT#2{#1}}
  \newcommand<>{\EMPH}  [1]{\iALERT#2{#1}}
  \newcommand  {\aname} [1]{{\rmfamily\scshape #1}}



%===== define subblock environments =====
\makeatletter
\newenvironment{nesting}{%
  \par\hspace{2\beamer@leftmargin}
  \begin{minipage}{\linewidth-\beamer@leftmargin-3.5\beamer@rightmargin}%
}{%
  \end{minipage}
}
\makeatother



%===== define hooks for accessing frame number inside part  =====
\makeatletter
\newcount\beamer@partstartframe
\beamer@partstartframe=1
\apptocmd{\beamer@part}{\addtocontents{nav}{\protect\headcommand{%
    \protect\beamer@partframes{\the\beamer@partstartframe}{\the\c@framenumber}}}}{}{}
\apptocmd{\beamer@part}{\beamer@partstartframe=\c@framenumber\advance\beamer@partstartframe by1\relax}{}{}
\AtEndDocument{\immediate\write\@auxout{\string\@writefile{nav}%
    {\noexpand\headcommand{\noexpand\beamer@partframes{\the\beamer@partstartframe}{\the\c@framenumber}}}}}{}{}
\def\beamer@startframeofpart{1}
\def\beamer@endframeofpart{1}
\def\beamer@partframes#1#2{%
    \ifnum\c@framenumber<#1%
    \else%
    \ifnum\c@framenumber>#2%
    \else%
    \gdef\beamer@startframeofpart{#1}%
    \gdef\beamer@endframeofpart{#2}%
    \fi%
    \fi%
}
\newcommand\insertpartstartframe{\beamer@startframeofpart}
\newcommand\insertpartendframe{\beamer@endframeofpart}
\makeatother
\def\inserttotalpartframenumber{%
    \pgfmathparse{(\insertpartendframe-\insertpartstartframe+1)}%
    \pgfmathprintnumber[fixed,precision=2]{\pgfmathresult}%
}
\def\insertpartframenumber{%
    \pgfmathparse{(\insertframenumber-\insertpartstartframe+1)}%
    \pgfmathprintnumber[fixed,precision=2]{\pgfmathresult}%
}



%===== define visible on macro for tikz pictures  =====
% see https://tex.stackexchange.com/questions/55806/mindmap-tikzpicture-in-beamer-reveal-step-by-step#55849
\tikzset{
  invisible/.style={opacity=0},
  visible on/.style={alt={#1{}{invisible}}},
  alt/.code args={<#1>#2#3}{%
    \alt<#1>{\pgfkeysalso{#2}}{\pgfkeysalso{#3}} % \pgfkeysalso doesn't change the path
  },
  action/.code args={<#1>#2}{%
    \action<#1>{\pgfkeysalso{#2}} % \pgfkeysalso doesn't change the path
  },
}
% see https://tex.stackexchange.com/questions/6135/how-to-make-beamer-overlays-with-tikz-node
\tikzset{onslide/.code args={<#1>#2}{% \pgfkeysalso doesn't change the path
  \only<#1>{\pgfkeysalso{#2}} %
}}
\tikzset{temporal/.code args={<#1>#2#3#4}{% \pgfkeysalso doesn't change the path
  \temporal<#1>{\pgfkeysalso{#2}}{\pgfkeysalso{#3}}{\pgfkeysalso{#4}} %
}}


%===== further helpful tikz macros =====
\newcommand{\budash}{{\tikz[baseline=0.1ex]\draw[thick](0,0)--({0.6em},0);}}
\newcommand{\nudash}{{\tikz[baseline=0.1ex]\draw[]     (0,0)--({0.6em},0);}}



%===== define a fitbox command  =====
\makeatletter
\newlength{\fitboxw}
\newlength{\fitboxh}
\newlength{\slideinnerheight}
\setlength{\slideinnerheight}{0.85\textheight} %%% could be reset later
\newcommand<>{\fitbox}[4][c]{%
  \only#5{%
    {%
      \setlength{\fitboxw}{#2}%
      \setlength{\fitboxh}{#3}%
      \parbox[#1][\fitboxh]{\fitboxw}{%
        \centering%
        \vfill%
        \adjustbox{%
          min width=\fitboxw,%
          min height=\fitboxh,%
          max width=\fitboxw,%
          max height=\fitboxh%
        }%
        {#4}%
        \vfill%
      }%
    }%
  }%
}
\newcommand<>{\hfitbox}[3][c]{%
  \only#4{%
    {%
      \setlength{\fitboxw}{#2}%
      \parbox[#1]{\fitboxw}{%
        \adjustbox{%
          min width=\fitboxw,%
          max width=\fitboxw}%
        {#3}%
      }%
    }%
  }%
}
\newcommand<>{\slidehfitbox}[1]{%
  \hfitbox#2{\textwidth}{#1}%
}
\newcommand<>{\slidefitbox}[1]{%
  \fitbox#2{\textwidth}{\slideinnerheight}{#1}%
}
\makeatother



%===== command for adding new part with part page / adding title page  =====
\newcommand{\startnewpart}[2][\usebeamercolor{background}\color{bg}\rule{1pt}{1pt}]{
  \part{#2}
  {
    \setbeamertemplate{navigation symbols}{}
    \begin{frame}[plain]
    \vfill\vfill
    {\hfill\begin{beamercolorbox}[%
       sep=8pt,dp=1ex,center,wd=0.8\textwidth,%
       rounded=true,
       shadow=\usetoggle{useShadows}%
    ]%
    {part title}
    \usebeamerfont{part title}\insertpart\par
    \end{beamercolorbox}\hfill}
    \vfill
    {\hfill{\fitbox{0.8\textwidth}{0.45\textheight}{#1}}\hfill}
    \vspace{0pt}
    \end{frame}
  }
}
\newcommand{\addtitlepage}{
  {
    \setbeamertemplate{navigation symbols}{}
    \begin{frame}[plain]
    \vspace{0.15\textheight}
    {\hfill\begin{beamercolorbox}[%
       sep=8pt,dp=1ex,center,wd=0.8\textwidth,%
       rounded=true,
       shadow=\usetoggle{useShadows}%
    ]%
    {title}
    \usebeamerfont{title}\inserttitle\par
    \end{beamercolorbox}\hfill}

    \begin{center}\large
      ~\\[3ex]
      {\insertauthor}\\[3ex]
      {\insertinstitute}
    \end{center}
    \end{frame}
  }
}







%%%%%%%%%%%%%%%%%%%%%%%%%%%%%%%%%%%%%%%%%%%
%%%%%%
%%%%%%     T E M P L A T E
%%%%%%
%%%%%%%%%%%%%%%%%%%%%%%%%%%%%%%%%%%%%%%%%%%

%===== other font size =====
\makeatletter
\newcommand\notsotiny{\@setfontsize\notsotiny\@vipt\@viipt}
\makeatother

%===== font theme =====
\usefonttheme{structurebold}
\setbeamerfont{title in head/foot}{size=\tiny}
\setbeamerfont{author in head/foot}{size=\tiny}
\setbeamerfont{date in head/foot}{size=\tiny}
\setbeamerfont*{frametitle}{family=\sffamily,series=\bfseries,shape=\upshape,size=\large}


%===== color theme (based on color theme "beaver") =====
%>>> define colors (see http://latexcolor.com/ for a good visual comparison)
% reds
\definecolor{bostonuniversityred}     {rgb}{0.80, 0.00, 0.00} % used in beaver
\definecolor{cornellred}              {rgb}{0.70, 0.11, 0.11}
\definecolor{red(ncs)}                {rgb}{0.77, 0.01, 0.20}
\definecolor{carmine}                 {rgb}{0.59, 0.00, 0.09}
\definecolor{crimsonglory}            {rgb}{0.75, 0.00, 0.20}
\definecolor{deepcarmine}             {rgb}{0.66, 0.13, 0.24}
\definecolor{harvardcrimson}          {rgb}{0.79, 0.00, 0.09}
\definecolor{lava}                    {rgb}{0.81, 0.06, 0.13}
\definecolor{mordantred19}            {rgb}{0.68, 0.05, 0.00}
\definecolor{persianred}              {rgb}{0.80, 0.20, 0.20}
\definecolor{raspberry}               {rgb}{0.89, 0.04, 0.36}
% greens                              
\definecolor{othergreen}              {rgb}{0.00, 0.80, 0.00}
\definecolor{officegreen}             {rgb}{0.00, 0.50, 0.00}
\definecolor{darkgreen}               {rgb}{0.00, 0.20, 0.13}
\definecolor{pakistangreen}           {rgb}{0.00, 0.40, 0.00}
\definecolor{cadmiumgreen}            {rgb}{0.00, 0.42, 0.24}
\definecolor{lincolngreen}            {rgb}{0.11, 0.35, 0.02}
\definecolor{dartmouthgreen}          {rgb}{0.05, 0.50, 0.06}
\definecolor{sacramentostategreen}    {rgb}{0.00, 0.34, 0.25}
\definecolor{tropicalrainforest}      {rgb}{0.00, 0.46, 0.37}
\definecolor{upforestgreen}           {rgb}{0.00, 0.27, 0.13}
\definecolor{lasallegreen}            {rgb}{0.03, 0.47, 0.19}
\definecolor{indiagreen}              {rgb}{0.07, 0.53, 0.03}
\definecolor{forestgreen(traditional)}{rgb}{0.00, 0.27, 0.13}
% blues                               
\definecolor{mediumblue}              {rgb}{0.00, 0.00, 0.80}
\definecolor{navyblue}                {rgb}{0.00, 0.00, 0.50}
\definecolor{ceruleanblue}            {rgb}{0.16, 0.32, 0.75}
\definecolor{internationalkleinblue}  {rgb}{0.00, 0.18, 0.65}
\definecolor{royalazure}              {rgb}{0.00, 0.22, 0.66}
\definecolor{smalt(darkpowderblue)}   {rgb}{0.00, 0.20, 0.60}
\definecolor{ultramarine}             {rgb}{0.07, 0.04, 0.56}
\definecolor{zaffre}                  {rgb}{0.00, 0.08, 0.66}
\definecolor{phthaloblue}             {rgb}{0.00, 0.06, 0.54}
\definecolor{persianblue}             {rgb}{0.11, 0.22, 0.73}
\definecolor{palatinateblue}          {rgb}{0.15, 0.23, 0.89}
% other
\definecolor{vividviolet}             {rgb}{0.62, 0.00, 1.00}
\definecolor{uclagold}                {rgb}{1.00, 0.70, 0.00}
\definecolor{tangerineyellow}         {rgb}{1.00, 0.80, 0.00}
\definecolor{shockingpink}            {rgb}{0.99, 0.06, 0.75}
\definecolor{schoolbusyellow}         {rgb}{1.00, 0.85, 0.00}
\definecolor{saddlebrown}             {rgb}{0.55, 0.27, 0.07}
\definecolor{purple(munsell)}         {rgb}{0.62, 0.00, 0.77}
\definecolor{portlandorange}          {rgb}{1.00, 0.35, 0.21}
\definecolor{persianrose}             {rgb}{1.00, 0.16, 0.64}
\definecolor{orange(colorwheel)}      {rgb}{1.00, 0.50, 0.00}
\definecolor{lightcyan}               {rgb}{0.88, 1.00, 1.00}
\definecolor{goldenyellow}            {rgb}{1.00, 0.87, 0.00}
\definecolor{electricindigo}          {rgb}{0.44, 0.00, 1.00}
\definecolor{darkorange}              {rgb}{1.00, 0.55, 0.00}
\definecolor{cyan(process)}           {rgb}{0.00, 0.72, 0.92}
\definecolor{aqua}                    {rgb}{0.00, 1.00, 1.00}
\definecolor{aquamarine}              {rgb}{0.50, 1.00, 0.83}
\definecolor{amber}                   {rgb}{1.00, 0.75, 0.00}
\definecolor{aliceblue}               {rgb}{0.94, 0.97, 1.00}


%===== specify main colors =====
\iftoggle{altColors}{%
  \colorlet{fgColorTheme}{bostonuniversityred}
  \colorlet{bgColorTheme}{gray}
  \colorlet{fgColorStruc}{royalazure}
  \colorlet{bgColorStruc}{bgColorTheme}
  \colorlet{fgColorAlert}{bostonuniversityred}
  \colorlet{bgColorAlert}{fgColorAlert}
  \colorlet{fgColorExamp}{indiagreen}
  \colorlet{bgColorExamp}{fgColorExamp}
}{%
  \colorlet{fgColorTheme}{royalazure}
  \colorlet{bgColorTheme}{gray}
  \colorlet{fgColorStruc}{royalazure}
  \colorlet{bgColorStruc}{bgColorTheme}
  \colorlet{fgColorAlert}{bostonuniversityred}
  \colorlet{bgColorAlert}{fgColorAlert}
  \colorlet{fgColorExamp}{indiagreen}
  \colorlet{bgColorExamp}{fgColorExamp}
}

% derived colors
\colorlet{fgColorThemedark}    {fgColorTheme!80!black}
\colorlet{fgColorThemeDark}    {fgColorTheme!70!black}
\colorlet{fgColorThemeDARK}    {fgColorTheme!60!black}
\colorlet{fgColorThemeDARKER}  {fgColorTheme!60!black}
\colorlet{fgColorThemeDARKEST} {fgColorTheme!60!black}
\colorlet{bgColorThemedark}    {bgColorTheme!60!white}
\colorlet{bgColorThemelight}   {bgColorTheme!30!white}
\colorlet{bgColorThemeLight}   {bgColorTheme!20!white}
\colorlet{bgColorThemeLIGHT}   {bgColorTheme!15!white}
\colorlet{bgColorThemeLIGHTER} {bgColorTheme!10!white}
\colorlet{bgColorThemeLIGHTEST}{bgColorTheme! 5!white}
\colorlet{fgColorStrucdark}    {fgColorStruc!80!black}
\colorlet{bgColorStruclight}   {bgColorStruc!30!white}
\colorlet{bgColorStrucLight}   {bgColorStruc!20!white}
\colorlet{bgColorStrucLIGHT}   {bgColorStruc!15!white}
\colorlet{bgColorStrucLIGHTER} {bgColorStruc!10!white}
\colorlet{bgColorStrucLIGHTEST}{bgColorStruc! 5!white}
\colorlet{fgColorAlertdark}    {fgColorAlert!80!black}
\colorlet{bgColorAlertlight}   {bgColorAlert!30!white}
\colorlet{bgColorAlertLight}   {bgColorAlert!20!white}
\colorlet{bgColorAlertLIGHT}   {bgColorAlert!15!white}
\colorlet{bgColorAlertLIGHTER} {bgColorAlert!10!white}
\colorlet{bgColorAlertLIGHTEST}{bgColorAlert! 5!white}
\colorlet{fgColorExampdark}    {fgColorExamp!80!black}
\colorlet{bgColorExamplight}   {bgColorExamp!30!white}
\colorlet{bgColorExampLight}   {bgColorExamp!20!white}
\colorlet{bgColorExampLIGHT}   {bgColorExamp!15!white}
\colorlet{bgColorExampLIGHTER} {bgColorExamp!10!white}
\colorlet{bgColorExampLIGHTEST}{bgColorExamp! 5!white}
% theme styles
\setbeamercolor*{palette primary}           {fg=fgColorThemeDARK,     bg=bgColorThemelight}
\setbeamercolor*{palette secondary}         {fg=fgColorThemeDark,     bg=bgColorThemeLIGHT}
\setbeamercolor*{palette tertiary}          {bg=fgColorThemedark,     fg=bgColorThemeLIGHTER}
\setbeamercolor*{palette quaternary}        {bg=fgColorTheme,         fg=white}
\setbeamercolor*{sidebar}                   {fg=fgColorTheme,         bg=bgColorThemeLIGHT}
\setbeamercolor*{palette sidebar primary}   {fg=fgColorThemeDARKEST}
\setbeamercolor*{palette sidebar secondary} {fg=white}
\setbeamercolor*{palette sidebar tertiary}  {fg=fgColorThemeDARKER}
\setbeamercolor*{palette sidebar quaternary}{fg=bgColorThemeLIGHTER}
\setbeamercolor*{separation line}           {}
\setbeamercolor*{fine separation line}      {}
% body text
\setbeamercolor {section in toc}            {fg=black,                bg=white}
\setbeamercolor {titlelike}                 {fg=bgColorThemeLIGHT,    bg=fgColorThemedark}
\setbeamercolor {frametitle}                {fg=fgColorThemedark,     bg=bgColorThemeLIGHT}
\setbeamercolor {frametitle right}          {fg=fgColorThemedark,     bg=bgColorThemedark}
\setbeamercolor {structure}                 {fg=fgColorStrucdark}
\setbeamercolor {alerted text}              {fg=fgColorAlertdark}
\setbeamercolor {example text}              {fg=fgColorExampdark}
% blocks
\iftoggle{useColorBlocks}{
  \setbeamercolor {block title}               {bg=bgColorStrucLIGHTER}
  \setbeamercolor {block body}                {bg=bgColorStrucLIGHTEST}
  \setbeamercolor {block title alerted}       {bg=bgColorAlertLIGHTER}
  \setbeamercolor {block body alerted}        {bg=bgColorAlertLIGHTEST}
  \setbeamercolor {block title example}       {bg=bgColorExampLIGHTER}
  \setbeamercolor {block body example}        {bg=bgColorExampLIGHTEST}
  \iftoggle{useColorBlockTitles}{
    \iftoggle{useInverseBlockTitles}{
      % nicely shaded blocks with inverse block titles (original weighting "75" and "10")
      \setbeamercolor{block title}        {use=structure,   fg=white,bg=structure.fg!100!black}
      \setbeamercolor{block title alerted}{use=alerted text,fg=white,bg=alerted text.fg!100!black}
      \setbeamercolor{block title example}{use=example text,fg=white,bg=example text.fg!100!black}
      \setbeamercolor{block body}         {parent=normal text,use=block title,        bg=block title.bg!5!bg}
      \setbeamercolor{block body alerted} {parent=normal text,use=block title alerted,bg=block title alerted.bg!5!bg}
      \setbeamercolor{block body example} {parent=normal text,use=block title example,bg=block title example.bg!5!bg}
    }{
      % shaded blocks with somewhat darker block titles
      \setbeamercolor{block body}         {parent=normal text,use=structure,   bg=structure.fg!5!bg}
      \setbeamercolor{block body alerted} {parent=normal text,use=alerted text,bg=alerted text.fg!5!bg}
      \setbeamercolor{block body example} {parent=normal text,use=example text,bg=example text.fg!5!bg}
      \setbeamercolor{block title}        {use=structure,   bg=structure.fg!10!bg}
      \setbeamercolor{block title alerted}{use=alerted text,bg=alerted text.fg!10!bg}
      \setbeamercolor{block title example}{use=example text,bg=example text.fg!10!bg}
    }
  }{
    % shaded blocks without extra shaded block titles
    \setbeamercolor{block body}         {parent=normal text,use=structure,   bg=structure.fg!5!bg}
    \setbeamercolor{block body alerted} {parent=normal text,use=alerted text,bg=alerted text.fg!5!bg}
    \setbeamercolor{block body example} {parent=normal text,use=example text,bg=example text.fg!5!bg}
    \setbeamercolor{block title}        {use=block body,        bg=block body.bg}
    \setbeamercolor{block title alerted}{use=block body alerted,bg=block body alerted.bg}
    \setbeamercolor{block title example}{use=block body example,bg=block body example.bg}
  }
}{
  % do nothing here
}

% remove shading between backgrounds of block title and block body
\makeatletter
\pgfdeclareverticalshading[lower.bg,upper.bg]{bmb@transition}{200cm}{%
  color(0pt)=(lower.bg); color(2pt)=(lower.bg); color(4pt)=(lower.bg)}
\makeatother




%===== outer theme (based on outer theme "infolines") =====
\makeatletter
% color palette usage
\setbeamercolor*{author     in head/foot}{parent=palette tertiary}
\setbeamercolor*{title      in head/foot}{parent=palette secondary}
\setbeamercolor*{date       in head/foot}{parent=palette primary}
\setbeamercolor*{section    in head/foot}{parent=palette tertiary}
\setbeamercolor*{subsection in head/foot}{parent=palette primary}
% header and footer
\setbeamertemplate{headline}{%
   \leavevmode%
   \hbox{%
     \begin{beamercolorbox}%
     [wd=\paperwidth,ht=2.5ex,dp=0.75ex,left]%
     {author in head/foot}%
       \usebeamerfont{author in head/foot}%
       \hspace*{1.5em}%
       \insertsection%
       \ifdefempty{\insertsubsection}{}{~/~\insertsubsection%
         \ifdefempty{\insertsubsubsection}{}{~/~\insertsubsubsection}%
       }%
     \end{beamercolorbox}%
   }%
   \vskip0pt%
}
\setbeamertemplate{footline}
{
  \leavevmode%
  \hbox{%
    \begin{beamercolorbox}%
    [wd=\paperwidth,ht=2.5ex,dp=0.75ex]%
    {date in head/foot}%
      \usebeamerfont{date in head/foot}%
      \hspace*{1.5em}%
      \insertshortauthor~(\insertshortinstitute)%
      ~~---~~%
      \insertshorttitle%
      \ifdefempty{\insertpart}{}{{:~~}\insertpart}%
      \hfill%
      %\ifdefempty{\insertpart}{}{\insertpartframenumber{}~/~\inserttotalpartframenumber}
      \insertpartframenumber{}~/~\inserttotalpartframenumber
      \hspace*{1.5em}%
    \end{beamercolorbox}%
  }%
  \vskip0pt%
}
% actual usage area
\setbeamersize{text margin left=1em,text margin right=1em}
% navigation symbols
\iftoggle{useNavSymbols}{
  \setbeamertemplate{navigation symbols}{%
    \insertframenavigationsymbol%
    \insertsubsectionnavigationsymbol%
    \insertsectionnavigationsymbol%
    \insertbackfindforwardnavigationsymbol%
  }
}{
  \setbeamertemplate{navigation symbols}{}
}
\makeatother


%===== inner theme (based on inner theme "rounded") =====
\setbeamertemplate{blocks}[rounded][shadow=\usetoggle{useShadows}]
\setbeamertemplate{sections/subsections in toc}[circle]
\setbeamertemplate{title page}[default][colsep=-4bp,rounded=true,shadow=\usetoggle{useShadows}]
\setbeamertemplate{part page}[default][colsep=-4bp,rounded=true,shadow=\usetoggle{useShadows}]

% between blocks
\newlength{\addbegblockskipamount}\setlength{\addbegblockskipamount}{0.0ex}
\newlength{\addendblockskipamount}\setlength{\addendblockskipamount}{0.0ex}
\addtobeamertemplate{block begin}        {\vskip\addbegblockskipamount}{}
\addtobeamertemplate{block end}        {}{\vskip\addendblockskipamount}
\addtobeamertemplate{block alerted begin}{\vskip\addbegblockskipamount}{}
\addtobeamertemplate{block alerted end}{}{\vskip\addendblockskipamount}
\addtobeamertemplate{block example begin}{\vskip\addbegblockskipamount}{}
\addtobeamertemplate{block example end}{}{\vskip\addendblockskipamount}

% formulas inside blocks
\addtobeamertemplate{frame begin}        {}{
  \setlength{\abovedisplayskip}{2ex plus 1ex minus 1.5ex}
  \setlength{\belowdisplayskip}{2ex plus 1ex minus 2.5ex}
}
\addtobeamertemplate{block begin}        {}{
  \setlength{\abovedisplayskip}{2ex plus 1ex minus 1.5ex}
  \setlength{\belowdisplayskip}{2ex plus 1ex minus 1.5ex}
}
\addtobeamertemplate{block alerted begin}{}{
  \setlength{\abovedisplayskip}{2ex plus 1ex minus 1.5ex}
  \setlength{\belowdisplayskip}{2ex plus 1ex minus 1.5ex}
}
\addtobeamertemplate{block example begin}{}{
  \setlength{\abovedisplayskip}{2ex plus 1ex minus 1.5ex}
  \setlength{\belowdisplayskip}{2ex plus 1ex minus 1.5ex}
}

% set default itemize and enumeration
\setbeamertemplate{itemize item}[square]
\setbeamertemplate{itemize subitem}[circle]
\setbeamertemplate{itemize subsubitem}[triangle]
\setbeamertemplate{enumerate items}[auto]
\setbeamertemplate{enumerate mini template}[blank]
\enumAutoDefault{square}{circle}{bracket}
\AfterEndPreamble{%
  \enumStylesDefault{\arabic}{\alph}{\roman}%
}





%===== abbreviations for often used stuff  =====

% begin/end
  \newcommand{\bblk}{\begin{block}}
  \newcommand{\eblk}{\end{block}}
  \newcommand{\bablk}{\begin{alertblock}}
  \newcommand{\eablk}{\end{alertblock}}
  \newcommand{\beblk}{\begin{exampleblock}}
  \newcommand{\eeblk}{\end{exampleblock}}
  \newcommand{\bnest}{\begin{nesting}}
  \newcommand{\enest}{\end{nesting}}
  \newcommand{\bsblk}{\begin{nesting}\begin{block}}
  \newcommand{\esblk}{\end{block}\end{nesting}}
  \newcommand{\basblk}{\begin{nesting}\begin{alertblock}}
  \newcommand{\easblk}{\end{alertblock}\end{nesting}}
  \newcommand{\besblk}{\begin{nesting}\begin{exampleblock}}
  \newcommand{\eesblk}{\end{exampleblock}\end{nesting}}
  \newcommand{\bit}{\begin{itemize}}
  \newcommand{\eit}{\end{itemize}}
  \newcommand{\ben}{\begin{enumerate}}
  \newcommand{\een}{\end{enumerate}}
  \newcommand{\beq}{\begin{equation}}
  \newcommand{\eeq}{\end{equation}}
  \newcommand{\beqn}{\begin{equation*}}
  \newcommand{\eeqn}{\end{equation*}}
  \newcommand{\beqa}{\begin{eqnarray}}
  \newcommand{\eeqa}{\end{eqnarray}}
  \newcommand{\beqan}{\begin{eqnarray*}}
  \newcommand{\eeqan}{\end{eqnarray*}}
  \newcommand{\bmin}{\begin{minipage}}
  \newcommand{\emin}{\end{minipage}}
  
% general formatting
  \newcommand{\loud}[1]{\STRUC{#1}}

% math formatting
  \newcommand{\R}{\mathbb{R}}
\renewcommand{\Pr}[1]{\mathrm{P}\!\left(#1\right)}
  \newcommand{\EV}[1]{\mathrm{E}\!\left\{\,#1\,\right\}}
  \newcommand{\EVX}[2]{\mathrm{E}_{#1}\!\left\{\,#2\,\right\}}
  \newcommand{\func}[2]{\mathrm{#1}\!\left(#2\right)}
  \newcommand{\set}[1]{{\mathcal{#1}}}
\renewcommand{\implies}{\quad\Longrightarrow\quad}
\renewcommand{\equiv}{\quad\Longleftrightarrow\quad}
  \newcommand{\cond}{\,|\,}
  \newcommand{\trans}{^{\mathrm{T}}}
  \newcommand{\bin}{_{\mathrm{b}}}
  \newcommand{\nonl}{\nonumber\\}
\renewcommand{\d}{\mathrm{d}}
  \newcommand{\ve}[1]{{\boldsymbol{#1}}} %{{\mathbf{#1}}}
  \newcommand{\im}{\mathrm{i}}
  \newcommand{\mnorm}[1]{\left\lVert#1\right\rVert}
  \newcommand{\bigmnorm}[1]{\big\lVert#1\big\rVert}
  \newcommand{\Bigmnorm}[1]{\Big\lVert#1\Big\rVert}
  \newcommand{\sha}{\ensuremath{\mathop{\text{\normalfont\fontencoding{T2A}\selectfont ш}}}}
  \newcommand{\Sha}{\ensuremath{\mathop{\text{\normalfont\fontencoding{T2A}\selectfont Ш}}}}
  
% some colors
\newcommand{\cx}[1]{\textcolor{myred}{#1}}
\newcommand{\ca}[1]{\textcolor{myblue}{#1}}
\newcommand{\cb}[1]{\textcolor{myviolet}{#1}}
\newcommand{\cc}[1]{\textcolor{mygreen}{#1}}
\newcommand{\cd}[1]{\textcolor{myorange}{#1}}
\newcommand{\ce}[1]{\textcolor{cyan(process)}{#1}}
\newcommand{\cf}[1]{\textcolor{saddlebrown}{#1}}
\newcommand{\cg}[1]{\textcolor{cadmiumgreen}{#1}}
\newcommand{\ch}[1]{\textcolor{deepcarmine}{#1}}
\newcommand{\ci}[1]{\textcolor{raspberry}{#1}}

\colorlet{myred}{bostonuniversityred}
\colorlet{mygreen}{officegreen!90!black}
\colorlet{myblue}{blue!80!black}
\colorlet{myorange}{orange(colorwheel)!90!black}
\colorlet{myviolet}{vividviolet}
\colorlet{myvvgray}{gray!10}
\colorlet{myvgray}{gray!20}
\colorlet{mylgray}{gray!40}
\colorlet{myngray}{gray}
\colorlet{mydgray}{gray!90!black}

\newcommand{\p}{\;\%}
\newcommand{\clr}[1]{{\color{myred}#1}}
\newcommand{\clb}[1]{{\color{myblue}#1}}
\newcommand{\clg}[1]{{\color{mygreen}#1}}
\newcommand{\clo}[1]{{\color{myorange}#1}}
\newcommand{\clv}[1]{{\color{myviolet}#1}}
\newcommand{\clz}[1]{{\color{black}#1}}
\newcommand{\clm}[1]{{\color{mylgray}#1}}
\newcommand{\cln}[1]{{\color{myngray}#1}}

% citations
\renewcommand{\cite}[1]{{\,\relsize{-1}\color{mygreen}[\,#1\,]}}




%===== define functions for plotting  =====
\makeatletter
\pgfmathdeclarefunction{erf}{1}{%
  \begingroup
    \pgfmathparse{#1 > 0 ? 1 : -1}%
    \edef\sign{\pgfmathresult}%
    \pgfmathparse{abs(#1)}%
    \edef\x{\pgfmathresult}%
    \pgfmathparse{1/(1+0.3275911*\x)}%
    \edef\t{\pgfmathresult}%
    \pgfmathparse{%
      1 - (((((1.061405429*\t -1.453152027)*\t) + 1.421413741)*\t 
      -0.284496736)*\t + 0.254829592)*\t*exp(-(\x*\x))}%
    \edef\y{\pgfmathresult}%
    \pgfmathparse{(\sign)*\y}%
    \pgfmath@smuggleone\pgfmathresult%
  \endgroup
}
\pgfmathdeclarefunction{geopmf}{2}{%
  \pgfmathparse{#1*(1-#1)^#2}%
}
\pgfmathdeclarefunction{geocmf}{2}{%
  \pgfmathparse{1-(1-#1)^(#2+1)}%
}
\pgfmathdeclarefunction{geopmfx}{2}{%
  \pgfmathparse{#1*(1-#1)^(#2-1)}%
}
\pgfmathdeclarefunction{geocmfx}{2}{%
  \pgfmathparse{1-(1-#1)^(#2)}%
}
\pgfmathdeclarefunction{binompmf}{3}{%
  \pgfmathparse{(#1!)/((#3)!*(#1-#3)!)*#2^#3*(1-#2)^(#1-#3)}%
}
\makeatother





%%===== definitions for syntax highlighting =====
%\makeatletter
%\newcommand\ltiny{\@setfontsize\ltiny\@vipt\@viipt}
%\makeatother
%\lstset{%
%  language=C++,
%  backgroundcolor=\color{myvvgray},frame=single,framerule=0pt,
%  basicstyle=\ttfamily\ltiny,
%  keywordstyle=\color{myblue},
%  stringstyle=\color{myred},
%  commentstyle=\color{mygreen},
%  morecomment=[l][\color{myviolet}]{\#}
%}




%%%%%%%%% preliminary slide(s) %%%%%%%%
\newcommand{\preliminaryInfo}{%
\iftoggle{specialHeiko}{%
  \begin{frame}{Organization}
  \vspace{-1ex}
  \begin{tabular}{ll}
  &\\[-2ex]
  Lecture:    & Monday 16:15-17:45\\
              & Room SR 006 / T9\\
  \\[-.5ex]
  Exercise:   & Monday 14:30-16:00\\
              & Room SR 006 / T9\\
  \\[-.5ex]
  Web page:   & \color{blue}\url{http://iphome.hhi.de/schwarz/DC.htm}\\
  \\[.5ex]
  Literature: & \\
  \end{tabular}
  \bit\relsize{-1}
  \item
  Sayood, K. (2018), ``{\bf Introduction to Data Compression},''
  Morgan Kaufmann, Cambridge, MA.
  \item\smallskip
  Cover, T. M. and Thomas, J. A. (2006), ``{\bf Elements of Information Theory},''
  John Wiley \& Sons, New York.
  \item\smallskip
  Gersho, A. and Gray, R. M. (1992), ``{\bf Vector Quantization and Signal Compression},''\\
  Kluwer Academic Publishers, Boston, Dordrecht, London. 
  \item\smallskip
  Jayant, N. S. and Noll, P. (1994), ``{\bf Digital Coding of Waveforms},''
  Prentice-Hall, Englewood Cliffs, NJ, USA.  
  \item\smallskip
  Wiegand, T. and Schwarz, H. (2010), ``{\bf Source Coding: Part I of Fundamentals of Source and Video Coding},''
  Foundations and Trends in Signal Processing, vol.~4, no.~1-2.~~~(\ital{pdf available on course web page})
  \eit
  \end{frame}
}{}


\iftoggle{specialThomas}{%
%  \begin{frame}{Organization}
%  \begin{tabular}{ll}
%  &\\[-1ex]
%  Lecture: & Thursday 10:15-11:45\\
%              & Zoom \\
%  \\[-.5ex]
% % Material:   & \color{blue}\url{http://www.ic.tu-berlin.de/menue/studium_und_lehre/}\\
% Contact:   & Jonathan Pfaff, jonathan.pfaff@hhi.fraunhofer.de \\
%
%  \\[-.5ex]
%  Literatur:  & \\
%  \end{tabular}
%  \bit
%  \item
%  Sayood, K. (2018), ``Introduction to Data Compression,''
%  Morgan Kaufmann, Cambridge, MA.
%  \item
%  Cover, T. M. and Thomas, J. A. (2006), ``Elements of Information Theory,''
%  John Wiley \& Sons, New York.
%  \item
%  Gersho, A. and Gray, R. M. (1992), ``Vector Quantization and Signal Compression,''
%  Kluwer Academic Publishers, Boston, Dordrecht, London. 
%  \item
%  Jayant, N. S. and Noll, P. (1994), ``Digital Coding of Waveforms,''
%  Prentice-Hall, Englewood Cliffs, NJ, USA.  
%  \item
%  Wiegand, T. and Schwarz, H. (2010), ``Source Coding: Part I of Fundamentals of Source and Video Coding,''
%  Foundations and Trends in Signal Processing, vol.~4, no.~1-2.~~~(\ital{pdf available on course web page})
%  \eit
%  \end{frame}
%}{}
  \begin{frame}{Motivation and Goal of the lecture}
  \textbf{Data compression:}
  \begin{itemize}
  \item Data compression is a core technology for modern communication. 
  \item Demand for transmitting large amounts of data increases.
  \item But: Memory and transmission capabilities are limited $\rightarrow$ Data compression algorithms are needed. 
  \end{itemize}
  
  \textbf{Goal of the lecture}: 
  \begin{itemize}
  \item Explain the fundamental problems and some fundamental principles of data compression.
  \item Explain how practical compression algorithm used in billions of devices worldwide work.
  \item Show that there are strong relations between the field of compression and the field of machine learning. 
%  \item Give interested students the opportunity to start doing research work in the field of (video) compression. 
%  \begin{itemize}
%  \item Opportunities to work as a student researcher in compression at Fraunhofer HHI.  
%  \item Interesting topics for Master Thesis projects. 
%  \end{itemize}
   \end{itemize}
  \end{frame}
}{}
%}

  \begin{frame}{Organization}
 \textbf{Lecture:}
  \begin{itemize}
  \item Thursday 10:15-11:45. 
 % \item Due to current situation, lecture will be held via Zoom. 
  \end{itemize}
  
  \textbf{Contact:} 
  \begin{itemize}
  \item Thomas Wiegand: thomas.wiegand@tu-berlin.de, thomas.wiegand@hhi.fraunhofer.de, Chair of Media Technology, TU Berlin, and Head of Fraunhofer HHI 
  \item Jennifer Rasch: j.rasch@protonmail.com
  \item Information about work of research group: 
  \color{blue}\url{https://www.hhi.fraunhofer.de/en/departments/vca/research-groups/video-coding-technologies.html}
   \end{itemize}
   
  \textbf{Important:}
  \begin{itemize}
  \item Introduce yourselves: tell us your name, semester, field of study and motivation
  \item Suggestions: Ask questions during or after the lecture or send informal email to Jennifer Rasch. Via email, informal  appointments for additional communication can be made. Master thesis are possible here :)
   \end{itemize}
  \end{frame}
%}{}
}



%%%%%%%%% end of style file %%%%%%%%%



\DeclareMathOperator{\cwd}{codeword}
\newtheorem{proposition}{Proposition}
\usepackage{forest}
\usepackage{lipsum}
\usepackage{subcaption}
\usepackage{mathtools}
\begin{document}

\section{RD-theory V} 
\begin{frame}
 \vspace{12.0ex}
\begin{center}
\begin{beamercolorbox}[sep=12pt,center]{part title}
\usebeamerfont{section title}\insertsection\par
\end{beamercolorbox}
\end{center}
\end{frame}


\subsection{The Shannon lower bound}
\begin{frame}{Fundamental source coding theorem for general sources}
Let $X$ be an arbitrary $\mathbb{R}$-valued source. 
\bit
\item Informational rate distortion function: 
\begin{align*}
R^{(I)}(D):=\inf\bigl\{&I(X;\hat{X})\colon  \text{$\hat{X}$ is an $\mathbb{R}$-valued r. v.  and there exists an $\mathbb{R}^2$-valued r. v. $Z=(Z_1,Z_2)$}\\ &\text{on some probability space with probability measure $\nu$ such that $Z_1\stackrel{d}{=} X$ and $Z_2\stackrel{d}{=} \hat{X}$} \\ &\text{and such that
$\int_{\mathbb{R}^2}d(x,\hat{x})d\nu_Z(x,\hat{x})<D$. }
\bigr\}
\end{align*} 
\item Informational rate distortion function is lower bound for rate-distortion function: One has
\begin{align*}
r(Q_N)\geq R^{(I)}(\delta(Q_N))
\end{align*}
for every source code $Q_N$ for $X^N$.  
\item Lower bound is asymptotically achievable: For every $D>0$, every $\epsilon>0$ and every $R'>R^{(I)}(D)$ one can achieve
\begin{align*}
r(Q_N)\leq R',\quad \delta(Q_N)\leq D+\epsilon 
\end{align*}
for some sequence $Q_N$ of source codes for $X_N$. 
\eit
\end{frame}

\begin{frame}{The Shannon Lower Bound}
\bit
%\item Explicit computation of (informational) rate distortion function not possible in general (but possible for Gaussian sources). 
\item Use MSE-distortion measure, i.e. $d(x,x')=(x-x')^2$ on $\mathbb{R}$, additive extension to $\mathbb{R}^N$.   
\item If $X$ has finite differential entropy, let
\[
R_{SLB}(D):=h(X)-\frac{1}{2}\log_2(\pi e D).
\]
\item $R_{SLB}(D)$ often used as approximation for rate-distortion performance: 
\eit
\begin{theorem}[Shannon Lower Bound]
\bit
\item If $X$ has finite idfferential entropy, then  
\begin{align*}
R(D)\geq R_{SLB}(D). 
\end{align*}
\item If additionally $X$ has a density and $\mathbb{E}(|X|^\alpha)<\infty$ for some $\alpha>0$, then Shannon lower bound is \loud{asymtptotically tight for small distortion}:
\[
\lim_{D\to 0}(R(D)-R_{SLB}(D))=0. 
\]
\eit
\end{theorem}

\end{frame}

\begin{frame}{Proof of Shannon lower bound}
\loud{Proof of Shannon lower bound:} 
\bit
\item Argue similar as for computation of RD-function for a Gaussian source. 
\item Let $\hat{X}$ be any random variable such that $X, \hat{X}$ satifsy the distortion constraint 
as in the fundamental source coding theorem.
\item One has 
\begin{align*}
I(X;\hat{X})=h(X)-h(X|\hat{X})
= h(X)-h(X|\hat{X})
= & h(X)-h(X-\hat{X}|\hat{X})\\
\geq & h(X)-h(X-\hat{X})\\
\geq & h(X)-\sup_{Z\colon var(Z)\leq D} h(Z). 
\end{align*}
\item Last lecture: Gaussian distribution maximes differential entropy for a given variance, i.e. 
\begin{align}
\sup_{Z\colon var(Z)\leq D} h(Z)= \frac{1}{2}\log_2\left(\pi D e\right). \qed
\end{align}
\eit
\loud{Proof of asmpytotic achievability:} Not done here, see Linder, Zamier 1994, also Koch 2015.
\end{frame}


\subsection{Fundamental source coding theorem for $M$ independent Gaussian sources}
\begin{frame}{Fundamental source coding theorem for vector valued sources}
Straight-forward extension to $\mathbb{R}^M$-valued sources: 
\bit
\item Let $X=(X_1,\dots,X_M)$ be an $\mathbb{R}^M$-valued source.
\item Let $d$ be a distortion function on $\mathbb{R}$
\item Additive extension $d$ to $\mathbb{R}^M$ by averaging over the componentes.
\item Let $Q$ be a source code for $X$.
\item Distortion $\delta(Q)$ of $Q$ taken with respect to $d$.
\item Rate $r(Q)$ of $Q$: Average expected code-length, divided by $M$. 
\item Informational rate-distortion function $R^{(I)}(D)$ for $X$ defined analogous to case of $\mathbb{R}$-valued $X$. 
\eit
Fundamental source coding for $X$:
\bit
\item Let $X^N=(X^1,\dots,X^N)$ be sequence of $N$ independent realizations $X^i=(X_1^i,\dots,X_M^i)$ of $X$.
\item Informational rate-distortion function is lower bound for rate-distortion function for each $X^N$.
\item Lower bound is asymptotically achievable for large $N$. 
\eit
\end{frame}

\begin{frame}{Mutual information for independent components}
\begin{lemma}
Let $X=(X_1,\dots,X_M)$ and $\hat{X}=(\hat{X}_1,\dots,\hat{X}_M)$ be $\mathbb{R}^M$-valued source. Assume that the $X_i$ are independent. 
Then one has 
\begin{align*}
I(X;\hat{X})\geq \sum_{i=1}^MI(X_i;\hat{X}_i). 
\end{align*}
\end{lemma} 
\bit
\item Suffices to assume that $X$ and $\hat{X}$ are discrete.
\item General case: Follows from the discrete case by passing to the limit. 
\eit
\end{frame}
\begin{frame}{Mutual information for independent components. Proof.}
Using the chain rule and that conditioning does not increase entropy, it follows that: 
\begin{align*}
I(X;{\hat{X}})=&H(X)-H(X|\hat{X})\\
=&\sum_{i=1}^MH(X_i)-\sum_{i=1}^MH(X_i|X_1,\dots,X_{i-1},\hat{X})\\
\geq & \sum_{i=1}^MH(X_i)-\sum_{i=1}^MH(X_i|{\hat{X}}_i)\\
=&\sum_{i=1}^M I(X_i;\hat{X}_i).\qed 
\end{align*}
\end{frame}

\begin{frame}{Gaussian sources}
\bit
\item Gaussian kernel: 
\begin{align*}
\phi_{\sigma}(x):=\frac{1}{\sqrt{2\pi\sigma^2}}\exp\left(-\frac{x^2}{2\sigma^2}\right).
\end{align*}
\item Gaussian source: $X$ is Gaussian of mean $\mu$ and variance $\sigma$, $X\stackrel{d}{=} \mathcal{N}(\mu,\sigma^2)$, if 
for every Borel-set $A\subseteq \mathbb{R}$ one has
\begin{align*}
\mu_X(A)=\int_A\phi_\sigma(x-\mu)dx,
\end{align*}
\item Rate-distortion function for Gaussian source: If $X\stackrel{d}{=} \mathcal{N}(\mu,\sigma^2)$ one has 
\begin{align}\label{RDGauss}
R^{(I)}(D)=\begin{cases} &0, \text{if $D\geq \sigma^2$} \\ &\frac{1}{2}\log_2\left(\frac{\sigma^2}{D}\right), \text{else.} \end{cases}
\end{align}
\eit
\end{frame}






\begin{frame}{Rate distortion function for independent Gaussian sources}
\begin{proposition}[Rate distortion function for $M$ independent Gaussian sources]
Let $X=(X_1,\dots,X_M)$ be an $\mathbb{R}^M$-valued source such that the $X_i$ are independent 
and such that 
\begin{align*}
X_i\sim \mathcal{N}(\mu_i,\sigma_i^2). 
\end{align*}
For $\theta>0$ let 
\begin{align*}
R_i(\theta):=&\max(0,\frac{1}{2}\log_2(\sigma_i^2/\theta)), \\ 
\quad D_i(\theta):= &\min(\theta,\sigma_i^2).
\end{align*}
Then the rate-distortion function for $X$ is comprised by the points
\begin{align*}
R(\theta)=\frac{1}{M}\sum_{i=1}^MR_i(\theta),\quad D(\theta)=\frac{1}{M}\sum_{i=1}^MD_i(\theta). 
\end{align*}
\end{proposition}

Previous proposition often reffered to as \loud{reverse water filling}.
Parameter $\theta$ then called \loud{water level}.
\end{frame}



\begin{frame}{Rate distortion function for independent Gaussian sources. Proof I}
%\loud{Case 1: }
\bit
\item By previous lemma and fundamental source coding theorem, one has
\begin{align*}
R(D)=\inf_{\frac{1}{M}(D_1+\dots+D_M)\leq D}\:\:\frac{1}{M}\sum_{i=1}^MR_i(D_i),
\end{align*}
where $R_i$ is the rate distortion function of $X_i$. 
%\eit
%Assume first that $R_i(D_i)>0$ for all $i$. 
%\bit
\item For convenience, use \textit{distortion-rate function} $D(R)$ of $X$ and $D_i(R_i)$ of $X_i$, i.e. 
\[
D_i(R_i)=\sigma_i^22^{-2R_i}. 
\]
\item One has 
\begin{align*}
D(R)=\inf_{\frac{1}{M}(R_1+\dots+R_M)\leq R}\:\frac{1}{M}\sum_{i=1}^MD_i(R_i). 
\end{align*}
\eit
\end{frame}

\begin{frame}{Rate distortion function for independent Gaussian sources. Proof II}
\bit
\item If $R_{i_1}=\cdots=R_{i_k}=0$, then
$D_{i_1}=\sigma_{i_1}^2,\dots, D_{i_K}=\sigma_{i_K}^2$ and
variances $\sigma_{i_1}^2,\dots,\sigma_{i_K}^2$ have to be the $K$ smallest variances of the $X_1,\dots,X_M$.
\item Thus: Suffices to treat the case where $R_i>0$ for all $i$. 
\item By \loud{inequality between arithmetic and geometric means}:  
\begin{align}\label{InEqArGeo}
\frac{1}{M}\sum_{i=1}^MD_i(R_i) =  \frac{1}{M} \sum_{i=1}	^M\sigma_i^22^{-2R_i} 
&\geq (\prod_{i=1}^M\sigma_i^22^{-2R_i})^{1/M}\nonumber
\\&=(\prod_{i=1}^M\sigma_i^2)^{1/M}2^{-2R}.
\end{align}
\item In \eqref{InEqArGeo}, equality can be achieved if  
\begin{align}\label{CondEq}
\sigma_i^22^{-2R_i}=(\prod_{j=1}^M\sigma_j^2)^{1/M}2^{-2R}.
\end{align}
\eit
\end{frame}

\begin{frame}{Rate distortion function for independent Gaussian sources. Proof III}
\bit
\item Equation \eqref{CondEq} is attained if 
\begin{align*}
\log_2(\sigma_i^2)-2R_i=\frac{1}{M}\sum_{j=1}^M\log_2(\sigma_j^2)-2R,
\end{align*}
i.e. if 
\begin{align}\label{rateGauss}
R_i=\frac{1}{2}\left(\log_2(\sigma_i^2)-\frac{1}{M}\sum_{j=1}^M\log_2(\sigma_j^2)\right)+R
\end{align}
\item If $R_i$ are chosen as in \eqref{rateGauss}, then all $D_i(R_i)$ are equal, and 
\begin{align*}
\frac{1}{M}\sum_{i=1}^MR_i=R.
\end{align*}
 \qed
\eit
\end{frame}


\section{Quantization I} 
\begin{frame}
 \vspace{12.0ex}
\begin{center}
\begin{beamercolorbox}[sep=12pt,center]{part title}
\usebeamerfont{section title}\insertsection\par
\end{beamercolorbox}
\end{center}
\end{frame}
\subsection{Setup}


\begin{frame}{Quantization}
  \begin{center}
    \vspace{-2.0ex}
      \hfitbox{0.90\linewidth}{%
      \begin{tikzpicture}
        \begin{axis}[
            width=8cm,height=2.5cm,
            axis lines=none,
            xmin=-1.2,xmax=10.2,ymin=-1.1,ymax=1.1,
            axis on top,
            xtick=\empty,ytick=\empty,
            enlargelimits=false,scale=1.7,
          ]
          \addplot+[ycomb,mark size=1.5pt,myblue,domain=-1:9.8,samples=50] {sin(x*45)};
          \addplot[black,->,style={-{Latex[length=2.5mm,width=1.2mm]},line width=0.5pt}]  
          coordinates { (-1.2,0) (10.2,0) };
        \end{axis}
        \begin{axis}[
            yshift=-2.5cm,
            width=8cm,height=2.5cm,
            axis lines=none,
            xmin=-1.2,xmax=10.2,ymin=-1.1,ymax=1.1,
            axis on top,
            xtick=\empty,ytick=\empty,
            enlargelimits=false,scale=1.7,
          ]
          \addplot+[ycomb,mark size=1.5pt,myred,mark options={myred},domain=-1:9.8,samples=50] {(floor(0.5-0.5+4*sin(x*45))+0.5)/4.0};
          \addplot[black,->,style={-{Latex[length=2.5mm,width=1.2mm]},line width=0.5pt}]  
          coordinates { (-1.2,0) (10.2,0) };
        \end{axis}
        \draw[-{Latex[length=2.5mm,width=1.2mm]}] (5,0) -- (5,-1.0) node[pos=.5,anchor=west] {\footnotesize quantization};
      \end{tikzpicture}
    }
  \end{center}
  \vspace{-1.0ex}%
  \bit
\item ``Lossy part'' of source coding
\item Non-reversible mapping from input to output samples
\item Determines trade-off between signal fidelity and bit rate
  \eit\vspace{-5ex}
\end{frame}



\begin{frame}{Quantization: Functional Mapping}
\begin{center}
   \hfitbox{0.4\linewidth}{%
    \begin{tikzpicture}
    [
      pstep/.style={
        draw=black,
        fill=myvvgray,
        line width=0.5pt,
        minimum width=1.5cm,
        minimum height=0.8cm,
        inner sep=0.2cm,
        align=center
      },
      nbox/.style={pstep,
        rectangle
      },
      narrow/.style={
        -{Latex[length=2.5mm,width=1.3mm]},
        line width=0.5pt
        },
    ]
    \node[] (quant) [nbox,align=center]    at (0,0) {quantizer $Q$};
    \draw[] [narrow] (quant) -- (2.5,0)
            node (rec) [pos=0.5,anchor=base,yshift=1.5ex,align=center] {$s'$};
    \draw[] [narrow] (-2.5,0) -- (quant)
            node (org) [pos=0.5,anchor=base,yshift=1.5ex,align=center] {$s$};
    \end{tikzpicture}
   }
\end{center}
\medskip
\bit\TabPositions{5em}
\item
  Quantization: Functional mapping of an input sample to an output sample
  $$
    s'=Q(\,s\,)
  $$
\item\medskip Input:
      \tab\struc{\icircle}\;\;Discrete or continuous\\[1ex]
\item\bigskip Output:
      \tab\struc{\icircle}\;\;Set of obtainable output points is countable\\[1ex]
      \tab\struc{\icircle}\;\;Less obtainable output points than input points
\item[\iarrow]\bigskip 
  \loud{Non-reversible loss in signal fidelity}
\item \bigskip Vector quantization: Signals of arbitrary dimension $N$. 
\item Next: Scalar quantization. Signals of dimension $N=1$.  
\eit
\end{frame}



\begin{frame}{Structure of Quantizers: Encoder and Decoder Mapping}
\begin{center}
   \hfitbox{0.6\linewidth}{%
    \begin{tikzpicture}
    [
      pstep/.style={
        draw=black,
        fill=white,
        line width=0.5pt,
        minimum width=2.0cm,
        minimum height=1.2cm,
        inner sep=0,
        align=center
      },
      nbox/.style={pstep,
        rectangle
      },
      narrow/.style={
        -{Latex[length=2.5mm,width=1.3mm]},
        line width=0.5pt
        },
    ]
    \draw[fill=gray!20] (-3.5,-0.85) rectangle (3.5,1.3);
    \node[anchor=north west] at (-3.5,1.3) {quantizer $Q$};
    \node[] (enc) [nbox,align=center]    at (-1.8,0) {encoder\\mapping $\alpha$};
    \node[] (dec) [nbox,align=center]    at ( 1.8,0) {decoder\\mapping $\beta$};
    \draw[] [narrow] (-5,0)  -- (enc)
            node (rpo) [pos=0.3,anchor=base,yshift=1ex,align=center] {${s}$};
    \draw[] [narrow] (enc)  -- (dec)
            node [pos=0.5,anchor=base,yshift=1ex,align=center] {${q}$};
    \draw[] [narrow] (dec)  -- (5,0)
            node (rpr) [pos=0.7,anchor=base,yshift=1ex,align=center] {${s'}$};
    \node[] (rate) [text=blue,align=center]    at (0,-1.5) {quantization indexes};
    \draw[] [narrow,blue] (rate.north)  -- (0,-0.1);
    \end{tikzpicture}
   }
\end{center}
\bit
\item<+-> Split quantizer $Q$ into \loud{encoder mapping $\alpha$} and \loud{decoder mapping $\beta$}
\item<+->\medskip
    Encoder mapping $\alpha$: Maps input sample ${s}$ to a quantizer index $q$ ~(integer)
      $$
      q=\alpha(\,{s}\,)
      $$
\item<+->\medskip
    Decoder mapping $\beta$: Maps quantizer index $q$ to reconstructed samples ${s'}$
      $$
      {s'}=\beta(\,q\,)=\beta(\,\alpha(\,{s}\,)\,)=Q(\,{s}\,)
      $$
\eit\vspace{-5ex}
\end{frame}



\begin{frame}{Quantization and Entropy Coding}
  \begin{center}
   \hfitbox{0.8\linewidth}{%
    \begin{tikzpicture}
    [
      pstep/.style={
        draw=black,
        fill=white,
        line width=0.5pt,
        minimum width=1.0cm,
        minimum height=1.0cm,
        inner sep=0.2cm,
        align=center
      },
      ncloud/.style={pstep,
        cloud,
        cloud puffs=9,
        cloud puff arc=170,
        minimum width=1.0cm,
        minimum height=0.5cm,
        inner sep=0.0cm,
        aspect=2
      },
      nbox/.style={pstep,
        rectangle
      },
      narrow/.style={
        -{Latex[length=2.5mm,width=1.3mm]},
        line width=0.5pt
        },
    ]
    \draw[fill=gray!20] (-3.0,-0.8) rectangle (3.0,1.15);
    \node[anchor=north west] at (-3.0,1.15) {transmission};
    \node[] (ch)   [ncloud]               at ( 0,0) {channel};
    \node[] (lenc) [nbox,align=center]    at (-2,0) {\relsize{1}$\gamma$};
    \node[] (ldec) [nbox,align=center]    at ( 2,0) {\relsize{1}$\gamma^{-1}$};
    \node[] (enc) [nbox,align=center]    at (-4.5,0) {\relsize{1}$\alpha$};
    \node[] (dec) [nbox,align=center]    at ( 4.5,0) {\relsize{1}$\beta$};
    \draw[] [narrow] (-6.2,0)  -- (enc)
            node [pos=0.5,anchor=base,yshift=1ex,align=center] {${s}$};
    \draw[] [narrow] (enc) -- (lenc)
            node [pos=0.4,anchor=base,yshift=1ex,align=center] {${q}$};
    \draw[] [narrow] (lenc)  -- (ch)
            node [pos=0.5,anchor=base,yshift=1ex,align=center] {$\ve{b}$};
    \draw[] [narrow] (ch)  -- (ldec)
            node [pos=0.5,anchor=base,yshift=1ex,align=center] {$\ve{b}$};
    \draw[] [narrow] (ldec) -- (dec)
            node [pos=0.6,anchor=base,yshift=1ex,align=center] {${q}$};
    \draw[] [narrow] (dec)  -- (6.2,0)
            node [pos=0.5,anchor=base,yshift=1ex,align=center] {${s'}$};
    \end{tikzpicture}
   }
\end{center}
\bit
\item<+-> Add \loud{lossless coding $\gamma$} of quantization indexes (e.g., Huffman or arithmetic coding)
\item<+->\bigskip
    Encoding/decoding process:
    \ben\TabPositions{10em}
    \item<+->\smallskip Encoder mapping $\alpha$:
         \tab Input samples ${s}$ $\mapsto$ quantization indexes ${q}$
    \item<+->\smallskip Lossless mapping $\gamma$:
         \tab Quantization indexes ${q}$ $\mapsto$ bitstream $\ve{b}$
    \item<+->\smallskip Transmission channel: 
         \tab Transmission of bitstream (assume: error-free)
    \item<+->\smallskip Lossless mapping $\gamma^{-1}$:
         \tab Bitstream $\ve{b}$ $\mapsto$ quantization indexes ${q}$
    \item<+->\smallskip Decoder mapping $\beta$:
         \tab Quantization indexes ${q}$ $\mapsto$ reconstructed samples ${s'}$
    \een
\eit
\end{frame}

\subsection{Scalar Quantization}

\begin{frame}{Principle of Scalar Quantization}
  \begin{center}
    \vspace{-2.5ex}
  \hfitbox{0.8\linewidth}{%
  \begin{tikzpicture}
     \begin{axis}[
        compat=newest,
        width=14cm,height=5.0cm,
        axis lines=none,
        xmin=-4.5,xmax=4.5,ymin=-0.15,ymax=0.7,
        clip mode=individual, % ensures curves are plot one after another
        after end axis/.code={
          \draw[->,style={-{Latex[length=2.5mm,width=1.2mm]},line width=0.5pt}] 
               (axis cs: -5,0) -- (axis cs: 5,0);
          %\node[black, anchor=north east] at (axis cs: 5,-0.01) {$s$};
        }
     ]
       \addplot[name path=func,black,samples=101,domain=-4.5:4.5,line width=0.5pt,mark=none] {0.5*exp(-abs(x))};
       \path[name path=axis] (axis cs: -4.5,0) -- (axis cs: 4.5,0);
       \node[anchor=north west,yshift=-.5ex] at (axis cs:  4.5,0) {$s$};
       \addplot[blue!20,visible on=<1>]        fill between[of=func and axis,soft clip={domain=-4.5: 4.5}];
       \addplot[visible on=<2->,red!20]         fill between[of=func and axis,soft clip={domain=-0.5: 0.5}];
       \addplot[visible on=<2->,green!20]       fill between[of=func and axis,soft clip={domain=-1.5:-0.5}];
       \addplot[visible on=<2->,green!20]       fill between[of=func and axis,soft clip={domain= 0.5: 1.5}];
       \addplot[visible on=<2->,blue!20]        fill between[of=func and axis,soft clip={domain=-2.5:-1.5}];
       \addplot[visible on=<2->,blue!20]        fill between[of=func and axis,soft clip={domain= 1.5: 2.5}];
       \addplot[visible on=<2->,vividviolet!20] fill between[of=func and axis,soft clip={domain=-3.5:-2.5}];
       \addplot[visible on=<2->,vividviolet!20] fill between[of=func and axis,soft clip={domain= 2.5: 3.5}];
       \addplot[visible on=<2->,cyan!20]        fill between[of=func and axis,soft clip={domain=-4.5:-3.5}];
       \addplot[visible on=<2->,cyan!20]        fill between[of=func and axis,soft clip={domain= 3.5: 4.5}];

       \pgfmathsetmacro{\su}{0.7}
       \pgfmathsetmacro{\ou}{-0.15}
       \begin{scope}[visible on=<2->]
         \addplot[line width=0.5pt,dashed] coordinates { ( 0.5,{\ou}) ( 0.5,{\su}) };
         \addplot[line width=0.5pt,dashed] coordinates { (-0.5,{\ou}) (-0.5,{\su}) };
         \addplot[line width=0.5pt,dashed] coordinates { ( 1.5,{\ou}) ( 1.5,{\su}) };
         \addplot[line width=0.5pt,dashed] coordinates { (-1.5,{\ou}) (-1.5,{\su}) };
         \addplot[line width=0.5pt,dashed] coordinates { ( 2.5,{\ou}) ( 2.5,{\su}) };
         \addplot[line width=0.5pt,dashed] coordinates { (-2.5,{\ou}) (-2.5,{\su}) };
         \addplot[line width=0.5pt,dashed] coordinates { ( 3.5,{\ou}) ( 3.5,{\su}) };
         \addplot[line width=0.5pt,dashed] coordinates { (-3.5,{\ou}) (-3.5,{\su}) };
       \end{scope}
     
       \begin{scope}[visible on=<3->]
         \node[anchor=north,yshift=-.3ex] at (axis cs: -3.5,{\ou}) {$u_{-3}$};
         \node[anchor=north,yshift=-.3ex] at (axis cs: -2.5,{\ou}) {$u_{-2}$};
         \node[anchor=north,yshift=-.3ex] at (axis cs: -1.5,{\ou}) {$u_{-1}$};
         \node[anchor=north,yshift=-.3ex] at (axis cs: -0.5,{\ou}) {$u_{0}$};
         \node[anchor=north,yshift=-.3ex] at (axis cs:  0.5,{\ou}) {$u_{1}$};
         \node[anchor=north,yshift=-.3ex] at (axis cs:  1.5,{\ou}) {$u_{2}$};
         \node[anchor=north,yshift=-.3ex] at (axis cs:  2.5,{\ou}) {$u_{3}$};
         \node[anchor=north,yshift=-.3ex] at (axis cs:  3.5,{\ou}) {$u_{4}$};
       \end{scope}

       \begin{scope}[visible on=<4->]
         \node[anchor=north,black]    at (axis cs: -4.8,{\su}) {$q=\ldots$\phantom{4}};
         \node[anchor=north,cyan]     at (axis cs: -4.0,{\su}) {$\phantom{q}{-4}\phantom{q}$};
         \node[anchor=north,myviolet] at (axis cs: -3.0,{\su}) {$\phantom{q}{-3}\phantom{q}$};
         \node[anchor=north,myblue]   at (axis cs: -2.0,{\su}) {$\phantom{q}{-2}\phantom{q}$};
         \node[anchor=north,mygreen]  at (axis cs: -1.0,{\su}) {$\phantom{q}{-1}\phantom{q}$};
         \node[anchor=north,myred]    at (axis cs: -0.0,{\su}) {$\phantom{q}{0}\phantom{q}$};
         \node[anchor=north,mygreen]  at (axis cs:  1.0,{\su}) {$\phantom{q}{1}\phantom{q}$};
         \node[anchor=north,myblue]   at (axis cs:  2.0,{\su}) {$\phantom{q}{2}\phantom{q}$};
         \node[anchor=north,myviolet] at (axis cs:  3.0,{\su}) {$\phantom{q}{3}\phantom{q}$};
         \node[anchor=north,cyan]     at (axis cs:  4.0,{\su}) {$\phantom{q}{4}\phantom{q}$};
       \end{scope}
       
       \pgfmathsetmacro{\ss}{0.33}
       \pgfmathsetmacro{\os}{0.00}
       \begin{scope}[visible on=<6->]
         \addplot[line width=1.0pt,myred]    coordinates { ( 0.0,{\os}) ( 0.0,{\ss}) };
         \addplot[line width=1.0pt,mygreen]  coordinates { ( 0.9,{\os}) ( 0.9,{\ss}) };
         \addplot[line width=1.0pt,mygreen]  coordinates { (-0.9,{\os}) (-0.9,{\ss}) };
         \addplot[line width=1.0pt,myblue]   coordinates { ( 1.9,{\os}) ( 1.9,{\ss}) };
         \addplot[line width=1.0pt,myblue]   coordinates { (-1.9,{\os}) (-1.9,{\ss}) };
         \addplot[line width=1.0pt,myviolet] coordinates { ( 2.9,{\os}) ( 2.9,{\ss}) };
         \addplot[line width=1.0pt,myviolet] coordinates { (-2.9,{\os}) (-2.9,{\ss}) };
         \addplot[line width=1.0pt,cyan]     coordinates { ( 3.9,{\os}) ( 3.9,{\ss}) };
         \addplot[line width=1.0pt,cyan]     coordinates { (-3.9,{\os}) (-3.9,{\ss}) };

         \draw[fill,myred]    (axis cs:  0.0,{\ss}) circle [radius=2.5pt];
         \draw[fill,mygreen]  (axis cs:  0.9,{\ss}) circle [radius=2.5pt];
         \draw[fill,mygreen]  (axis cs: -0.9,{\ss}) circle [radius=2.5pt];
         \draw[fill,myblue]   (axis cs:  1.9,{\ss}) circle [radius=2.5pt];
         \draw[fill,myblue]   (axis cs: -1.9,{\ss}) circle [radius=2.5pt];
         \draw[fill,myviolet] (axis cs:  2.9,{\ss}) circle [radius=2.5pt];
         \draw[fill,myviolet] (axis cs: -2.9,{\ss}) circle [radius=2.5pt];
         \draw[fill,cyan]     (axis cs:  3.9,{\ss}) circle [radius=2.5pt];
         \draw[fill,cyan]     (axis cs: -3.9,{\ss}) circle [radius=2.5pt];
         
         \node[blue,anchor=north,yshift=-.0ex] at (axis cs: -3.9,{\os}) {$s'_{-4}$};
         \node[blue,anchor=north,yshift=-.0ex] at (axis cs: -2.9,{\os}) {$s'_{-3}$};
         \node[blue,anchor=north,yshift=-.0ex] at (axis cs: -1.9,{\os}) {$s'_{-2}$};
         \node[blue,anchor=north,yshift=-.0ex] at (axis cs: -0.9,{\os}) {$s'_{-1}$};
         \node[blue,anchor=north,yshift=-.0ex] at (axis cs:  0.0,{\os}) {$s'_{0}$};
         \node[blue,anchor=north,yshift=-.0ex] at (axis cs:  0.9,{\os}) {$s'_{1}$};
         \node[blue,anchor=north,yshift=-.0ex] at (axis cs:  1.9,{\os}) {$s'_{2}$};
         \node[blue,anchor=north,yshift=-.0ex] at (axis cs:  2.9,{\os}) {$s'_{3}$};
         \node[blue,anchor=north,yshift=-.0ex] at (axis cs:  3.9,{\os}) {$s'_{4}$};
       \end{scope}
  \end{axis}
  \end{tikzpicture}}
  \end{center}
  \vspace{-2.0ex}
\bit
\item<2-> Partition real line into a countable (typically finite) number of quantization intervals $\set{I}_k$
  \bit
\item<3-> Partitioning is given by decision thresholds $\{u_k\}$
\item<4-> Quantization intervals are labeled by quantization index $q$
\item<5-> A quantization interval is the given by $\set{I}_k=[u_k,u_{k+1})$
  \eit
\item<6->\smallskip Each quantization interval $\set{I}_k$ is associated with a reconstruction level $s'_k\in\set{I}_k$
\item<7->[\iarrow]\medskip\loud{Scalar quantization}: Replace input value $s$ that falls inside $\set{I}_k$ with reconstruction value $s'_k$
\eit\vspace{-10ex}
\end{frame}



\begin{frame}{Scalar Quantization: Input-Output Function}
\vspace{-2.5ex}
\begin{center}
  \hfitbox{0.8\linewidth}{%
  \begin{tikzpicture}
    \draw[->,style={-{Latex[length=2.5mm,width=1.2mm]},line width=0.3pt}] 
         (-3,0) -- (3,0);
    \draw[->,style={-{Latex[length=2.5mm,width=1.2mm]},line width=0.3pt}] 
         (0,-2.5) -- (0,2.5);
    \node[black, anchor=north] at (3,-0.05) {$s$};
    \node[black, anchor=west] at (0.05,2.5) {$s'=Q(s)$};

    \draw[blue,line width=1.5pt] (-0.7,0) -- (0.7,0);
    \draw[blue,line width=1.5pt] ( 0.7,1) -- (1.7,1);
    \draw[blue,line width=1.5pt] ( 1.7,2) -- (3.0,2);
    \draw[blue,line width=1.5pt] (-0.7,-1) -- (-1.7,-1);
    \draw[blue,line width=1.5pt] (-1.7,-2) -- (-3.0,-2);

    \draw[blue,line width=0.3pt] (0.7,0) -- (0.7,1);
    \draw[blue,line width=0.3pt] (1.7,1) -- (1.7,2);
    \draw[blue,line width=0.3pt] (-0.7,0) -- (-0.7,-1);
    \draw[blue,line width=0.3pt] (-1.7,-1) -- (-1.7,-2);

  \begin{scope}[visible on=<2->]
    \draw[dashed,line width=0.3pt] (-1.7,-2) -- (  0,-2);
    \draw[dashed,line width=0.3pt] (-0.7,-1) -- (  0,-1);
    \draw[dashed,line width=0.3pt] ( 0.7, 1) -- (  0, 1);
    \draw[dashed,line width=0.3pt] ( 1.7, 2) -- (  0, 2);
    
    \node[anchor=east ] (ska) at ( 0.0, 1.00) {$s'_{k}$};
    \node[anchor=east ] (skb) at ( 0.0, 2.00) {$s'_{k+1}$};
    \node[anchor=east ] at (-3.0,0.00) {$s'_{k-1}$};
    \node[anchor=west ] at ( 0.0,-1.00) {$s'_{k-2}$};
    \node[anchor=west ] at ( 0.0,-2.00) {$s'_{k-3}$};

    \node[anchor=east,text=myred] (lev) at (-2.5,1.5) {$K$ reconstruction levels};
    \draw[myred,->,style={-{Latex[length=2.5mm,width=1.2mm]},line width=0.3pt}] (lev) -- (ska);
    \draw[myred,->,style={-{Latex[length=2.5mm,width=1.2mm]},line width=0.3pt}] (lev) -- (skb);
  \end{scope}

  \begin{scope}[visible on=<3->]
    \draw[dashed,line width=0.3pt] (-1.7,-2) -- (-1.7,0);
    \draw[dashed,line width=0.3pt] ( 1.7, 2) -- ( 1.7,0);

    \node[anchor=south] at (-1.7,-0.02) {$u_{k-2}$};
    \node[anchor=south] at (-0.7,-0.02) {$u_{k-1}$};
    \node[anchor=north] (uka) at ( 0.7,-0.05) {$u_k$};
    \node[anchor=north] (ukb) at ( 1.7,-0.05) {$u_{k+1}$};

    \node[anchor=west,text=myred] (dec) at (2.5,-1.5) {$K-1$ decision thresholds};
    \draw[myred,->,style={-{Latex[length=2.5mm,width=1.2mm]},line width=0.3pt}] (dec.north west) -- (uka);
    \draw[myred,->,style={-{Latex[length=2.5mm,width=1.2mm]},line width=0.3pt}] (dec.north west) -- (ukb);
  \end{scope}

  \begin{scope}[visible on=<4->]
    \draw[officegreen,<->,style={{Latex[length=2.5mm,width=1.2mm]}-{Latex[length=2.5mm,width=1.2mm]},line width=0.3pt}] 
         (0.7,0.3) -- (1.7,0.3)
         node (delta) [pos=0.5,anchor=base,yshift=1.0ex,align=center] {$\Delta_k$};
    \node[anchor=west,text=officegreen,align=left] (qstep) at (2.5,1.0) {quantization\\step sizes};
    \draw[officegreen,->,style={-{Latex[length=2.5mm,width=1.2mm]},line width=0.3pt}] (qstep) -- (delta);
  \end{scope}
  \end{tikzpicture}}
\end{center}
\vspace{-1.5ex}
\bit\TabPositions{13em}
\item<2-> Scalar quantizer mapping:     \tab$Q:\;\,\mathbb{R}\mapsto\{\,\cdots,\,s'_{k-1},\,s'_k,\,s'_{k+1},\,\cdots\,\}$
\item<3-> Quantization intervals: \tab$\set{I}_k=[u_k,u_{k+1})$
\item<4-> Quantization step sizes:      \tab$\Delta_k=u_{k+1}-u_k$
\eit
\vspace{-5ex}
\end{frame}





\begin{frame}{Scalar Quantization: Discretization of Pdf}
\begin{minipage}{0.55\linewidth}
  \hfitbox{\linewidth}{%
  \begin{tikzpicture}
     \begin{axis}[
        compat=newest,
        width=10cm,height=4.5cm,
        axis lines=none,
        xmin=-4.5,xmax=4.5,ymin=0,ymax=0.51,
        clip mode=individual, % ensures curves are plot one after another
        after end axis/.code={
          \draw[->,style={-{Latex[length=2.5mm,width=1.2mm]},line width=0.5pt}] 
               (axis cs: -5,0) -- (axis cs: 5,0);
          %\node[black, anchor=north east] at (axis cs: 5,-0.01) {$s$};
        }
     ]
     \addplot[name path=func,black,samples=101,domain=-4.5:4.5,line width=0.5pt,mark=none] 
                           {0.5*exp(-abs(x))};
     \path[name path=axis] (axis cs: -4.5,0) -- (axis cs: 4.5,0);
     \addplot[red!20]         fill between[of=func and axis,soft clip={domain=-0.5: 0.5}];
     \addplot[green!20]       fill between[of=func and axis,soft clip={domain=-1.5:-0.5}];
     \addplot[green!20]       fill between[of=func and axis,soft clip={domain= 0.5: 1.5}];
     \addplot[blue!20]        fill between[of=func and axis,soft clip={domain=-2.5:-1.5}];
     \addplot[blue!20]        fill between[of=func and axis,soft clip={domain= 1.5: 2.5}];
     \addplot[vividviolet!20] fill between[of=func and axis,soft clip={domain=-3.5:-2.5}];
     \addplot[vividviolet!20] fill between[of=func and axis,soft clip={domain= 2.5: 3.5}];
     \addplot[cyan!20]        fill between[of=func and axis,soft clip={domain=-4.5:-3.5}];
     \addplot[cyan!20]        fill between[of=func and axis,soft clip={domain= 3.5: 4.5}];

     \node[anchor=north,yshift=-.5ex] at (axis cs: -3.5,0) {$u_{-3}$};
     \node[anchor=north,yshift=-.5ex] at (axis cs: -2.5,0) {$u_{-2}$};
     \node[anchor=north,yshift=-.5ex] at (axis cs: -1.5,0) {$u_{-1}$};
     \node[anchor=north,yshift=-.5ex] at (axis cs: -0.5,0) {$u_{0}$};
     \node[anchor=north,yshift=-.5ex] at (axis cs:  0.5,0) {$u_{1}$};
     \node[anchor=north,yshift=-.5ex] at (axis cs:  1.5,0) {$u_{2}$};
     \node[anchor=north,yshift=-.5ex] at (axis cs:  2.5,0) {$u_{3}$};
     \node[anchor=north,yshift=-.5ex] at (axis cs:  3.5,0) {$u_{4}$};
     \node[anchor=north west,yshift=-.5ex] at (axis cs:  4.5,0) {$s$};
  \end{axis}
  \end{tikzpicture}}\\[4ex]
  \hfitbox{\linewidth}{%
  \begin{tikzpicture}
     \begin{axis}[
        compat=newest,
        width=10cm,height=4.5cm,
        axis lines=none,
        xmin=-4.5,xmax=4.5,ymin=0,ymax=0.4,
        clip mode=individual, % ensures curves are plot one after another
        after end axis/.code={
          \draw[->,style={-{Latex[length=2.5mm,width=1.2mm]},line width=0.5pt}] 
               (axis cs: -5,0) -- (axis cs: 5,0);
          %\node[black, anchor=north east] at (axis cs: 5,-0.01) {$s$};
        }
     ]
     \pgfmathsetmacro{\s}{1}
     \addplot[line width=1.0pt,red]         coordinates { ( 0.0,0) ( 0.0,{\s*0.393469}) };
     \addplot[line width=1.0pt,officegreen] coordinates { ( 0.9,0) ( 0.9,{\s*0.191700}) };
     \addplot[line width=1.0pt,officegreen] coordinates { (-0.9,0) (-0.9,{\s*0.191700}) };
     \addplot[line width=1.0pt,blue]        coordinates { ( 1.9,0) ( 1.9,{\s*0.070522}) };
     \addplot[line width=1.0pt,blue]        coordinates { (-1.9,0) (-1.9,{\s*0.070522}) };
     \addplot[line width=1.0pt,vividviolet] coordinates { ( 2.9,0) ( 2.9,{\s*0.025944}) };
     \addplot[line width=1.0pt,vividviolet] coordinates { (-2.9,0) (-2.9,{\s*0.025944}) };
     \addplot[line width=1.0pt,cyan]        coordinates { ( 3.9,0) ( 3.9,{\s*0.009544}) };
     \addplot[line width=1.0pt,cyan]        coordinates { (-3.9,0) (-3.9,{\s*0.009544}) };
    
     \draw[red,fill]         (axis cs:  0.0,{\s*0.393469}) circle [radius=2pt];
     \draw[officegreen,fill] (axis cs:  0.9,{\s*0.191700}) circle [radius=2pt];
     \draw[officegreen,fill] (axis cs: -0.9,{\s*0.191700}) circle [radius=2pt];
     \draw[blue,fill]        (axis cs:  1.9,{\s*0.070522}) circle [radius=2pt];
     \draw[blue,fill]        (axis cs: -1.9,{\s*0.070522}) circle [radius=2pt];
     \draw[vividviolet,fill] (axis cs:  2.9,{\s*0.025944}) circle [radius=2pt];
     \draw[vividviolet,fill] (axis cs: -2.9,{\s*0.025944}) circle [radius=2pt];
     \draw[cyan,fill]        (axis cs:  3.9,{\s*0.009544}) circle [radius=2pt];
     \draw[cyan,fill]        (axis cs: -3.9,{\s*0.009544}) circle [radius=2pt];

     \node[anchor=north,yshift=-.5ex] at (axis cs: -3.9,0) {$s'_{-4}$};
     \node[anchor=north,yshift=-.5ex] at (axis cs: -2.9,0) {$s'_{-3}$};
     \node[anchor=north,yshift=-.5ex] at (axis cs: -1.9,0) {$s'_{-2}$};
     \node[anchor=north,yshift=-.5ex] at (axis cs: -0.9,0) {$s'_{-1}$};
     \node[anchor=north,yshift=-.5ex] at (axis cs:  0.0,0) {$s'_{0}$};
     \node[anchor=north,yshift=-.5ex] at (axis cs:  0.9,0) {$s'_{1}$};
     \node[anchor=north,yshift=-.5ex] at (axis cs:  1.9,0) {$s'_{2}$};
     \node[anchor=north,yshift=-.5ex] at (axis cs:  2.9,0) {$s'_{3}$};
     \node[anchor=north,yshift=-.5ex] at (axis cs:  3.9,0) {$s'_{4}$};
     \node[anchor=north west,yshift=-.5ex] at (axis cs:  4.5,0) {$s'$};
  \end{axis}
  \end{tikzpicture}}
\end{minipage}%
%\hspace{-0.21\linewidth}%
\begin{minipage}{0.4\linewidth}
 \vspace{8ex}
 $$
  \boxed{
  p_k=\Pr{S'=s'_k}=\int\limits_{u_k}^{u_{k+1}}f(s)\,\d s
  }
 $$
\end{minipage}
\end{frame}



\begin{frame}{Performance of Scalar Quantizers: Bit Rate}
\vspace{-1.5ex}\begin{center}
  \hfitbox{0.54\linewidth}{%
  \begin{tikzpicture}
     \begin{axis}[
        compat=newest,
        width=10cm,height=4cm,
        axis lines=none,
        xmin=-4.5,xmax=4.5,ymin=0,ymax=0.4,
        clip mode=individual, % ensures curves are plot one after another
        after end axis/.code={
          \draw[->,style={-{Latex[length=2.5mm,width=1.2mm]},line width=0.5pt}] 
               (axis cs: -5,0) -- (axis cs: 5,0);
          %\node[black, anchor=north east] at (axis cs: 5,-0.01) {$s$};
        }
     ]
     \pgfmathsetmacro{\s}{1}
     \addplot[line width=1.0pt,red]         coordinates { ( 0.0,0) ( 0.0,{\s*0.393469}) };
     \addplot[line width=1.0pt,officegreen] coordinates { ( 0.9,0) ( 0.9,{\s*0.191700}) };
     \addplot[line width=1.0pt,officegreen] coordinates { (-0.9,0) (-0.9,{\s*0.191700}) };
     \addplot[line width=1.0pt,blue]        coordinates { ( 1.9,0) ( 1.9,{\s*0.070522}) };
     \addplot[line width=1.0pt,blue]        coordinates { (-1.9,0) (-1.9,{\s*0.070522}) };
     \addplot[line width=1.0pt,vividviolet] coordinates { ( 2.9,0) ( 2.9,{\s*0.025944}) };
     \addplot[line width=1.0pt,vividviolet] coordinates { (-2.9,0) (-2.9,{\s*0.025944}) };
     \addplot[line width=1.0pt,cyan]        coordinates { ( 3.9,0) ( 3.9,{\s*0.009544}) };
     \addplot[line width=1.0pt,cyan]        coordinates { (-3.9,0) (-3.9,{\s*0.009544}) };
    
     \draw[red,fill]         (axis cs:  0.0,{\s*0.393469}) circle [radius=2pt];
     \draw[officegreen,fill] (axis cs:  0.9,{\s*0.191700}) circle [radius=2pt];
     \draw[officegreen,fill] (axis cs: -0.9,{\s*0.191700}) circle [radius=2pt];
     \draw[blue,fill]        (axis cs:  1.9,{\s*0.070522}) circle [radius=2pt];
     \draw[blue,fill]        (axis cs: -1.9,{\s*0.070522}) circle [radius=2pt];
     \draw[vividviolet,fill] (axis cs:  2.9,{\s*0.025944}) circle [radius=2pt];
     \draw[vividviolet,fill] (axis cs: -2.9,{\s*0.025944}) circle [radius=2pt];
     \draw[cyan,fill]        (axis cs:  3.9,{\s*0.009544}) circle [radius=2pt];
     \draw[cyan,fill]        (axis cs: -3.9,{\s*0.009544}) circle [radius=2pt];

     \node[anchor=north,yshift=-.5ex] at (axis cs: -3.9,0) {$s'_{-4}$};
     \node[anchor=north,yshift=-.5ex] at (axis cs: -2.9,0) {$s'_{-3}$};
     \node[anchor=north,yshift=-.5ex] at (axis cs: -1.9,0) {$s'_{-2}$};
     \node[anchor=north,yshift=-.5ex] at (axis cs: -0.9,0) {$s'_{-1}$};
     \node[anchor=north,yshift=-.5ex] at (axis cs:  0.0,0) {$s'_{0}$};
     \node[anchor=north,yshift=-.5ex] at (axis cs:  0.9,0) {$s'_{1}$};
     \node[anchor=north,yshift=-.5ex] at (axis cs:  1.9,0) {$s'_{2}$};
     \node[anchor=north,yshift=-.5ex] at (axis cs:  2.9,0) {$s'_{3}$};
     \node[anchor=north,yshift=-.5ex] at (axis cs:  3.9,0) {$s'_{4}$};
     \node[anchor=north west,yshift=-.5ex] at (axis cs:  4.5,0) {$s'$};
  \end{axis}
  \end{tikzpicture}}
\end{center}
\vspace{-1.5ex}
\bit\TabPositions{1.5em,15em}
\item<2-> Average bit rate $R$ \quad($\ell_k$ = codeword length for quantization index $k$)
  \vspace{-.5ex}$$
    R
      =
      \EV{\func{\ell}{\,S'\,}}
      =
      \EV{\func{\ell}{\,\alpha(S)\,}}
      \uncover<3->{=
      \sum_{k}p_k\,\ell_k}
  \uncover<4->{\qquad\text{with}\qquad
   p_k=\int_{u_k}^{u_{k+1}}f(s)\,\d s}
  $$
\item<5->\vspace{-1.5ex}Approximations (without knowledge of actual entropy coding)\\[1ex]
  \tab\loud{\iarrow}\;\;fixed-length coding:\tab
  $\displaystyle R=\big\lceil\,\log_2K\,\big\rceil\qquad\qquad(\,\text{$K$: number of quantization intervals}\,)$\\[1.5ex]
  \uncover<6->{\tab\loud{\iarrow}\;\;optimal entropy coding:\tab
  $\displaystyle R= H(S') = H(\,\alpha(S)\,) = -\sum_k p_k\,\log_2p_k$}
\eit\vspace{-3ex}
\end{frame}




\begin{frame}{Performance of Scalar Quantizers: MSE Distortion}
\begin{center}
  \hfitbox{0.8\linewidth}{%
  \begin{tikzpicture}
     \begin{axis}[
        compat=newest,
        width=14cm,height=4.7cm,
        axis lines=none,
        xmin=-4.5,xmax=4.5,ymin=-0.15,ymax=0.52,
        clip mode=individual, % ensures curves are plot one after another
        after end axis/.code={
          \draw[->,style={-{Latex[length=2.5mm,width=1.2mm]},line width=0.5pt}] 
               (axis cs: -5,0) -- (axis cs: 5,0);
          %\node[black, anchor=north east] at (axis cs: 5,-0.01) {$s$};
        }
     ]
       \addplot[name path=func,black,samples=101,domain=-4.5:4.5,line width=0.5pt,mark=none] {0.5*exp(-abs(x))};
       \path[name path=axis] (axis cs: -4.5,0) -- (axis cs: 4.5,0);
       \node[anchor=north west,yshift=-.5ex] at (axis cs:  4.5,0) {$s$};
       \addplot[red!20]         fill between[of=func and axis,soft clip={domain=-0.5: 0.5}];
       \addplot[green!20]       fill between[of=func and axis,soft clip={domain=-1.5:-0.5}];
       \addplot[green!20]       fill between[of=func and axis,soft clip={domain= 0.5: 1.5}];
       \addplot[blue!20]        fill between[of=func and axis,soft clip={domain=-2.5:-1.5}];
       \addplot[blue!20]        fill between[of=func and axis,soft clip={domain= 1.5: 2.5}];
       \addplot[vividviolet!20] fill between[of=func and axis,soft clip={domain=-3.5:-2.5}];
       \addplot[vividviolet!20] fill between[of=func and axis,soft clip={domain= 2.5: 3.5}];
       \addplot[cyan!20]        fill between[of=func and axis,soft clip={domain=-4.5:-3.5}];
       \addplot[cyan!20]        fill between[of=func and axis,soft clip={domain= 3.5: 4.5}];

       \pgfmathsetmacro{\su}{0.52}
       \pgfmathsetmacro{\ou}{-0.15}
       \begin{scope}
         \addplot[line width=0.5pt,dashed] coordinates { ( 0.5,{\ou}) ( 0.5,{\su}) };
         \addplot[line width=0.5pt,dashed] coordinates { (-0.5,{\ou}) (-0.5,{\su}) };
         \addplot[line width=0.5pt,dashed] coordinates { ( 1.5,{\ou}) ( 1.5,{\su}) };
         \addplot[line width=0.5pt,dashed] coordinates { (-1.5,{\ou}) (-1.5,{\su}) };
         \addplot[line width=0.5pt,dashed] coordinates { ( 2.5,{\ou}) ( 2.5,{\su}) };
         \addplot[line width=0.5pt,dashed] coordinates { (-2.5,{\ou}) (-2.5,{\su}) };
         \addplot[line width=0.5pt,dashed] coordinates { ( 3.5,{\ou}) ( 3.5,{\su}) };
         \addplot[line width=0.5pt,dashed] coordinates { (-3.5,{\ou}) (-3.5,{\su}) };
       \end{scope}
     
       \begin{scope}
         \node[anchor=north,yshift=-.3ex] at (axis cs: -3.5,{\ou}) {$u_{-3}$};
         \node[anchor=north,yshift=-.3ex] at (axis cs: -2.5,{\ou}) {$u_{-2}$};
         \node[anchor=north,yshift=-.3ex] at (axis cs: -1.5,{\ou}) {$u_{-1}$};
         \node[anchor=north,yshift=-.3ex] at (axis cs: -0.5,{\ou}) {$u_{0}$};
         \node[anchor=north,yshift=-.3ex] at (axis cs:  0.5,{\ou}) {$u_{1}$};
         \node[anchor=north,yshift=-.3ex] at (axis cs:  1.5,{\ou}) {$u_{2}$};
         \node[anchor=north,yshift=-.3ex] at (axis cs:  2.5,{\ou}) {$u_{3}$};
         \node[anchor=north,yshift=-.3ex] at (axis cs:  3.5,{\ou}) {$u_{4}$};
       \end{scope}
       
       \pgfmathsetmacro{\ss}{0.33}
       \pgfmathsetmacro{\os}{0.00}
       \begin{scope}
         \addplot[line width=1.0pt,myred]    coordinates { ( 0.0,{\os}) ( 0.0,{\ss}) };
         \addplot[line width=1.0pt,mygreen]  coordinates { ( 0.9,{\os}) ( 0.9,{\ss}) };
         \addplot[line width=1.0pt,mygreen]  coordinates { (-0.9,{\os}) (-0.9,{\ss}) };
         \addplot[line width=1.0pt,myblue]   coordinates { ( 1.9,{\os}) ( 1.9,{\ss}) };
         \addplot[line width=1.0pt,myblue]   coordinates { (-1.9,{\os}) (-1.9,{\ss}) };
         \addplot[line width=1.0pt,myviolet] coordinates { ( 2.9,{\os}) ( 2.9,{\ss}) };
         \addplot[line width=1.0pt,myviolet] coordinates { (-2.9,{\os}) (-2.9,{\ss}) };
         \addplot[line width=1.0pt,cyan]     coordinates { ( 3.9,{\os}) ( 3.9,{\ss}) };
         \addplot[line width=1.0pt,cyan]     coordinates { (-3.9,{\os}) (-3.9,{\ss}) };

         \draw[fill,myred]    (axis cs:  0.0,{\ss}) circle [radius=2.5pt];
         \draw[fill,mygreen]  (axis cs:  0.9,{\ss}) circle [radius=2.5pt];
         \draw[fill,mygreen]  (axis cs: -0.9,{\ss}) circle [radius=2.5pt];
         \draw[fill,myblue]   (axis cs:  1.9,{\ss}) circle [radius=2.5pt];
         \draw[fill,myblue]   (axis cs: -1.9,{\ss}) circle [radius=2.5pt];
         \draw[fill,myviolet] (axis cs:  2.9,{\ss}) circle [radius=2.5pt];
         \draw[fill,myviolet] (axis cs: -2.9,{\ss}) circle [radius=2.5pt];
         \draw[fill,cyan]     (axis cs:  3.9,{\ss}) circle [radius=2.5pt];
         \draw[fill,cyan]     (axis cs: -3.9,{\ss}) circle [radius=2.5pt];
         
         \node[blue,anchor=north,yshift=-.0ex] at (axis cs: -3.9,{\os}) {$s'_{-4}$};
         \node[blue,anchor=north,yshift=-.0ex] at (axis cs: -2.9,{\os}) {$s'_{-3}$};
         \node[blue,anchor=north,yshift=-.0ex] at (axis cs: -1.9,{\os}) {$s'_{-2}$};
         \node[blue,anchor=north,yshift=-.0ex] at (axis cs: -0.9,{\os}) {$s'_{-1}$};
         \node[blue,anchor=north,yshift=-.0ex] at (axis cs:  0.0,{\os}) {$s'_{0}$};
         \node[blue,anchor=north,yshift=-.0ex] at (axis cs:  0.9,{\os}) {$s'_{1}$};
         \node[blue,anchor=north,yshift=-.0ex] at (axis cs:  1.9,{\os}) {$s'_{2}$};
         \node[blue,anchor=north,yshift=-.0ex] at (axis cs:  2.9,{\os}) {$s'_{3}$};
         \node[blue,anchor=north,yshift=-.0ex] at (axis cs:  3.9,{\os}) {$s'_{4}$};
       \end{scope}
  \end{axis}
  \end{tikzpicture}}
\end{center}
\bit
\item Average MSE distortion $D$ is given by
  \begin{align}\label{QuantMSE}
    D
      &\;=\;
      \EV{\big(\,S-Q(S)\,\big)^2}
      \uncover<2->{=
      \int_{-\infty}^{\infty}(s-Q(s))^2f(s)ds}
      \uncover<3->{
        \;=\;
      \sum_{\forall k}\int_{u_k}^{u_{k+1}}(s-s'_k)^2\,f(s)\;\d s}
  \end{align}
\item<4->[\iarrow]\smallskip Similar for other additive distortion measures (e.g., all $p$-norm distortion measures)
\eit\vspace{-5ex}
\end{frame}



\begin{frame}{Optimal Scalar Quantizer for Fixed-Length Coding}
\STRUC{Goal: Minimize MSE Distortion for Quantizer with $K$ Quantization Intervals}
\bit
\item Neglect impact of entropy coding\quad\loud{\iarrow}\quad Consider fixed-length coding
\item[\iarrow]\smallskip Rate $R$ and MSE distortion $D$ are given by
  \begin{align*}
    R&\;=\; \big\lceil\log_2K\big\rceil\qquad\qquad(\,\text{typically $K=2^B$, with $B$ being the bits per codeword}\,)\\[1ex]
    D&\;=\;
    \sum_{\forall k}\int_{u_k}^{u_{k+1}}(s-s'_k)^2\,f(s)\;\d s
  \end{align*}
  \eit

  \uncover<2->{\STRUC{Optimize Quantizer of size $K$}
  \bit
\item Bit rate $R$ is independent on decision thresholds and reconstruction levels ($R$ is given by $K$)
\item<3-> Distortion (MSE) depends on
  \bit\relsize{1}\itemMode{arrow}
\item $K$ reconstruction levels $s'_k$
\item $K-1$ decision thresholds $u_k$
  \eit
  \eit}
\end{frame}


\subsection{Centroid condition and nearest neighbor condition}
\begin{frame}{Centroid condition and nearest neighbor condition}
Necessary conditions for optimal $u_k$ if $s_k'$ are given 
or for optimal $s_k'$ if $u_k$ are given: 
\begin{proposition}[Centroid condition and nearest neighbor condition] 
\bit
\item For fixed $u_k$, reconstruction points $s_k'$ that minimize \eqref{QuantMSE} %have to 
satisfy the \loud{centroid condition}
\begin{align}\label{CentrCond}
s_k'=\frac{\int_{u_k}^{u_{k+1}}sf(s)ds}{\int_{u_k}^{u_k+1}f(s)ds}.
\end{align}
\item For fixed $s_k'$, decision thresholds $u_k$ that minimize \eqref{QuantMSE} %have to 
satifsy the \loud{nearest neighbor condition}
\begin{align}\label{NearestNB}
u_k=\frac{1}{2}\left(s_{k-1}'+s_k'\right)
\end{align}
\eit
\end{proposition}
\bit
\item Above two necessary conditions are basis of the \loud{Lloyd algorithm}, see below. 
\item Centroids $s_k'$ of \eqref{CentrCond} are the expexctation values given the event $S\in[u_k,u_{k+1})$ ..  
\eit
%
%
%Minimization with respect to $s_k'$: Set derivative to zero:
%\begin{align*}
%0=\frac{\partial}{\partial s_k'}D=-\int_{u_k}^{u_{k+1}}2(s-s'_k)f(s)ds
%\end{align*}
%Centroid condition: 
%\begin{align*}
%s_k'=\frac{\int_{u_k}^{u_{k+1}}sf(s)ds}{\int_{u_k}^{u_k+1}f(s)ds}
%\end{align*}
\end{frame}

\begin{frame}{Proof of centroid condition}
\loud{Set derivatives with respect to $s_k'$ to zero:}
%\bit
%\item
\begin{align*}
0\stackrel{!}{=}&\frac{\partial}{\partial s_k'}D\\=&\frac{\partial}{\partial s_k'}\int_{u_k}^{u_{k+1}}(s-s'_k)^2f(s)ds\\
=&-\int_{u_k}^{u_{k+1}}2(s-s'_k)f(s)ds.
\end{align*}
This implies
\begin{align*}
\int_{u_k}^{u_{k+1}}sf(s)ds=s'_k\int_{u_k}^{u_{k+1}}f(s)ds. \qed
\end{align*} 
%\item[\iarrow] adf
%\eit
\end{frame}
\begin{frame}{Proof of nearest neighbor condition} 
\loud{Set derivatives with respect to $u_k$ to zero:} 

By fundamental theorem of calculus, one has 
\begin{align*}
0 \stackrel{!}{=}  &\frac{\partial}{\partial u_k}D\\=&\frac{\partial}{\partial u_k}\left(\int_{u_{k-1}}^{u_{k}}(s-s'_{k-1})^2f(s)ds+\int_{u_k}^{u_{k+1}}(s-s'_k)^2f(s)ds\right)\\
=&(u_k-s'_{k-1})^2f(u_k)-(u_k-s'_{k})^2f(u_k).
\end{align*}
This implies that
\begin{align*}
(u_k-s'_{k-1})^2=(u_k-s'_{k})^2.
\end{align*}
Since $s'_{k-1}\leq u_k\leq s'_k$, it follows that
\begin{align*}
u_k-s'_{k-1}=s'_k-u_k. \qed
\end{align*}
\end{frame}

\subsection{Lloyd Algorithm}



\begin{frame}{Lloyd Quantizer: Minimization of Distortion}
\STRUC{Necessary Conditions for Minimizing MSE Distortion}
\ben
\item<+-> Centroid condition
  $$
  s'_k = \frac{\int_{u_k}^{u_{k+1}}s\,f(s)\;\d s}
              {\int_{u_k}^{u_{k+1}}f(s)\;\d s}
  $$
\item<+-> Nearest neighbour condition
  $$
  u_k=\frac{1}{2}\big(s'_{k}+s'_{k-1}\big)
  $$
\een
\uncover<+->{\STRUC{Design of Lloyd quantizers}
\bit
\item In general: Cannot be derived in closed form 
\item<+->[\iarrow]\smallskip Iterative algorithm consisting of
  \bit\normalsize
  \item Optimize decision thresholds $u_k$ given reconstruction levels~$s'_k$
  \item Optimize reconstruction levels~$s'_k$ given decision thresholds~$u_k$ 
  \eit
\eit}\vspace{-3ex}
\end{frame}




\begin{frame}{Lloyd Algorithm for Given Pdf ~(MSE Distortion)}
  \vspace{-1ex}Given is:
  \begin{minipage}[t]{0.8\linewidth}
    \vskip-1.5ex%
\bit\itemMode{circle}
\item<+-> the size $K$ of the quantizer (i.e., the number of quantization intervals)
\item<.-> the marginal probability density function $f(s)$ of the source
  \eit
  \end{minipage}
  
\medskip
\uncover<+->{\STRUC{Iterative quantizer design}}
\ben
\item<.-> Choose an initial set of $K$ reconstruction levels~$\{s'_k\}$
\item<+->\smallskip Update the $K-1$ decision thresholds~$\{u_k\}$ according to
	{\[
           u_k=\frac{s'_k+s'_{k-1}}{2}
	\qquad\text{(nearest neighbor condition)}
	\]}
\item<+-> Update the $K$ reconstruction levels~$\{s'_k\}$ according to
	{\[
	s'_k=\frac{\int_{u_k}^{u_{k+1}}s\,f(s)\;\d s}
              {\int_{u_k}^{u_{k+1}}f(s)\;\d s}
	\qquad
	\text{(centroid condition)}
	\]}
\item<+-> Repeat the previous two steps until convergence
\een
\end{frame}



\begin{frame}{Lloyd Algorithm for a Training Set ~(MSE Distortion)}
  \vspace{-1ex}Given is:
  \begin{minipage}[t]{0.8\linewidth}
    \vskip-1.5ex%
\bit\itemMode{circle}
\item<+-> the size $K$ of the quantizer (i.e., the number of quantization intervals)
\item<.-> a sufficiently large realization $\{s_n\}$ of considered source
  \eit
  \end{minipage}
  
\medskip
\uncover<+->{\STRUC{Iterative quantizer design}}
\ben
\item<.-> Choose an initial set of $K$ reconstruction levels~$\{s'_k\}$
\item<+-> Associate all samples of the training set $\{s_n\}$ with one
      of the quantization intervals~$\set{I}_k$
	\[
	q(s_n)=\arg \min_{\forall k}\;(s_n-s'_k)^2
	\qquad\quad\;\,
	\text{(nearest neighbor condition)}
	\]
\item<+-> Update the reconstruction levels $\{s'_k\}$ according to
  \vspace{-.5ex}\[
  s'_k=\frac{1}{N_k}\,\sum_{n:\,q(s_n)=k}s_n
	\qquad\;\;
	\text{(centroid condition)}\vspace{-1.0ex}
	\]
        where $N_k$ is the number of samples associated with $\set{I}_k$
\item<+->\vspace{1.0ex} Repeat the previous two steps until convergence
\een
\end{frame}


\input{RD_V/LloydExampleGauss2Bit_New.tex}



\begin{frame}{Example: Convergence of Lloyd Algorithm for Gaussian Source}
\begin{minipage}{0.47\linewidth}
  \hfitbox{\linewidth}{\begin{tikzpicture}
    \begin{axis}[
        %ybar stacked,
        xmin=-1,xmax=13,ymin=0,ymax=10,
        compat=newest,
        axis on top=true,
        width=10cm,height=5cm,
        bar width=0.5cm,
        %axis lines=none,
        axis y line=none,
        axis line style=thick,
        xtick style={draw=none},
        %axis x line style={line width=1pt},
        xtick={0,1,2,3,4,5,6,7,8,9,10,11,12},
        %after end axis/.code={
        %  \draw[->,style={-{Latex[length=2.5mm,width=1.2mm]},line width=0.5pt}] 
        %  (axis cs: -1,0) -- (axis cs: 13.5,0);
        %  },
      ]
      \addplot[ybar stacked,draw=none,fill=myviolet!30] coordinates {
        ( 0, 2.000)
        ( 1, 2.963)
        ( 2, 3.433)
        ( 3, 3.683)
        ( 4, 3.823)
        ( 5, 3.903)
        ( 6, 3.951)
        ( 7, 3.978)
        ( 8, 3.995)
        ( 9, 4.004)
        (10, 4.010)
        (11, 4.013)
        (12, 4.015)
      };
      \addplot[ybar stacked,draw=none,fill=green!30] coordinates {
        ( 0, 3.000)
        ( 1, 2.037)
        ( 2, 1.567)
        ( 3, 1.317)
        ( 4, 1.177)
        ( 5, 1.097)
        ( 6, 1.049)
        ( 7, 1.022)
        ( 8, 1.005)
        ( 9, 0.996)
        (10, 0.990)
        (11, 0.987)
        (12, 0.985)
      };
      \addplot[ybar stacked,draw=none,fill=red!30] coordinates {
        ( 0, 3.000)
        ( 1, 2.037)
        ( 2, 1.567)
        ( 3, 1.317)
        ( 4, 1.177)
        ( 5, 1.097)
        ( 6, 1.049)
        ( 7, 1.022)
        ( 8, 1.005)
        ( 9, 0.996)
        (10, 0.990)
        (11, 0.987)
        (12, 0.985)
      };
      \addplot[ybar stacked,draw=none,fill=blue!30] coordinates {
        ( 0, 2.000)
        ( 1, 2.963)
        ( 2, 3.433)
        ( 3, 3.683)
        ( 4, 3.823)
        ( 5, 3.903)
        ( 6, 3.951)
        ( 7, 3.978)
        ( 8, 3.995)
        ( 9, 4.004)
        (10, 4.010)
        (11, 4.013)
        (12, 4.015)
      };
      \addplot[only marks,mark size=1.5pt,myviolet!50!black] coordinates {
        ( 0,-4.500+5)
        ( 1,-3.283+5)
        ( 2,-2.406+5)
        ( 3,-1.996+5)
        ( 4,-1.785+5)
        ( 5,-1.669+5)
        ( 6,-1.603+5)
        ( 7,-1.565+5)
        ( 8,-1.543+5)
        ( 9,-1.529+5)
        (10,-1.522+5)
        (11,-1.517+5)
        (12,-1.514+5)
      };
      \addplot[only marks,mark size=1.5pt,mygreen!50!black] coordinates {
        ( 0,-1.500+5)
        ( 1,-0.791+5)
        ( 2,-0.728+5)
        ( 3,-0.639+5)
        ( 4,-0.570+5)
        ( 5,-0.524+5)
        ( 6,-0.496+5)
        ( 7,-0.479+5)
        ( 8,-0.468+5)
        ( 9,-0.462+5)
        (10,-0.458+5)
        (11,-0.456+5)
        (12,-0.455+5)
      };
      \addplot[only marks,mark size=1.5pt,myred!50!black] coordinates {
        ( 0,1.500+5)
        ( 1,0.791+5)
        ( 2,0.728+5)
        ( 3,0.639+5)
        ( 4,0.570+5)
        ( 5,0.524+5)
        ( 6,0.496+5)
        ( 7,0.479+5)
        ( 8,0.468+5)
        ( 9,0.462+5)
        (10,0.458+5)
        (11,0.456+5)
        (12,0.455+5)
      };
      \addplot[only marks,mark size=1.5pt,myblue!50!black] coordinates {
        ( 0,4.500+5)
        ( 1,3.283+5)
        ( 2,2.406+5)
        ( 3,1.996+5)
        ( 4,1.785+5)
        ( 5,1.669+5)
        ( 6,1.603+5)
        ( 7,1.565+5)
        ( 8,1.543+5)
        ( 9,1.529+5)
        (10,1.522+5)
        (11,1.517+5)
        (12,1.514+5)
      };
    \end{axis}
  \end{tikzpicture}}\\[3ex]
  \hfitbox{\linewidth}{\begin{tikzpicture}
    \begin{axis}[
        xmin=-1,xmax=13,ymin=0,ymax=12,
        compat=newest,
        width=10cm,height=5cm,
        axis lines=none,
        after end axis/.code={
          \draw[->,style={-{Latex[length=3.5mm,width=2.0mm]},thick}] 
          (axis cs: -1,0) -- (axis cs: 13,0);
          \draw[->,style={-{Latex[length=3.5mm,width=2.0mm]},thick}] 
          (axis cs: -0.5,0) -- (axis cs: -0.5,12);
          \node[anchor=north west,xshift=.5ex] at (axis cs:-0.5,12) {SNR};
          },
      ]
      \draw[dashed,black] (axis cs:-0.5,9.30) -- (axis cs:12,9.30);
      \node[anchor=north west] at (axis cs:-0.5,9.30) {9.30\,dB};
      \addplot+[mark size=1.5pt] coordinates {
        ( 0, 0.697)
        ( 1, 5.423)
        ( 2, 6.996)
        ( 3, 8.223)
        ( 4, 8.867)
        ( 5, 9.138)
        ( 6, 9.242)
        ( 7, 9.279)
        ( 8, 9.293)
        ( 9, 9.298)
        (10, 9.299)
        (11, 9.300)
        (12, 9.300)
      };
    \end{axis}
  \end{tikzpicture}}
\end{minipage}%
\hfill%
\begin{minipage}{0.47\linewidth}
  \hfitbox{\linewidth}{\begin{tikzpicture}
    \begin{axis}[
        %ybar stacked,
        xmin=-1,xmax=13,ymin=0,ymax=10,
        compat=newest,
        axis on top=true,
        width=10cm,height=5cm,
        bar width=0.5cm,
        %axis lines=none,
        axis y line=none,
        axis line style=thick,
        xtick style={draw=none},
        %axis x line style={line width=1pt},
        xtick={0,1,2,3,4,5,6,7,8,9,10,11,12},
        %after end axis/.code={
        %  \draw[->,style={-{Latex[length=2.5mm,width=1.2mm]},line width=0.5pt}] 
        %  (axis cs: -1,0) -- (axis cs: 13.5,0);
        %  },
      ]
      \addplot[ybar stacked,draw=none,fill=myviolet!30] coordinates {
        ( 0, 4.925)
        ( 1, 4.558)
        ( 2, 4.342)
        ( 3, 4.212)
        ( 4, 4.134)
        ( 5, 4.087)
        ( 6, 4.059)
        ( 7, 4.043)
        ( 8, 4.033)
        ( 9, 4.027)
        (10, 4.023)
        (11, 4.021)
        (12, 4.020)
      };
      \addplot[ybar stacked,draw=none,fill=green!30] coordinates {
        ( 0, 0.075)
        ( 1, 0.442)
        ( 2, 0.658)
        ( 3, 0.788)
        ( 4, 0.866)
        ( 5, 0.913)
        ( 6, 0.941)
        ( 7, 0.957)
        ( 8, 0.967)
        ( 9, 0.973)
        (10, 0.977)
        (11, 0.979)
        (12, 0.980)
      };
      \addplot[ybar stacked,draw=none,fill=red!30] coordinates {
        ( 0, 0.075)
        ( 1, 0.442)
        ( 2, 0.658)
        ( 3, 0.788)
        ( 4, 0.866)
        ( 5, 0.913)
        ( 6, 0.941)
        ( 7, 0.957)
        ( 8, 0.967)
        ( 9, 0.973)
        (10, 0.977)
        (11, 0.979)
        (12, 0.980)
      };
      \addplot[ybar stacked,draw=none,fill=blue!30] coordinates {
        ( 0, 4.925)
        ( 1, 4.558)
        ( 2, 4.342)
        ( 3, 4.212)
        ( 4, 4.134)
        ( 5, 4.087)
        ( 6, 4.059)
        ( 7, 4.043)
        ( 8, 4.033)
        ( 9, 4.027)
        (10, 4.023)
        (11, 4.021)
        (12, 4.020)
      };
      \addplot[only marks,mark size=1.5pt,myviolet!50!black] coordinates {
        ( 0,-0.113+5)
        ( 1,-0.846+5)
        ( 2,-1.099+5)
        ( 3,-1.259+5)
        ( 4,-1.358+5)
        ( 5,-1.419+5)
        ( 6,-1.456+5)
        ( 7,-1.478+5)
        ( 8,-1.491+5)
        ( 9,-1.499+5)
        (10,-1.504+5)
        (11,-1.506+5)
        (12,-1.508+5)
      };
      \addplot[only marks,mark size=1.5pt,mygreen!50!black] coordinates {
        ( 0,-0.038+5)
        ( 1,-0.037+5)
        ( 2,-0.217+5)
        ( 3,-0.317+5)
        ( 4,-0.374+5)
        ( 5,-0.407+5)
        ( 6,-0.426+5)
        ( 7,-0.437+5)
        ( 8,-0.443+5)
        ( 9,-0.447+5)
        (10,-0.449+5)
        (11,-0.451+5)
        (12,-0.452+5)
      };
      \addplot[only marks,mark size=1.5pt,myred!50!black] coordinates {
        ( 0,0.038+5)
        ( 1,0.037+5)
        ( 2,0.217+5)
        ( 3,0.317+5)
        ( 4,0.374+5)
        ( 5,0.407+5)
        ( 6,0.426+5)
        ( 7,0.437+5)
        ( 8,0.443+5)
        ( 9,0.447+5)
        (10,0.449+5)
        (11,0.451+5)
        (12,0.452+5)
      };
      \addplot[only marks,mark size=1.5pt,myblue!50!black] coordinates {
        ( 0,0.113+5)
        ( 1,0.846+5)
        ( 2,1.099+5)
        ( 3,1.259+5)
        ( 4,1.358+5)
        ( 5,1.419+5)
        ( 6,1.456+5)
        ( 7,1.478+5)
        ( 8,1.491+5)
        ( 9,1.499+5)
        (10,1.504+5)
        (11,1.506+5)
        (12,1.508+5)
      };
    \end{axis}
  \end{tikzpicture}}\\[3ex]
  \hfitbox{\linewidth}{\begin{tikzpicture}
    \begin{axis}[
        xmin=-1,xmax=13,ymin=0,ymax=12,
        compat=newest,
        width=10cm,height=5cm,
        axis lines=none,
        after end axis/.code={
          \draw[->,style={-{Latex[length=3.5mm,width=2.0mm]},thick}] 
          (axis cs: -1,0) -- (axis cs: 13,0);
          \draw[->,style={-{Latex[length=3.5mm,width=2.0mm]},thick}] 
          (axis cs: -0.5,0) -- (axis cs: -0.5,12);
          \node[anchor=north west,xshift=.5ex] at (axis cs:-0.5,12) {SNR};
          },
      ]
      \draw[dashed,black] (axis cs:-0.5,9.30) -- (axis cs:12,9.30);
      \node[anchor=north west] at (axis cs:-0.5,9.30) {9.30\,dB};
      \addplot+[mark size=1.5pt] coordinates {
        ( 0, 0.795)
        ( 1, 6.166)
        ( 2, 7.962)
        ( 3, 8.793)
        ( 4, 9.118)
        ( 5, 9.236)
        ( 6, 9.278)
        ( 7, 9.292)
        ( 8, 9.298)
        ( 9, 9.299)
        (10, 9.300)
        (11, 9.300)
        (12, 9.300)
      };
    \end{axis}
  \end{tikzpicture}}
\end{minipage}%
\end{frame}






\begin{frame}{Example: Lloyd Algorithm for Laplacian Source}
  \begin{minipage}{0.5\linewidth}
    Laplacian Source
    \bit
  \item Zero mean $\mu=0$
    \item Unit variance $\sigma^2=1$
      \eit

      \bigskip\bigskip
\STRUC{Lloyd Quantizer of size $K=4$}
\bit\TabPositions{10em}
\item Decision thresholds:\tab\hspace{-.35ex}%
  \begin{tabular}[t]{r@{$\;=\;$}r}
  $u_1$ & $-1.127$\\[.5ex]
  $u_2$ & $ 0.000$\\[.5ex]
  $u_3$ & $ 1.127$\\
  \end{tabular}
\item\medskip Reconstruction levels:\tab
  \begin{tabular}[t]{r@{$\;=\;$}r}
  $s'_0$ & $-1.834$\\[.5ex]
  $s'_1$ & $-0.420$\\[.5ex]
  $s'_2$ & $ 0.420$\\[.5ex]
  $s'_3$ & $ 1.834$\\
  \end{tabular}
  \eit
  \end{minipage}%
  \begin{minipage}{0.5\linewidth}
    \begin{center}
    \hfitbox{\linewidth}{%
  \begin{tikzpicture}
     \begin{axis}[
        compat=newest,
        width=9cm,height=6cm,
        axis lines=none,
        xmin=-3.5,xmax=3.5,ymin=-0.15,ymax=0.75,
        clip mode=individual, % ensures curves are plot one after another
        after end axis/.code={
          \draw[->,style={-{Latex[length=2.5mm,width=1.2mm]},line width=0.5pt}] 
               (axis cs: -3.5,0) -- (axis cs: 3.5,0);
          %\node[black, anchor=north east] at (axis cs: 5,-0.01) {$s$};
        }
     ]
       \addplot[name path=func,black,samples=100,domain=-3.5:3.3,line width=0.5pt,mark=none]
               {1/sqrt(2)*exp(-sqrt(2)*abs(x))};
       \path[name path=axis] (axis cs: -3.5,0) -- (axis cs: 3.2,0);
       \node[anchor=north west,yshift=-.5ex] at (axis cs:  3.4,0) {$s$};
       \addplot[myviolet!20]    fill between[of=func and axis,soft clip={domain=-4.000:-1.127}];
       \addplot[green!20]       fill between[of=func and axis,soft clip={domain=-1.127: 0.000}];
       \addplot[red!20]         fill between[of=func and axis,soft clip={domain= 0.000: 1.127}];
       \addplot[blue!20]        fill between[of=func and axis,soft clip={domain= 1.127: 3.300}];

       \pgfmathsetmacro{\su}{0.75}
       \pgfmathsetmacro{\ou}{-0.10}
       \begin{scope}
         \addplot[line width=0.5pt,dashed] coordinates { ( 0.000,{\ou}) ( 0.000,{\su}) };
         \addplot[line width=0.5pt,dashed] coordinates { (-1.127,{\ou}) (-1.127,{\su}) };
         \addplot[line width=0.5pt,dashed] coordinates { ( 1.127,{\ou}) ( 1.127,{\su}) };
       \end{scope}
     
       \begin{scope}
         \node[anchor=north,yshift=-.3ex] at (axis cs: -1.127,{\ou}) {$u_{1}$};
         \node[anchor=north,yshift=-.3ex] at (axis cs:  0.000,{\ou}) {$u_{2}$};
         \node[anchor=north,yshift=-.3ex] at (axis cs:  1.127,{\ou}) {$u_{3}$};
       \end{scope}
       
       \pgfmathsetmacro{\ss}{0.25}
       \pgfmathsetmacro{\os}{0.00}
       \begin{scope}
         \addplot[line width=1.0pt,myviolet] coordinates { ( -1.834,{\os}) (-1.834,{\ss}) };
         \addplot[line width=1.0pt,mygreen]  coordinates { ( -0.420,{\os}) (-0.420,{\ss}) };
         \addplot[line width=1.0pt,myred]    coordinates { (  0.420,{\os}) ( 0.420,{\ss}) };
         \addplot[line width=1.0pt,myblue]   coordinates { (  1.834,{\os}) ( 1.834,{\ss}) };

         \draw[fill,myviolet] (axis cs: -1.834,{\ss}) circle [radius=2.5pt];
         \draw[fill,mygreen]  (axis cs: -0.420,{\ss}) circle [radius=2.5pt];
         \draw[fill,myred]    (axis cs:  0.420,{\ss}) circle [radius=2.5pt];
         \draw[fill,myblue]   (axis cs:  1.834,{\ss}) circle [radius=2.5pt];
         
         \node[blue,anchor=north,yshift=-.0ex] at (axis cs: -1.834,{\os}) {$s'_{0}$};
         \node[blue,anchor=north,yshift=-.0ex] at (axis cs: -0.420,{\os}) {$s'_{1}$};
         \node[blue,anchor=north,yshift=-.0ex] at (axis cs:  0.420,{\os}) {$s'_{2}$};
         \node[blue,anchor=north,yshift=-.0ex] at (axis cs:  1.834,{\os}) {$s'_{3}$};
       \end{scope}
  \end{axis}
    \end{tikzpicture}}\\[2ex]
    \begin{align*}
      R&\;=\;2.0\qquad(\text{fixed-length coding})\\[.2ex]
      D&\;=\;0.176\\[.2ex]
      \text{SNR} &\;=\; 7.54\;\text{dB}
  \end{align*}
    \end{center}
  \end{minipage}%
\end{frame}





\begin{frame}{Example: Convergence of Lloyd Algorithm for Laplacian Source}
\begin{minipage}{0.47\linewidth}
  \hfitbox{\linewidth}{\begin{tikzpicture}
    \begin{axis}[
        %ybar stacked,
        xmin=-1,xmax=13,ymin=0,ymax=10,
        compat=newest,
        axis on top=true,
        width=10cm,height=5cm,
        bar width=0.5cm,
        %axis lines=none,
        axis y line=none,
        axis line style=thick,
        xtick style={draw=none},
        %axis x line style={line width=1pt},
        xtick={0,1,2,3,4,5,6,7,8,9,10,11,12},
        %after end axis/.code={
        %  \draw[->,style={-{Latex[length=2.5mm,width=1.2mm]},line width=0.5pt}] 
        %  (axis cs: -1,0) -- (axis cs: 13.5,0);
        %  },
      ]
      \addplot[ybar stacked,draw=none,fill=myviolet!30] coordinates {
        ( 0, 2.000)
        ( 1, 2.815)
        ( 2, 3.252)
        ( 3, 3.500)
        ( 4, 3.645)
        ( 5, 3.732)
        ( 6, 3.786)
        ( 7, 3.819)
        ( 8, 3.839)
        ( 9, 3.852)
        (10, 3.860)
        (11, 3.865)
        (12, 3.868)
      };
      \addplot[ybar stacked,draw=none,fill=green!30] coordinates {
        ( 0, 3.000)
        ( 1, 2.185)
        ( 2, 1.748)
        ( 3, 1.500)
        ( 4, 1.355)
        ( 5, 1.268)
        ( 6, 1.214)
        ( 7, 1.181)
        ( 8, 1.161)
        ( 9, 1.148)
        (10, 1.140)
        (11, 1.135)
        (12, 1.132)
      };
      \addplot[ybar stacked,draw=none,fill=red!30] coordinates {
        ( 0, 3.000)
        ( 1, 2.185)
        ( 2, 1.748)
        ( 3, 1.500)
        ( 4, 1.355)
        ( 5, 1.268)
        ( 6, 1.214)
        ( 7, 1.181)
        ( 8, 1.161)
        ( 9, 1.148)
        (10, 1.140)
        (11, 1.135)
        (12, 1.132)
      };
      \addplot[ybar stacked,draw=none,fill=blue!30] coordinates {
        ( 0,2.000)
        ( 1,2.815)
        ( 2,3.252)
        ( 3,3.500)
        ( 4,3.645)
        ( 5,3.732)
        ( 6,3.786)
        ( 7,3.819)
        ( 8,3.839)
        ( 9,3.852)
        (10,3.860)
        (11,3.865)
        (12,3.868)
      };
      \addplot[only marks,mark size=1.5pt,myviolet!50!black] coordinates {
        ( 0,-4.500+5)
        ( 1,-3.707+5)
        ( 2,-2.892+5)
        ( 3,-2.455+5)
        ( 4,-2.207+5)
        ( 5,-2.062+5)
        ( 6,-1.975+5)
        ( 7,-1.921+5)
        ( 8,-1.888+5)
        ( 9,-1.868+5)
        (10,-1.855+5)
        (11,-1.847+5)
        (12,-1.842+5)
      };
      \addplot[only marks,mark size=1.5pt,mygreen!50!black] coordinates {
        ( 0,-1.500+5)
        ( 1,-0.663+5)
        ( 2,-0.603+5)
        ( 3,-0.546+5)
        ( 4,-0.503+5)
        ( 5,-0.473+5)
        ( 6,-0.454+5)
        ( 7,-0.441+5)
        ( 8,-0.433+5)
        ( 9,-0.428+5)
        (10,-0.425+5)
        (11,-0.423+5)
        (12,-0.422+5)
      };
      \addplot[only marks,mark size=1.5pt,myred!50!black] coordinates {
        ( 0,1.500+5)
        ( 1,0.663+5)
        ( 2,0.603+5)
        ( 3,0.546+5)
        ( 4,0.503+5)
        ( 5,0.473+5)
        ( 6,0.454+5)
        ( 7,0.441+5)
        ( 8,0.433+5)
        ( 9,0.428+5)
        (10,0.425+5)
        (11,0.423+5)
        (12,0.422+5)
      };
      \addplot[only marks,mark size=1.5pt,myblue!50!black] coordinates {
        ( 0,4.500+5)
        ( 1,3.707+5)
        ( 2,2.892+5)
        ( 3,2.455+5)
        ( 4,2.207+5)
        ( 5,2.062+5)
        ( 6,1.975+5)
        ( 7,1.921+5)
        ( 8,1.888+5)
        ( 9,1.868+5)
        (10,1.855+5)
        (11,1.847+5)
        (12,1.842+5)
      };
    \end{axis}
  \end{tikzpicture}}\\[3ex]
  \hfitbox{\linewidth}{\begin{tikzpicture}
    \begin{axis}[
        xmin=-1,xmax=13,ymin=0,ymax=10,
        compat=newest,
        width=10cm,height=5cm,
        axis lines=none,
        after end axis/.code={
          \draw[->,style={-{Latex[length=3.5mm,width=2.0mm]},thick}] 
          (axis cs: -1,0) -- (axis cs: 13,0);
          \draw[->,style={-{Latex[length=3.5mm,width=2.0mm]},thick}] 
          (axis cs: -0.5,0) -- (axis cs: -0.5,10);
          \node[anchor=north west,xshift=.5ex] at (axis cs:-0.5,10) {SNR};
          },
      ]
      \draw[dashed,black] (axis cs:-0.5,7.541) -- (axis cs:12,7.541);
      \node[anchor=north west] at (axis cs:-0.5,7.541) {7.54\,dB};
      \addplot+[mark size=1.5pt] coordinates {
        ( 0,-0.284)
        ( 1, 5.142)
        ( 2, 6.246)
        ( 3, 6.936)
        ( 4, 7.282)
        ( 5, 7.435)
        ( 6, 7.499)
        ( 7, 7.524)
        ( 8, 7.535)
        ( 9, 7.539)
        (10, 7.540)
        (11, 7.541)
        (12, 7.541)
      };
    \end{axis}
  \end{tikzpicture}}
\end{minipage}%
\hfill%
\begin{minipage}{0.47\linewidth}
  \hfitbox{\linewidth}{\begin{tikzpicture}
    \begin{axis}[
        %ybar stacked,
        xmin=-1,xmax=13,ymin=0,ymax=10,
        compat=newest,
        axis on top=true,
        width=10cm,height=5cm,
        bar width=0.5cm,
        %axis lines=none,
        axis y line=none,
        axis line style=thick,
        xtick style={draw=none},
        %axis x line style={line width=1pt},
        xtick={0,1,2,3,4,5,6,7,8,9,10,11,12},
        %after end axis/.code={
        %  \draw[->,style={-{Latex[length=2.5mm,width=1.2mm]},line width=0.5pt}] 
        %  (axis cs: -1,0) -- (axis cs: 13.5,0);
        %  },
      ]
      \addplot[ybar stacked,draw=none,fill=myviolet!30] coordinates {
        ( 0, 4.925)
        ( 1, 4.591)
        ( 2, 4.349)
        ( 3, 4.183)
        ( 4, 4.072)
        ( 5, 4.000)
        ( 6, 3.953)
        ( 7, 3.924)
        ( 8, 3.905)
        ( 9, 3.893)
        (10, 3.886)
        (11, 3.881)
        (12, 3.878)
      };
      \addplot[ybar stacked,draw=none,fill=green!30] coordinates {
        ( 0, 0.075)
        ( 1, 0.409)
        ( 2, 0.651)
        ( 3, 0.817)
        ( 4, 0.928)
        ( 5, 1.000)
        ( 6, 1.047)
        ( 7, 1.076)
        ( 8, 1.095)
        ( 9, 1.107)
        (10, 1.114)
        (11, 1.119)
        (12, 1.122)
      };
      \addplot[ybar stacked,draw=none,fill=red!30] coordinates {
        ( 0, 0.075)
        ( 1, 0.409)
        ( 2, 0.651)
        ( 3, 0.817)
        ( 4, 0.928)
        ( 5, 1.000)
        ( 6, 1.047)
        ( 7, 1.076)
        ( 8, 1.095)
        ( 9, 1.107)
        (10, 1.114)
        (11, 1.119)
        (12, 1.122)
      };
      \addplot[ybar stacked,draw=none,fill=blue!30] coordinates {
        ( 0, 4.925)
        ( 1, 4.591)
        ( 2, 4.349)
        ( 3, 4.183)
        ( 4, 4.072)
        ( 5, 4.000)
        ( 6, 3.953)
        ( 7, 3.924)
        ( 8, 3.905)
        ( 9, 3.893)
        (10, 3.886)
        (11, 3.881)
        (12, 3.878)
      };
      \addplot[only marks,mark size=1.5pt,myviolet!50!black] coordinates {
        ( 0,-0.113+5)
        ( 1,-0.782+5)
        ( 2,-1.117+5)
        ( 3,-1.358+5)
        ( 4,-1.524+5)
        ( 5,-1.635+5)
        ( 6,-1.707+5)
        ( 7,-1.754+5)
        ( 8,-1.783+5)
        ( 9,-1.802+5)
        (10,-1.814+5)
        (11,-1.821+5)
        (12,-1.826+5)
      };
      \addplot[only marks,mark size=1.5pt,mygreen!50!black] coordinates {
        ( 0,-0.038+5)
        ( 1,-0.037+5)
        ( 2,-0.185+5)
        ( 3,-0.276+5)
        ( 4,-0.332+5)
        ( 5,-0.365+5)
        ( 6,-0.386+5)
        ( 7,-0.399+5)
        ( 8,-0.407+5)
        ( 9,-0.412+5)
        (10,-0.415+5)
        (11,-0.417+5)
        (12,-0.418+5)
      };
      \addplot[only marks,mark size=1.5pt,myred!50!black] coordinates {
        ( 0,0.038+5)
        ( 1,0.037+5)
        ( 2,0.185+5)
        ( 3,0.276+5)
        ( 4,0.332+5)
        ( 5,0.365+5)
        ( 6,0.386+5)
        ( 7,0.399+5)
        ( 8,0.407+5)
        ( 9,0.412+5)
        (10,0.415+5)
        (11,0.417+5)
        (12,0.418+5)
      };
      \addplot[only marks,mark size=1.5pt,myblue!50!black] coordinates {
        ( 0,0.113+5)
        ( 1,0.782+5)
        ( 2,1.117+5)
        ( 3,1.358+5)
        ( 4,1.524+5)
        ( 5,1.635+5)
        ( 6,1.707+5)
        ( 7,1.754+5)
        ( 8,1.783+5)
        ( 9,1.802+5)
        (10,1.814+5)
        (11,1.821+5)
        (12,1.826+5)
      };
    \end{axis}
  \end{tikzpicture}}\\[3ex]
  \hfitbox{\linewidth}{\begin{tikzpicture}
    \begin{axis}[
        xmin=-1,xmax=13,ymin=0,ymax=10,
        compat=newest,
        width=10cm,height=5cm,
        axis lines=none,
        after end axis/.code={
          \draw[->,style={-{Latex[length=3.5mm,width=2.0mm]},thick}] 
          (axis cs: -1,0) -- (axis cs: 13,0);
          \draw[->,style={-{Latex[length=3.5mm,width=2.0mm]},thick}] 
          (axis cs: -0.5,0) -- (axis cs: -0.5,10);
          \node[anchor=north west,xshift=.5ex] at (axis cs:-0.5,10) {SNR};
          },
      ]
      \draw[dashed,black] (axis cs:-0.5,7.541) -- (axis cs:12,7.541);
      \node[anchor=north west] at (axis cs:-0.5,7.541) {7.54\,dB};
      \addplot+[mark size=1.5pt] coordinates {
        ( 0, 0.691)
        ( 1, 4.455)
        ( 2, 6.061)
        ( 3, 6.906)
        ( 4, 7.284)
        ( 5, 7.440)
        ( 6, 7.501)
        ( 7, 7.526)
        ( 8, 7.535)
        ( 9, 7.539)
        (10, 7.541)
        (11, 7.541)
        (12, 7.541)
      };
    \end{axis}
  \end{tikzpicture}}
\end{minipage}%
\end{frame}





\end{document}
