

\begin{frame}{The Discrete Cosine Transform (DCT)}
  \vspace{-1ex}
  \STRUC{Transform Matrix of the Discrete Cosine Transform (DCT)}
  \bit
\item The DCT is an orthogonal transform
\item<+-> The transform matrix $\ve{A}_{DCT}=\{a_{kn}\}$ has the elements
  $$
  a_{kn}=\alpha_{k}\cdot\cos\left(\frac{\pi}{N}\,k\left(n+\frac{1}{2}\right)\right)
  \qquad\text{with}\qquad
  \alpha_k=\left\{\begin{array}{rcl}
  \sqrt{1/N} &:& k=0\\[.5ex]
  \sqrt{2/N} &:& k\neq0\\
  \end{array}\right.
  $$
\item<+-> The basis vectors $\ve{b}_k=\{a_{kn}\}$ represent sampled cosine functions of different frequencies
  \eit
%  \onslide<+->
%  \medskip
%  \STRUC{Relation to KLT}
%  \bit
%  \item Unit-norm eigenvectors of $\ve{C}_{\!SS}$ approach DCT basis vectors for $\varrho\rightarrow1$
%  \eit

  \onslide<+->
  \medskip
  \STRUC{Advantages of DCT}
  \bit
\item Transform matrix does not depends on the input signal
\item Fast algorithms for computing the forward and inverse transforms
  \eit
  \onslide
\end{frame}

