%\documentclass{beamer}
%\usepackage{amsmath}
%\usepackage{amsfonts}
%\usepackage{amsthm}
%\usepackage{amssymb}
%\usepackage{tikz}
%\usetikzlibrary{trees}

%===== main document class =====
%\ifdefined\slideModeHandout
\documentclass%
[%
  %handout,          % avoid unnecessary overlays
  aspectratio=169,  % aspect ratio fo 16:9 
  t,                % place content at top of frames
  10pt,             % use 10pt as standard font size (default size is 11pt)
  compress,         % compress things like navigation bars...
]{beamer}
%\else
%\documentclass%
%[%
%  aspectratio=169,  % aspect ratio fo 16:9 
%  t,                % place content at top of frames
%  10pt,             % use 10pt as standard font size (default size is 11pt)
%  compress,         % compress things like navigation bars...
%]{beamer}
%\fi

\mode<presentation>
%===============tabto==============
% tabto.sty
%
% version 1.4  (Dec 2018)
%
% Tabbing to fixed positions in a paragraph.
%
% Copyright 2006,2009,2012,2013,2018 by 
% Donald Arseneau,   Vancouver, Canada (asnd@triumf.ca)
% Permission to use, distribute and modify this software is granted
% under the conditions of the LaTeX Project Public License, either 
% version 1.3 or (at your option) any later version.  The license is
% found at http://www.latex-project.org/lppl.txt, and is part of all 
% recent distributions of LaTeX.
%
% This work has the LPPL maintenance status `maintained' (by author).
%
% Two new text positioning commands are defined: \tabto and \tab.
% 
% \tabto{<length>}
% Tab to a position relative to the left margin in a paragraph (any
% indentation due to a list or \leftskip is part of the `margin' in
% this context). If the text on the line already goes past the desired
% position, the tab starts a new line and moves to the requested
% horizontal position.
%
% \tabto*{<length>}
% Similar to \tabto, except it will perform backspacing, and over-
% print previous text on the line whenever that text is already
% longer than the specified length (i.e., no linebreak is produced).
% Line-breaks are suppressed immediately after \tabto or \tabto*.
%
% The length register "\CurrentLineWidth" will report the width
% of the existing text on the line, and it may be used in the
% <length> argument (using calc.sty, for example). Also, there
% is "\TabPrevPos" which gives the "\CurrentLineWidth" from the
% previous tab command (the position where the tab command occurred,
% not where it went to), and can be used to return to that position
% if no line breaks have occurred in between, or directly below it,
% if there were line breaks.
%
% \tab
% Tab to the next tab-stop chosen from a list of tab positions, in
% the traditional style of typewriters.  A \tab will always move
% to the next tab stop (or the next line), even if it is already
% exactly at a tab stop. Thus, "\tab" at the beginning of a line,
% or "\tab\tab" elsewhere skips a position. A linebreak is permitted 
% immediately following a \tab, in case the ensuing text does not 
% fit well in the remaining space.
%
% If you do not want to skip positions, use "\tabto{\NextTabStop}"
% instead of "\tab".  This is particularly useful when you want to
% use \tab in some other command, but do not want to skip a column
% for the first item.
%
% The tab-stop positions are declared using either \TabPositions
% or \NumTabs:
%
% \TabPositions{<length>, <length>,...<length>}
% Declares the tab stops as a comma-separated list of positions 
% relative to the left margin. A tab-stop at 0pt is implicit, and 
% need not be listed.
%
% \NumTabs{<number>}
% Declares a list of <number> equally-spaced tabs, starting at the
% left margin and spanning \linewidth.  For example \NumTabs{2} 
% declares tab-stops at 0pt and 0.5\linewidth, the same as
% \TabPositions{0pt, 0.5\linewidth} or \TabPositions{0.5\linewidth}
%
% After these declarations, the list of tab positions is saved in
% \TabStopList, and the next tab position, relative to the current 
% position, is given by \NextTabStop.  You do not normally need
% to access them, but they are available.
%
% Problems:
%
% Tall objects after a tab stop may overlap the line above, rather
% than forcing a greater separation between lines.

%\ProvidesPackage{tabto}[2018/12/28 \space v 1.4 \space 
%  Another tabbing mechanism]\relax

%%%%%%%%%Code Begin

%\newdimen\CurrentLineWidth
%\newdimen\TabPrevPos
%
%\newcommand\tabto[1]{%
% \leavevmode
% \begingroup
% \def\@tempa{*}\def\@tempb{#1}%
% \ifx\@tempa\@tempb % \tab* 
%   \endgroup
%   \TTo@overlaptrue % ... set a flag and re-issue \tabto to get argument
%   \expandafter\tabto
% \else
%   \ifinner % in a \hbox, so ignore
%   \else % unrestricted horizontal mode
%     \null% \predisplaysize will tell the position of this box (must be box)
%     \parfillskip\fill
%     \everydisplay{}\everymath{}%
%     \predisplaypenalty\@M \postdisplaypenalty\@M
%     $$% math display so we can test \predisplaysize
%      \lineskiplimit=-999pt % so we get pure \baselineskip
%      \abovedisplayskip=-\baselineskip \abovedisplayshortskip=-\baselineskip
%      \belowdisplayskip\z@skip \belowdisplayshortskip\z@skip
%      \halign{##\cr\noalign{%
%        % get the width of the line above
%        \ifdim\predisplaysize=\maxdimen %\message{Mixed R and L, so say the line is full. }%
%           \CurrentLineWidth\linewidth
%        \else
%           \ifdim\predisplaysize=-\maxdimen 
%             % \message{Not in a paragraph, so call the line empty. }%
%             \CurrentLineWidth\z@
%           \else
%             \ifnum\TTo@Direction<\z@
%               \CurrentLineWidth\linewidth \advance\CurrentLineWidth\predisplaysize
%             \else
%               \CurrentLineWidth\predisplaysize 
%             \fi
%             % Correct the 2em offset
%             \advance\CurrentLineWidth -2em
%             \advance\CurrentLineWidth -\displayindent
%             \advance\CurrentLineWidth -\leftskip
%        \fi\fi
%        \ifdim\CurrentLineWidth<\z@ \CurrentLineWidth\z@\fi
%        % Enshrine the tab-to position; #1 might reference \CurrentLineWidth
%        \setlength\@tempdimb{#1}% allow calc.sty
%        %\message{*** Tab to \the\@tempdimb, previous width is \the\CurrentLineWidth. ***}%
%        % Save width for possible return use
%        \global\TabPrevPos\CurrentLineWidth
%        % Build the action to perform
%        \protected@xdef\TTo@action{%
%           \vrule\@width\z@\@depth\the\prevdepth
%           \ifdim\CurrentLineWidth>\@tempdimb
%              \ifTTo@overlap\else
%                 \protect\newline \protect\null
%           \fi\fi
%           \protect\nobreak
%           \protect\hskip\the\@tempdimb\relax
%        }%
%        %\message{\string\TTo@action: \meaning \TTo@action. }%
%        % get back to the baseline, regardless of its depth.
%        \vskip-\prevdepth
%        \prevdepth-99\p@
%        \vskip\prevdepth
%      }}%
%      $$
%     % Don't count the display as lines in the paragraph
%     \count@\prevgraf \advance\count@-4 \prevgraf\count@
%     \TTo@action
%     %%   \penalty\@m % to allow a penalized line break
%   \fi
%   \endgroup
%   \TTo@overlapfalse
%   \ignorespaces
% \fi
%}
%
%% \tab -- to the next position
%% \hskip so \tab\tab moves two positions
%% Allow a (penalized but flexible) line-break right after the tab.
%%
%\newcommand\tab{\leavevmode\hskip2sp\tabto{\NextTabStop}%
%  \nobreak\hskip\z@\@plus 30\p@\penalty4000\hskip\z@\@plus-30\p@\relax}
%
%
%% Expandable macro to select the next tab position from the list
%
%\newcommand\NextTabStop{%
%  \expandafter \TTo@nexttabstop \TabStopList,\maxdimen,>%
%}
%
%\def\TTo@nexttabstop #1,{%
%    \ifdim#1<\CurrentLineWidth
%      \expandafter\TTo@nexttabstop
%    \else
%      \ifdim#1<0.9999\linewidth#1\else\z@\fi
%      \expandafter\strip@prefix
%    \fi
%}
%\def\TTo@foundtabstop#1>{}
%
%\newcommand\TabPositions[1]{\def\TabStopList{\z@,#1}}
%
%\newcommand\NumTabs[1]{%
%   \def\TabStopList{}%
%   \@tempdimb\linewidth 
%   \divide\@tempdimb by#1\relax
%   \advance\@tempdimb 1sp % counteract rounding-down by \divide
%   \CurrentLineWidth\z@
%   \@whiledim\CurrentLineWidth<\linewidth\do {%
%     \edef\TabStopList{\TabStopList\the\CurrentLineWidth,}%
%     \advance\CurrentLineWidth\@tempdimb
%   }%
%   \edef\TabStopList{\TabStopList\linewidth}%
%}
%
% %default setting of tab positions:
%\TabPositions{\parindent,.5\linewidth}
%
%%\newif\ifTTo@overlap \TTo@overlapfalse
%
%%\@ifundefined{predisplaydirection}{
%% \let\TTo@Direction\predisplaysize
%% \let\predisplaydirection\@undefined
%%}{
%% \let\TTo@Direction\predisplaydirection
%%}
%
%

% ===== include packages =====
\usepackage{lmodern}
\usepackage[utf8]{inputenc}      % Unicode UTF-8 encoding support
\usepackage[T2A,T1]{fontenc}         % T1 font encoding
\usepackage{etoolbox}            % Programming tools (used for \insertpartstartframe, \insertpartendframe)
\usepackage{ifthen}              % Simple conditional statements
\usepackage{amsmath}             % AMS math package
\usepackage{amssymb}             % Extended collection of math symbols
\usepackage{mathtools}           % More math stuff
\usepackage{nicefrac}            % Nice fractions
\usepackage{xcolor}              % Driver independent colors
\usepackage{colortbl}            % rowcolor for tables
\usepackage{array}               % tables and arrays with extended features (e.g., overlays)
\usepackage{makecell}            % Simple formatting of single table cells
\usepackage{multirow}            % Multi-rows in tabular environments
\usepackage{booktabs}            % More flexible lines for tabular environments
\usepackage{tabto}               % Easy way of specifying tabulators
\usepackage{adjustbox}           % Macros for adjusting boxed content (used in defining \fitbox)
\usepackage{relsize}             % Relative font sizes (\larger=\relsize{1}, \smaller=\relsize{-1})
\usepackage{graphicx}            % Inserting pictures
\usepackage{hyperref}            % Cross referencing
%\usepackage{media9}              % Include media objects
%\usepackage{animate}             % Animations
\usepackage{tikz}                % TikZ library for drawing
\usepackage{pgfplots}            % Plots
\usepackage{import}              % including stuff with relative paths
%\usepackage{listings}            % source code inclusion

%===== TikZ libraries =====
\usetikzlibrary{shapes}          % Additional shapes: Ellipse
\usetikzlibrary{shapes.symbols}  % Additional shapes: Symbols (e.g., "cloud")
\usetikzlibrary{shapes.arrows}   % Additional shapes: Arrows (e.g., "single arrow")
\usetikzlibrary{arrows}          % Arrow tips
\usetikzlibrary{arrows.meta}     % Adjustable arrow heads
\usetikzlibrary{positioning}     % Relative positioning of nodes

%===== pgfsetting =====
\pgfplotsset{compat=newest}      % Use newest version of pgf
\usepgfplotslibrary{fillbetween} % filling between curves

%===== nice tables =====
\setlength{\heavyrulewidth}{0.08em}
\setlength{\lightrulewidth}{0.08em}
\setlength{\cmidrulewidth}{0.04em}
\setlength{\aboverulesep}{0.4ex}
\setlength{\belowrulesep}{0.6ex}
\newcommand{\cmidbeg}{\addlinespace[0.40ex]}
\newcommand{\cmidend}{\addlinespace[0.10ex]}

\usepackage{caption}
\usepackage{subcaption}


%%%%%%%%%%%%%%%%%%%%%%%%%%%%%%%%%%%%%%%%%%%%%%%%%%%%%%%%%%%%
%%%%%%
%%%%%%     M A I N   D O C U M E N T   S W I T C H E S     
%%%%%%
%%%%%%%%%%%%%%%%%%%%%%%%%%%%%%%%%%%%%%%%%%%%%%%%%%%%%%%%%%%%

% define command for directly using switches
\newcommand{\usetoggle}[1]{\iftoggle{#1}{true}{false}}

% define switches
\newtoggle{useNavSymbols}               % display of navigation symbols
\newtoggle{useShadows}                  % use blocks with shadows
\newtoggle{useColorBlocks}              % use colored blocks
\newtoggle{useColorBlockTitles}         % use colored block titles
\newtoggle{useInverseBlockTitles}       % use colored block title background with white text
\newtoggle{altColors}                   % use alternative color theme
\newtoggle{addExercises}                % whether exercises are used
\newtoggle{specialHeiko}                % special stuff for Heiko
\newtoggle{specialThomas}               % special stuff for Thomas

\def\slideStyleThomas{..}


\ifdefined\slideStyleThomas

  %>>>>>>>>>> parameters for Thomas >>>>>>>>>>
  \title%
      [Image and Video Coding]%
      {Image and Video Coding I:\\[0.5ex] Fundamentals}
  \author%
      [T. Wiegand, J. Pfaff, J. Rasch]%
      {Thomas Wiegand, Jonathan Pfaff, Jennifer Rasch}
  \institute%
      [TU Berlin, Fraunhofer HHI]%
      {Technische Universität Berlin, Fraunhofer Heinrich Hertz Institute, Berlin}

  \newcommand{\slideOrganization}{}

  \settoggle{useNavSymbols}        {false}
  \settoggle{useShadows}           {true}
  \settoggle{useColorBlocks}       {true}
  \settoggle{useColorBlockTitles}  {true}
  \settoggle{useInverseBlockTitles}{true}
  \settoggle{altColors}            {false}
  \settoggle{addExercises}         {false}

  \settoggle{specialHeiko}         {false}
  \settoggle{specialThomas}        {true}
  %<<<<<<<<<< parameters for Thomas <<<<<<<<<<

\else

  %>>>>>>>>>> parameters for Heiko >>>>>>>>>>
  \title%
      {Data Compression}
  \author%
      [Heiko Schwarz]%
      {Heiko Schwarz}
  \institute%
      [Freie Universität Berlin]%
      {Freie Universität Berlin\\%
      Fachbereich Mathematik und Informatik}

\ifdefined\slideModeHandout
  \settoggle{useNavSymbols}        {false}
\else
  \settoggle{useNavSymbols}        {false}
\fi
  \settoggle{useShadows}           {true}
  \settoggle{useColorBlocks}       {true}
  \settoggle{useColorBlockTitles}  {true}
  \settoggle{useInverseBlockTitles}{true}
  \settoggle{altColors}            {false}
  \settoggle{addExercises}         {true}

  \settoggle{specialHeiko}         {true}
  \settoggle{specialThomas}        {false}
  %<<<<<<<<<< parameters for Heiko <<<<<<<<<<

\fi % end of conditional





%%%%%%%%%%%%%%%%%%%%%%%%%%%%%%%%%%%%%%%%%%%%%%%%%
%%%%%%
%%%%%%     C U S T O M I Z E   D E S I G N
%%%%%%
%%%%%%%%%%%%%%%%%%%%%%%%%%%%%%%%%%%%%%%%%%%%%%%%%

%===== spacing for lists and paragraphs (modified copy from beamerbaselocalstructure.sty) =====
\makeatletter
\setlength  \parskip         {2ex}
\setlength  \leftmargini     {2em}
\setlength  \leftmarginii    {2em}
\setlength  \leftmarginiii   {2em}
\setlength  \labelsep        {.5em}
\setlength  \labelwidth      {\leftmargini}
\addtolength\labelwidth      {-\labelsep}
\def\@listi  {\leftmargin\leftmargini
              \partopsep  \parskip
              \parskip    0.0\p@
              \parsep     0.0\p@
              \topsep     3.0\p@ \@plus1.0\p@ \@minus2.0\p@
              \itemsep    3.0\p@ \@plus1.0\p@ \@minus2.0\p@}
\let\@listI\@listi
\def\@listii {\leftmargin\leftmarginii
              \parsep     0.0\p@
              \topsep     3.0\p@ \@plus1.0\p@ \@minus2.0\p@
              \itemsep    3.0\p@ \@plus1.0\p@ \@minus2.0\p@}
\def\@listiii{\leftmargin\leftmarginiii
              \parsep     0.0\p@
              \topsep     3.0\p@ \@plus1.0\p@ \@minus2.0\p@
              \itemsep    3.0\p@ \@plus1.0\p@ \@minus2.0\p@}
\makeatother


%===== counter for exercises =====
\newcounter{exercise}


%===== enumerate symbols (modified copy from beamerbaseauxtemplates.sty) =====
\makeatletter

%--- define commands for changing enum style ---
\newcommand{\setenumstyledepth}[2]{% {enumdepth}{command for displaying counter}
  \ifthenelse{\equal{#1}{1}}%
     {\renewcommand*{\theenumi}{#2{enumi}}}%
     {\ifthenelse{\equal{#1}{2}}%
        {\renewcommand*{\theenumii}{#2{enumii}}}%
        {\renewcommand*{\theenumiii}{#2{enumiii}}}%
     }%
}
% Note: For the following command you can also use your own styles.
%       For example, a style "A." in a smaller font can be defined by
%         \newcommand{\AlphaDot}[1]{{\smaller\smaller\Alph{#1}.}}
\newcommand{\enumStyle}[1]{\setenumstyledepth{\the\@enumdepth}{#1}}
\newcommand{\enumStylesDefault}[3]{%
  \setenumstyledepth{1}{#1}%
  \setenumstyledepth{2}{#2}%
  \setenumstyledepth{3}{#3}%
}

%--- define commands for putting enum symbols ---
\newcommand{\putenumsquare}[1]{%
  \smaller%
  \usebeamercolor[bg]{\beameritemnestingprefix item projected}%
  \begin{pgfpicture}{-1ex}{-0.25ex}{1.1ex}{2.0ex}%
    \pgfpathrectangle{\pgfpoint{-1.2ex}{-0.6ex}}{\pgfpoint{2.4ex}{2.4ex}}%
    \pgfusepath{fill}%
    \pgftext[base,y=-0.15ex]{\color{fg}#1}%
  \end{pgfpicture}%
}
\newcommand{\putenumcircle}[1]{%
  \smaller%
  \usebeamercolor[bg]{\beameritemnestingprefix item projected}%
  \begin{pgfpicture}{-1ex}{-0.25ex}{1.1ex}{2.0ex}%
    \pgfpathcircle{\pgfpoint{0pt}{.6ex}}{1.3ex}%
    \pgfusepath{fill}%
    \pgftext[base,y=-0.15ex]{\color{fg}#1}%
  \end{pgfpicture}%
}
\newcommand{\putenumblank}[1]{%
  \smaller%
  \begin{pgfpicture}{-1ex}{-0.25ex}{1.1ex}{2.0ex}%
    \pgfpathrectangle{\pgfpoint{-1.2ex}{-0.6ex}}{\pgfpoint{2.4ex}{2.4ex}}%
    \pgftext[base,y=-0.15ex]{#1}%
  \end{pgfpicture}%
}
\newcommand{\putenumbracket}[1]{%
  \smaller%
  \begin{pgfpicture}{-1ex}{-0.25ex}{1.1ex}{2.0ex}%
    \pgfpathrectangle{\pgfpoint{-1.2ex}{-0.6ex}}{\pgfpoint{2.4ex}{2.4ex}}%
    \pgftext[base,y=-0.15ex]{$\boldsymbol{\langle}$#1$\boldsymbol{\rangle}$}%
  \end{pgfpicture}%
}
\newcommand{\putenumautoi}[1]{\putenumcircle{#1}}
\newcommand{\putenumautoii}[1]{\putenumcircle{#1}}
\newcommand{\putenumautoiii}[1]{\putenumcircle{#1}}
\newcommand{\putenumauto}[1]{%
  \ifthenelse{\equal{\the\@itemdepth}{1}}%
     {\putenumautoi{#1}}%
     {\ifthenelse{\equal{\the\@itemdepth}{2}}%
        {\putenumautoii{#1}}%
        {\putenumautoiii{#1}}%
     }%
}

%--- define beamer enum templates [square][circle][blank][bracket][auto] ---
\expandafter\let\csname beamer@@tmpop@enumerate item@square\endcsname\relax
\expandafter\let\csname beamer@@tmpop@enumerate subitem@square\endcsname\relax
\expandafter\let\csname beamer@@tmpop@enumerate subsubitem@square\endcsname\relax
\expandafter\let\csname beamer@@tmpop@enumerate mini template@square\endcsname\relax

\expandafter\let\csname beamer@@tmpop@enumerate item@circle\endcsname\relax
\expandafter\let\csname beamer@@tmpop@enumerate subitem@circle\endcsname\relax
\expandafter\let\csname beamer@@tmpop@enumerate subsubitem@circle\endcsname\relax
\expandafter\let\csname beamer@@tmpop@enumerate mini template@circle\endcsname\relax

\defbeamertemplate{enumerate item}{square}{\putenumsquare{\insertenumlabel}}
\defbeamertemplate{enumerate subitem}{square}{\putenumsquare{\insertsubenumlabel}}
\defbeamertemplate{enumerate subsubitem}{square}{\putenumsquare{\insertsubsubenumlabel}}
\defbeamertemplate{enumerate mini template}{square}{\putenumsquare{\insertenumlabel}}

\defbeamertemplate{enumerate item}{circle}{\putenumcircle{\insertenumlabel}}
\defbeamertemplate{enumerate subitem}{circle}{\putenumcircle{\insertsubenumlabel}}
\defbeamertemplate{enumerate subsubitem}{circle}{\putenumcircle{\insertsubsubenumlabel}}
\defbeamertemplate{enumerate mini template}{circle}{\putenumcircle{\insertenumlabel}}

\defbeamertemplate{enumerate item}{blank}{\putenumblank{\insertenumlabel}}
\defbeamertemplate{enumerate subitem}{blank}{\putenumblank{\insertsubenumlabel}}
\defbeamertemplate{enumerate subsubitem}{blank}{\putenumblank{\insertsubsubenumlabel}}
\defbeamertemplate{enumerate mini template}{blank}{\putenumblank{\insertenumlabel}}

\defbeamertemplate{enumerate item}{bracket}{\putenumbracket{\insertenumlabel}}
\defbeamertemplate{enumerate subitem}{bracket}{\putenumbracket{\insertsubenumlabel}}
\defbeamertemplate{enumerate subsubitem}{bracket}{\putenumbracket{\insertsubsubenumlabel}}
\defbeamertemplate{enumerate mini template}{bracket}{\putenumbracket{\insertenumlabel}}

\defbeamertemplate{enumerate item}{auto}{\putenumauto{\insertenumlabel}}
\defbeamertemplate{enumerate subitem}{auto}{\putenumauto{\insertsubenumlabel}}
\defbeamertemplate{enumerate subsubitem}{auto}{\putenumauto{\insertsubsubenumlabel}}
\defbeamertemplate{enumerate mini template}{auto}{\putenumauto{\insertenumlabel}}

%--- define commands for easily changing enum modes ---
% The outcommented simple version has a problem with enum nested in itemize (wrong level)
%   \newcommand{\enumMode}[1]{\setbeamertemplate{enumerate \beameritemnestingprefix item}[#1]} 
\newcommand{\enumMode}[1]{%
  \ifthenelse{\equal{\the\@enumdepth}{1}}%
     {\setbeamertemplate{enumerate item}[#1]}%
     {\ifthenelse{\equal{\the\@enumdepth}{2}}%
        {\setbeamertemplate{enumerate subitem}[#1]}%
        {\setbeamertemplate{enumerate subsubitem}[#1]}%
     }%
}
\newcommand{\enumAutoDefault}[3]{%
  \renewcommand*{\putenumautoi}[1]{\csname putenum#1\endcsname{##1}}%
  \renewcommand*{\putenumautoii}[1]{\csname putenum#2\endcsname{##1}}%
  \renewcommand*{\putenumautoiii}[1]{\csname putenum#3\endcsname{##1}}%
}

\makeatother



%===== itemize symbols (modified copy from beamerbaseauxtemplates.sty) =====
\makeatletter

%--- new item symbols ---
\newcommand{\isquare}{%
  \begin{pgfpicture}%
    \pgfpathrectangle{\pgfpointorigin}{\pgfpoint{1.0ex}{1.0ex}}%
    \pgfusepath{fill}%
    \pgfsetbaseline{-0.2ex}%
  \end{pgfpicture}%
}
\newcommand{\icircle}{%
  \begin{pgfpicture}%
    \pgfpathcircle{\pgfpoint{0pt}{.5ex}}{0.5ex}%
    \pgfusepath{fill}%
    \pgfsetbaseline{-0.2ex}%
  \end{pgfpicture}%
}
\newcommand{\itriangle}{%
  \begin{pgfpicture}%
    \pgfpathmoveto{\pgfpointorigin}
    \pgfpathlineto{\pgfpoint{0.0ex}{1.0ex}}%
    \pgfpathlineto{\pgfpoint{1.0ex}{0.5ex}}%
    \pgfusepath{fill}%
    \pgfsetbaseline{-0.2ex}%
  \end{pgfpicture}%
}
\newcommand{\idash}{%
  \begin{pgfpicture}%
    \pgfpathrectangle{\pgfpoint{0.0ex}{0.4ex}}{\pgfpoint{1.0ex}{0.2ex}}%
    \pgfusepath{fill}%
    \pgfsetbaseline{-0.2ex}%
  \end{pgfpicture}%
}
\newcommand{\iarrow}{%
  \begin{pgfpicture}%
    \pgfpathmoveto{\pgfpointorigin}
    \pgfpathlineto{\pgfpoint{-0.80ex}{ 0.75ex}}%    (-hl)( ht)  % tl =      total length
    \pgfpathlineto{\pgfpoint{-0.80ex}{ 0.25ex}}%    (-hl)( tt)  % tt = half total thickness
    \pgfpathlineto{\pgfpoint{-2.00ex}{ 0.25ex}}%    (-tl)( tt)  % hl =      head length
    \pgfpathlineto{\pgfpoint{-2.00ex}{-0.25ex}}%    (-tl)(-tt)  % ht = half head thickness
    \pgfpathlineto{\pgfpoint{-0.80ex}{-0.25ex}}%    (-hl)(-tt)
    \pgfpathlineto{\pgfpoint{-0.80ex}{-0.75ex}}%    (-hl)(-ht)
    \pgfusepath{fill}%
  \end{pgfpicture}%
}
\newcommand{\idarrow}{%
  \begin{pgfpicture}%
    \pgfpathmoveto{\pgfpointorigin}
    \pgfpathlineto{\pgfpoint{-0.80ex}{ 0.75ex}}%    (-hl)( ht)  % tl =      total length
    \pgfpathlineto{\pgfpoint{-0.80ex}{ 0.25ex}}%    (-hl)( tt)  % tt = half total thickness
    \pgfpathlineto{\pgfpoint{-2.00ex}{ 0.25ex}}%    (-tl)( tt)  % hl =      head length
    \pgfpathlineto{\pgfpoint{-2.00ex}{ 0.75ex}}%
    \pgfpathlineto{\pgfpoint{-2.80ex}{ 0.00ex}}%
    \pgfpathlineto{\pgfpoint{-2.00ex}{-0.75ex}}%
    \pgfpathlineto{\pgfpoint{-2.00ex}{-0.25ex}}%    (-tl)(-tt)  % ht = half head thickness
    \pgfpathlineto{\pgfpoint{-0.80ex}{-0.25ex}}%    (-hl)(-tt)
    \pgfpathlineto{\pgfpoint{-0.80ex}{-0.75ex}}%    (-hl)(-ht)
    \pgfusepath{fill}%
  \end{pgfpicture}%
}

%--- define beamer item templates [square][circle][triangle][dash][arrow] ---
\expandafter\let\csname beamer@@tmpop@itemize item@square\endcsname\relax
\expandafter\let\csname beamer@@tmpop@itemize subitem@square\endcsname\relax
\expandafter\let\csname beamer@@tmpop@itemize subsubitem@square\endcsname\relax

\expandafter\let\csname beamer@@tmpop@itemize item@circle\endcsname\relax
\expandafter\let\csname beamer@@tmpop@itemize subitem@circle\endcsname\relax
\expandafter\let\csname beamer@@tmpop@itemize subsubitem@circle\endcsname\relax

\expandafter\let\csname beamer@@tmpop@itemize item@triangle\endcsname\relax
\expandafter\let\csname beamer@@tmpop@itemize subitem@triangle\endcsname\relax
\expandafter\let\csname beamer@@tmpop@itemize subsubitem@triangle\endcsname\relax

\defbeamertemplate{itemize item}{square}{\isquare}
\defbeamertemplate{itemize subitem}{square}{\isquare}
\defbeamertemplate{itemize subsubitem}{square}{\isquare}

\defbeamertemplate{itemize item}{circle}{\icircle}
\defbeamertemplate{itemize subitem}{circle}{\icircle}
\defbeamertemplate{itemize subsubitem}{circle}{\icircle}

\defbeamertemplate{itemize item}{triangle}{\itriangle}
\defbeamertemplate{itemize subitem}{triangle}{\itriangle}
\defbeamertemplate{itemize subsubitem}{triangle}{\itriangle}

\defbeamertemplate{itemize item}{dash}{\idash}
\defbeamertemplate{itemize subitem}{dash}{\idash}
\defbeamertemplate{itemize subsubitem}{dash}{\idash}

\defbeamertemplate{itemize item}{arrow}{\iarrow}
\defbeamertemplate{itemize subitem}{arrow}{\iarrow}
\defbeamertemplate{itemize subsubitem}{arrow}{\iarrow}

%--- define command for easily changing item styles ---
\newcommand{\itemMode}[1]{\setbeamertemplate{itemize \beameritemnestingprefix item}[#1]}

\makeatother



%===== commands for text highlighting  =====
% helping command
\newcommand{\setfontrm} {\fontshape{\rmdefault}\selectfont}
\newcommand{\setfontit} {\fontshape{\itdefault}\selectfont}
\newcommand{\setfontrs} {\fontseries{\mddefault}\selectfont}
\newcommand{\setfontbs} {\fontseries{\bfdefault}\selectfont}
\newcommand{\setfontrrm}{\fontseries{\mddefault}\fontshape{\rmdefault}\selectfont}
\newcommand{\setfontrit}{\fontseries{\mddefault}\fontshape{\itdefault}\selectfont}
\newcommand{\setfontbrm}{\fontseries{\bfdefault}\fontshape{\rmdefault}\selectfont}
\newcommand{\setfontbit}{\fontseries{\bfdefault}\fontshape{\itdefault}\selectfont}
% normal text attributes:
  % setting series
  \newcommand<>{\regu}  [1]{{\only#2{\setfontrs}#1}}
  \newcommand<>{\bold}  [1]{{\only#2{\setfontbs}#1}}
  % setting shape
  \newcommand<>{\norm}  [1]{{\only#2{\setfontrm}#1}}
  \newcommand<>{\ital}  [1]{{\only#2{\setfontit}#1}}
  % setting series and shape
  \newcommand<>{\rnorm} [1]{{\only#2{\setfontrrm}#1}}
  \newcommand<>{\rital} [1]{{\only#2{\setfontrit}#1}}
  \newcommand<>{\bnorm} [1]{{\only#2{\setfontbrm}#1}}
  \newcommand<>{\bital} [1]{{\only#2{\setfontbit}#1}}
% special text highlighting:
  % changing color only
\renewcommand<>{\alert} [1]{{\only#2{\usebeamercolor{alerted text}\color{fg}}#1}}
  \newcommand<>{\struc} [1]{{\only#2{\usebeamercolor{structure}\color{fg}}#1}}
  \newcommand<>{\examp} [1]{{\only#2{\usebeamercolor{example text}\color{fg}}#1}}
  % changing color and series
  \newcommand<>{\Alert} [1]{{\only#2{\setfontrs\usebeamercolor{alerted text}\color{fg}}#1}}
  \newcommand<>{\Struc} [1]{{\only#2{\setfontrs\usebeamercolor{structure}\color{fg}}#1}}
  \newcommand<>{\Examp} [1]{{\only#2{\setfontrs\usebeamercolor{example text}\color{fg}}#1}}
  \newcommand<>{\ALERT} [1]{{\only#2{\setfontbs\usebeamercolor{alerted text}\color{fg}}#1}}
  \newcommand<>{\STRUC} [1]{{\only#2{\setfontbs\usebeamercolor{structure}\color{fg}}#1}}
  \newcommand<>{\EXAMP} [1]{{\only#2{\setfontbs\usebeamercolor{example text}\color{fg}}#1}}
  % changing color and shape
  \newcommand<>{\ralert}[1]{{\only#2{\setfontrm\usebeamercolor{alerted text}\color{fg}}#1}}
  \newcommand<>{\rstruc}[1]{{\only#2{\setfontrm\usebeamercolor{structure}\color{fg}}#1}}
  \newcommand<>{\rexamp}[1]{{\only#2{\setfontrm\usebeamercolor{example text}\color{fg}}#1}}
  \newcommand<>{\ialert}[1]{{\only#2{\setfontit\usebeamercolor{alerted text}\color{fg}}#1}}
  \newcommand<>{\istruc}[1]{{\only#2{\setfontit\usebeamercolor{structure}\color{fg}}#1}}
  \newcommand<>{\iexamp}[1]{{\only#2{\setfontit\usebeamercolor{example text}\color{fg}}#1}}
  % changing color, shape, and series
  \newcommand<>{\rAlert}[1]{{\only#2{\setfontrrm\usebeamercolor{alerted text}\color{fg}}#1}}
  \newcommand<>{\rStruc}[1]{{\only#2{\setfontrrm\usebeamercolor{structure}\color{fg}}#1}}
  \newcommand<>{\rExamp}[1]{{\only#2{\setfontrrm\usebeamercolor{example text}\color{fg}}#1}}
  \newcommand<>{\rALERT}[1]{{\only#2{\setfontbrm\usebeamercolor{alerted text}\color{fg}}#1}}
  \newcommand<>{\rSTRUC}[1]{{\only#2{\setfontbrm\usebeamercolor{structure}\color{fg}}#1}}
  \newcommand<>{\rEXAMP}[1]{{\only#2{\setfontbrm\usebeamercolor{example text}\color{fg}}#1}}
  \newcommand<>{\iAlert}[1]{{\only#2{\setfontrit\usebeamercolor{alerted text}\color{fg}}#1}}
  \newcommand<>{\iStruc}[1]{{\only#2{\setfontrit\usebeamercolor{structure}\color{fg}}#1}}
  \newcommand<>{\iExamp}[1]{{\only#2{\setfontrit\usebeamercolor{example text}\color{fg}}#1}}
  \newcommand<>{\iALERT}[1]{{\only#2{\setfontbit\usebeamercolor{alerted text}\color{fg}}#1}}
  \newcommand<>{\iSTRUC}[1]{{\only#2{\setfontbit\usebeamercolor{structure}\color{fg}}#1}}
  \newcommand<>{\iEXAMP}[1]{{\only#2{\setfontbit\usebeamercolor{example text}\color{fg}}#1}}
% specials: Emphasizing and names [May want to redefine in actually used style]
\renewcommand<>{\emph}  [1]{\alert#2{#1}}
  \newcommand<>{\Emph}  [1]{\ALERT#2{#1}}
  \newcommand<>{\EMPH}  [1]{\iALERT#2{#1}}
  \newcommand  {\aname} [1]{{\rmfamily\scshape #1}}



%===== define subblock environments =====
\makeatletter
\newenvironment{nesting}{%
  \par\hspace{2\beamer@leftmargin}
  \begin{minipage}{\linewidth-\beamer@leftmargin-3.5\beamer@rightmargin}%
}{%
  \end{minipage}
}
\makeatother



%===== define hooks for accessing frame number inside part  =====
\makeatletter
\newcount\beamer@partstartframe
\beamer@partstartframe=1
\apptocmd{\beamer@part}{\addtocontents{nav}{\protect\headcommand{%
    \protect\beamer@partframes{\the\beamer@partstartframe}{\the\c@framenumber}}}}{}{}
\apptocmd{\beamer@part}{\beamer@partstartframe=\c@framenumber\advance\beamer@partstartframe by1\relax}{}{}
\AtEndDocument{\immediate\write\@auxout{\string\@writefile{nav}%
    {\noexpand\headcommand{\noexpand\beamer@partframes{\the\beamer@partstartframe}{\the\c@framenumber}}}}}{}{}
\def\beamer@startframeofpart{1}
\def\beamer@endframeofpart{1}
\def\beamer@partframes#1#2{%
    \ifnum\c@framenumber<#1%
    \else%
    \ifnum\c@framenumber>#2%
    \else%
    \gdef\beamer@startframeofpart{#1}%
    \gdef\beamer@endframeofpart{#2}%
    \fi%
    \fi%
}
\newcommand\insertpartstartframe{\beamer@startframeofpart}
\newcommand\insertpartendframe{\beamer@endframeofpart}
\makeatother
\def\inserttotalpartframenumber{%
    \pgfmathparse{(\insertpartendframe-\insertpartstartframe+1)}%
    \pgfmathprintnumber[fixed,precision=2]{\pgfmathresult}%
}
\def\insertpartframenumber{%
    \pgfmathparse{(\insertframenumber-\insertpartstartframe+1)}%
    \pgfmathprintnumber[fixed,precision=2]{\pgfmathresult}%
}



%===== define visible on macro for tikz pictures  =====
% see https://tex.stackexchange.com/questions/55806/mindmap-tikzpicture-in-beamer-reveal-step-by-step#55849
\tikzset{
  invisible/.style={opacity=0},
  visible on/.style={alt={#1{}{invisible}}},
  alt/.code args={<#1>#2#3}{%
    \alt<#1>{\pgfkeysalso{#2}}{\pgfkeysalso{#3}} % \pgfkeysalso doesn't change the path
  },
  action/.code args={<#1>#2}{%
    \action<#1>{\pgfkeysalso{#2}} % \pgfkeysalso doesn't change the path
  },
}
% see https://tex.stackexchange.com/questions/6135/how-to-make-beamer-overlays-with-tikz-node
\tikzset{onslide/.code args={<#1>#2}{% \pgfkeysalso doesn't change the path
  \only<#1>{\pgfkeysalso{#2}} %
}}
\tikzset{temporal/.code args={<#1>#2#3#4}{% \pgfkeysalso doesn't change the path
  \temporal<#1>{\pgfkeysalso{#2}}{\pgfkeysalso{#3}}{\pgfkeysalso{#4}} %
}}


%===== further helpful tikz macros =====
\newcommand{\budash}{{\tikz[baseline=0.1ex]\draw[thick](0,0)--({0.6em},0);}}
\newcommand{\nudash}{{\tikz[baseline=0.1ex]\draw[]     (0,0)--({0.6em},0);}}



%===== define a fitbox command  =====
\makeatletter
\newlength{\fitboxw}
\newlength{\fitboxh}
\newlength{\slideinnerheight}
\setlength{\slideinnerheight}{0.85\textheight} %%% could be reset later
\newcommand<>{\fitbox}[4][c]{%
  \only#5{%
    {%
      \setlength{\fitboxw}{#2}%
      \setlength{\fitboxh}{#3}%
      \parbox[#1][\fitboxh]{\fitboxw}{%
        \centering%
        \vfill%
        \adjustbox{%
          min width=\fitboxw,%
          min height=\fitboxh,%
          max width=\fitboxw,%
          max height=\fitboxh%
        }%
        {#4}%
        \vfill%
      }%
    }%
  }%
}
\newcommand<>{\hfitbox}[3][c]{%
  \only#4{%
    {%
      \setlength{\fitboxw}{#2}%
      \parbox[#1]{\fitboxw}{%
        \adjustbox{%
          min width=\fitboxw,%
          max width=\fitboxw}%
        {#3}%
      }%
    }%
  }%
}
\newcommand<>{\slidehfitbox}[1]{%
  \hfitbox#2{\textwidth}{#1}%
}
\newcommand<>{\slidefitbox}[1]{%
  \fitbox#2{\textwidth}{\slideinnerheight}{#1}%
}
\makeatother



%===== command for adding new part with part page / adding title page  =====
\newcommand{\startnewpart}[2][\usebeamercolor{background}\color{bg}\rule{1pt}{1pt}]{
  \part{#2}
  {
    \setbeamertemplate{navigation symbols}{}
    \begin{frame}[plain]
    \vfill\vfill
    {\hfill\begin{beamercolorbox}[%
       sep=8pt,dp=1ex,center,wd=0.8\textwidth,%
       rounded=true,
       shadow=\usetoggle{useShadows}%
    ]%
    {part title}
    \usebeamerfont{part title}\insertpart\par
    \end{beamercolorbox}\hfill}
    \vfill
    {\hfill{\fitbox{0.8\textwidth}{0.45\textheight}{#1}}\hfill}
    \vspace{0pt}
    \end{frame}
  }
}
\newcommand{\addtitlepage}{
  {
    \setbeamertemplate{navigation symbols}{}
    \begin{frame}[plain]
    \vspace{0.15\textheight}
    {\hfill\begin{beamercolorbox}[%
       sep=8pt,dp=1ex,center,wd=0.8\textwidth,%
       rounded=true,
       shadow=\usetoggle{useShadows}%
    ]%
    {title}
    \usebeamerfont{title}\inserttitle\par
    \end{beamercolorbox}\hfill}

    \begin{center}\large
      ~\\[3ex]
      {\insertauthor}\\[3ex]
      {\insertinstitute}
    \end{center}
    \end{frame}
  }
}







%%%%%%%%%%%%%%%%%%%%%%%%%%%%%%%%%%%%%%%%%%%
%%%%%%
%%%%%%     T E M P L A T E
%%%%%%
%%%%%%%%%%%%%%%%%%%%%%%%%%%%%%%%%%%%%%%%%%%

%===== other font size =====
\makeatletter
\newcommand\notsotiny{\@setfontsize\notsotiny\@vipt\@viipt}
\makeatother

%===== font theme =====
\usefonttheme{structurebold}
\setbeamerfont{title in head/foot}{size=\tiny}
\setbeamerfont{author in head/foot}{size=\tiny}
\setbeamerfont{date in head/foot}{size=\tiny}
\setbeamerfont*{frametitle}{family=\sffamily,series=\bfseries,shape=\upshape,size=\large}


%===== color theme (based on color theme "beaver") =====
%>>> define colors (see http://latexcolor.com/ for a good visual comparison)
% reds
\definecolor{bostonuniversityred}     {rgb}{0.80, 0.00, 0.00} % used in beaver
\definecolor{cornellred}              {rgb}{0.70, 0.11, 0.11}
\definecolor{red(ncs)}                {rgb}{0.77, 0.01, 0.20}
\definecolor{carmine}                 {rgb}{0.59, 0.00, 0.09}
\definecolor{crimsonglory}            {rgb}{0.75, 0.00, 0.20}
\definecolor{deepcarmine}             {rgb}{0.66, 0.13, 0.24}
\definecolor{harvardcrimson}          {rgb}{0.79, 0.00, 0.09}
\definecolor{lava}                    {rgb}{0.81, 0.06, 0.13}
\definecolor{mordantred19}            {rgb}{0.68, 0.05, 0.00}
\definecolor{persianred}              {rgb}{0.80, 0.20, 0.20}
\definecolor{raspberry}               {rgb}{0.89, 0.04, 0.36}
% greens                              
\definecolor{othergreen}              {rgb}{0.00, 0.80, 0.00}
\definecolor{officegreen}             {rgb}{0.00, 0.50, 0.00}
\definecolor{darkgreen}               {rgb}{0.00, 0.20, 0.13}
\definecolor{pakistangreen}           {rgb}{0.00, 0.40, 0.00}
\definecolor{cadmiumgreen}            {rgb}{0.00, 0.42, 0.24}
\definecolor{lincolngreen}            {rgb}{0.11, 0.35, 0.02}
\definecolor{dartmouthgreen}          {rgb}{0.05, 0.50, 0.06}
\definecolor{sacramentostategreen}    {rgb}{0.00, 0.34, 0.25}
\definecolor{tropicalrainforest}      {rgb}{0.00, 0.46, 0.37}
\definecolor{upforestgreen}           {rgb}{0.00, 0.27, 0.13}
\definecolor{lasallegreen}            {rgb}{0.03, 0.47, 0.19}
\definecolor{indiagreen}              {rgb}{0.07, 0.53, 0.03}
\definecolor{forestgreen(traditional)}{rgb}{0.00, 0.27, 0.13}
% blues                               
\definecolor{mediumblue}              {rgb}{0.00, 0.00, 0.80}
\definecolor{navyblue}                {rgb}{0.00, 0.00, 0.50}
\definecolor{ceruleanblue}            {rgb}{0.16, 0.32, 0.75}
\definecolor{internationalkleinblue}  {rgb}{0.00, 0.18, 0.65}
\definecolor{royalazure}              {rgb}{0.00, 0.22, 0.66}
\definecolor{smalt(darkpowderblue)}   {rgb}{0.00, 0.20, 0.60}
\definecolor{ultramarine}             {rgb}{0.07, 0.04, 0.56}
\definecolor{zaffre}                  {rgb}{0.00, 0.08, 0.66}
\definecolor{phthaloblue}             {rgb}{0.00, 0.06, 0.54}
\definecolor{persianblue}             {rgb}{0.11, 0.22, 0.73}
\definecolor{palatinateblue}          {rgb}{0.15, 0.23, 0.89}
% other
\definecolor{vividviolet}             {rgb}{0.62, 0.00, 1.00}
\definecolor{uclagold}                {rgb}{1.00, 0.70, 0.00}
\definecolor{tangerineyellow}         {rgb}{1.00, 0.80, 0.00}
\definecolor{shockingpink}            {rgb}{0.99, 0.06, 0.75}
\definecolor{schoolbusyellow}         {rgb}{1.00, 0.85, 0.00}
\definecolor{saddlebrown}             {rgb}{0.55, 0.27, 0.07}
\definecolor{purple(munsell)}         {rgb}{0.62, 0.00, 0.77}
\definecolor{portlandorange}          {rgb}{1.00, 0.35, 0.21}
\definecolor{persianrose}             {rgb}{1.00, 0.16, 0.64}
\definecolor{orange(colorwheel)}      {rgb}{1.00, 0.50, 0.00}
\definecolor{lightcyan}               {rgb}{0.88, 1.00, 1.00}
\definecolor{goldenyellow}            {rgb}{1.00, 0.87, 0.00}
\definecolor{electricindigo}          {rgb}{0.44, 0.00, 1.00}
\definecolor{darkorange}              {rgb}{1.00, 0.55, 0.00}
\definecolor{cyan(process)}           {rgb}{0.00, 0.72, 0.92}
\definecolor{aqua}                    {rgb}{0.00, 1.00, 1.00}
\definecolor{aquamarine}              {rgb}{0.50, 1.00, 0.83}
\definecolor{amber}                   {rgb}{1.00, 0.75, 0.00}
\definecolor{aliceblue}               {rgb}{0.94, 0.97, 1.00}


%===== specify main colors =====
\iftoggle{altColors}{%
  \colorlet{fgColorTheme}{bostonuniversityred}
  \colorlet{bgColorTheme}{gray}
  \colorlet{fgColorStruc}{royalazure}
  \colorlet{bgColorStruc}{bgColorTheme}
  \colorlet{fgColorAlert}{bostonuniversityred}
  \colorlet{bgColorAlert}{fgColorAlert}
  \colorlet{fgColorExamp}{indiagreen}
  \colorlet{bgColorExamp}{fgColorExamp}
}{%
  \colorlet{fgColorTheme}{royalazure}
  \colorlet{bgColorTheme}{gray}
  \colorlet{fgColorStruc}{royalazure}
  \colorlet{bgColorStruc}{bgColorTheme}
  \colorlet{fgColorAlert}{bostonuniversityred}
  \colorlet{bgColorAlert}{fgColorAlert}
  \colorlet{fgColorExamp}{indiagreen}
  \colorlet{bgColorExamp}{fgColorExamp}
}

% derived colors
\colorlet{fgColorThemedark}    {fgColorTheme!80!black}
\colorlet{fgColorThemeDark}    {fgColorTheme!70!black}
\colorlet{fgColorThemeDARK}    {fgColorTheme!60!black}
\colorlet{fgColorThemeDARKER}  {fgColorTheme!60!black}
\colorlet{fgColorThemeDARKEST} {fgColorTheme!60!black}
\colorlet{bgColorThemedark}    {bgColorTheme!60!white}
\colorlet{bgColorThemelight}   {bgColorTheme!30!white}
\colorlet{bgColorThemeLight}   {bgColorTheme!20!white}
\colorlet{bgColorThemeLIGHT}   {bgColorTheme!15!white}
\colorlet{bgColorThemeLIGHTER} {bgColorTheme!10!white}
\colorlet{bgColorThemeLIGHTEST}{bgColorTheme! 5!white}
\colorlet{fgColorStrucdark}    {fgColorStruc!80!black}
\colorlet{bgColorStruclight}   {bgColorStruc!30!white}
\colorlet{bgColorStrucLight}   {bgColorStruc!20!white}
\colorlet{bgColorStrucLIGHT}   {bgColorStruc!15!white}
\colorlet{bgColorStrucLIGHTER} {bgColorStruc!10!white}
\colorlet{bgColorStrucLIGHTEST}{bgColorStruc! 5!white}
\colorlet{fgColorAlertdark}    {fgColorAlert!80!black}
\colorlet{bgColorAlertlight}   {bgColorAlert!30!white}
\colorlet{bgColorAlertLight}   {bgColorAlert!20!white}
\colorlet{bgColorAlertLIGHT}   {bgColorAlert!15!white}
\colorlet{bgColorAlertLIGHTER} {bgColorAlert!10!white}
\colorlet{bgColorAlertLIGHTEST}{bgColorAlert! 5!white}
\colorlet{fgColorExampdark}    {fgColorExamp!80!black}
\colorlet{bgColorExamplight}   {bgColorExamp!30!white}
\colorlet{bgColorExampLight}   {bgColorExamp!20!white}
\colorlet{bgColorExampLIGHT}   {bgColorExamp!15!white}
\colorlet{bgColorExampLIGHTER} {bgColorExamp!10!white}
\colorlet{bgColorExampLIGHTEST}{bgColorExamp! 5!white}
% theme styles
\setbeamercolor*{palette primary}           {fg=fgColorThemeDARK,     bg=bgColorThemelight}
\setbeamercolor*{palette secondary}         {fg=fgColorThemeDark,     bg=bgColorThemeLIGHT}
\setbeamercolor*{palette tertiary}          {bg=fgColorThemedark,     fg=bgColorThemeLIGHTER}
\setbeamercolor*{palette quaternary}        {bg=fgColorTheme,         fg=white}
\setbeamercolor*{sidebar}                   {fg=fgColorTheme,         bg=bgColorThemeLIGHT}
\setbeamercolor*{palette sidebar primary}   {fg=fgColorThemeDARKEST}
\setbeamercolor*{palette sidebar secondary} {fg=white}
\setbeamercolor*{palette sidebar tertiary}  {fg=fgColorThemeDARKER}
\setbeamercolor*{palette sidebar quaternary}{fg=bgColorThemeLIGHTER}
\setbeamercolor*{separation line}           {}
\setbeamercolor*{fine separation line}      {}
% body text
\setbeamercolor {section in toc}            {fg=black,                bg=white}
\setbeamercolor {titlelike}                 {fg=bgColorThemeLIGHT,    bg=fgColorThemedark}
\setbeamercolor {frametitle}                {fg=fgColorThemedark,     bg=bgColorThemeLIGHT}
\setbeamercolor {frametitle right}          {fg=fgColorThemedark,     bg=bgColorThemedark}
\setbeamercolor {structure}                 {fg=fgColorStrucdark}
\setbeamercolor {alerted text}              {fg=fgColorAlertdark}
\setbeamercolor {example text}              {fg=fgColorExampdark}
% blocks
\iftoggle{useColorBlocks}{
  \setbeamercolor {block title}               {bg=bgColorStrucLIGHTER}
  \setbeamercolor {block body}                {bg=bgColorStrucLIGHTEST}
  \setbeamercolor {block title alerted}       {bg=bgColorAlertLIGHTER}
  \setbeamercolor {block body alerted}        {bg=bgColorAlertLIGHTEST}
  \setbeamercolor {block title example}       {bg=bgColorExampLIGHTER}
  \setbeamercolor {block body example}        {bg=bgColorExampLIGHTEST}
  \iftoggle{useColorBlockTitles}{
    \iftoggle{useInverseBlockTitles}{
      % nicely shaded blocks with inverse block titles (original weighting "75" and "10")
      \setbeamercolor{block title}        {use=structure,   fg=white,bg=structure.fg!100!black}
      \setbeamercolor{block title alerted}{use=alerted text,fg=white,bg=alerted text.fg!100!black}
      \setbeamercolor{block title example}{use=example text,fg=white,bg=example text.fg!100!black}
      \setbeamercolor{block body}         {parent=normal text,use=block title,        bg=block title.bg!5!bg}
      \setbeamercolor{block body alerted} {parent=normal text,use=block title alerted,bg=block title alerted.bg!5!bg}
      \setbeamercolor{block body example} {parent=normal text,use=block title example,bg=block title example.bg!5!bg}
    }{
      % shaded blocks with somewhat darker block titles
      \setbeamercolor{block body}         {parent=normal text,use=structure,   bg=structure.fg!5!bg}
      \setbeamercolor{block body alerted} {parent=normal text,use=alerted text,bg=alerted text.fg!5!bg}
      \setbeamercolor{block body example} {parent=normal text,use=example text,bg=example text.fg!5!bg}
      \setbeamercolor{block title}        {use=structure,   bg=structure.fg!10!bg}
      \setbeamercolor{block title alerted}{use=alerted text,bg=alerted text.fg!10!bg}
      \setbeamercolor{block title example}{use=example text,bg=example text.fg!10!bg}
    }
  }{
    % shaded blocks without extra shaded block titles
    \setbeamercolor{block body}         {parent=normal text,use=structure,   bg=structure.fg!5!bg}
    \setbeamercolor{block body alerted} {parent=normal text,use=alerted text,bg=alerted text.fg!5!bg}
    \setbeamercolor{block body example} {parent=normal text,use=example text,bg=example text.fg!5!bg}
    \setbeamercolor{block title}        {use=block body,        bg=block body.bg}
    \setbeamercolor{block title alerted}{use=block body alerted,bg=block body alerted.bg}
    \setbeamercolor{block title example}{use=block body example,bg=block body example.bg}
  }
}{
  % do nothing here
}

% remove shading between backgrounds of block title and block body
\makeatletter
\pgfdeclareverticalshading[lower.bg,upper.bg]{bmb@transition}{200cm}{%
  color(0pt)=(lower.bg); color(2pt)=(lower.bg); color(4pt)=(lower.bg)}
\makeatother




%===== outer theme (based on outer theme "infolines") =====
\makeatletter
% color palette usage
\setbeamercolor*{author     in head/foot}{parent=palette tertiary}
\setbeamercolor*{title      in head/foot}{parent=palette secondary}
\setbeamercolor*{date       in head/foot}{parent=palette primary}
\setbeamercolor*{section    in head/foot}{parent=palette tertiary}
\setbeamercolor*{subsection in head/foot}{parent=palette primary}
% header and footer
\setbeamertemplate{headline}{%
   \leavevmode%
   \hbox{%
     \begin{beamercolorbox}%
     [wd=\paperwidth,ht=2.5ex,dp=0.75ex,left]%
     {author in head/foot}%
       \usebeamerfont{author in head/foot}%
       \hspace*{1.5em}%
       \insertsection%
       \ifdefempty{\insertsubsection}{}{~/~\insertsubsection%
         \ifdefempty{\insertsubsubsection}{}{~/~\insertsubsubsection}%
       }%
     \end{beamercolorbox}%
   }%
   \vskip0pt%
}
\setbeamertemplate{footline}
{
  \leavevmode%
  \hbox{%
    \begin{beamercolorbox}%
    [wd=\paperwidth,ht=2.5ex,dp=0.75ex]%
    {date in head/foot}%
      \usebeamerfont{date in head/foot}%
      \hspace*{1.5em}%
      \insertshortauthor~(\insertshortinstitute)%
      ~~---~~%
      \insertshorttitle%
      \ifdefempty{\insertpart}{}{{:~~}\insertpart}%
      \hfill%
      %\ifdefempty{\insertpart}{}{\insertpartframenumber{}~/~\inserttotalpartframenumber}
      \insertpartframenumber{}~/~\inserttotalpartframenumber
      \hspace*{1.5em}%
    \end{beamercolorbox}%
  }%
  \vskip0pt%
}
% actual usage area
\setbeamersize{text margin left=1em,text margin right=1em}
% navigation symbols
\iftoggle{useNavSymbols}{
  \setbeamertemplate{navigation symbols}{%
    \insertframenavigationsymbol%
    \insertsubsectionnavigationsymbol%
    \insertsectionnavigationsymbol%
    \insertbackfindforwardnavigationsymbol%
  }
}{
  \setbeamertemplate{navigation symbols}{}
}
\makeatother


%===== inner theme (based on inner theme "rounded") =====
\setbeamertemplate{blocks}[rounded][shadow=\usetoggle{useShadows}]
\setbeamertemplate{sections/subsections in toc}[circle]
\setbeamertemplate{title page}[default][colsep=-4bp,rounded=true,shadow=\usetoggle{useShadows}]
\setbeamertemplate{part page}[default][colsep=-4bp,rounded=true,shadow=\usetoggle{useShadows}]

% between blocks
\newlength{\addbegblockskipamount}\setlength{\addbegblockskipamount}{0.0ex}
\newlength{\addendblockskipamount}\setlength{\addendblockskipamount}{0.0ex}
\addtobeamertemplate{block begin}        {\vskip\addbegblockskipamount}{}
\addtobeamertemplate{block end}        {}{\vskip\addendblockskipamount}
\addtobeamertemplate{block alerted begin}{\vskip\addbegblockskipamount}{}
\addtobeamertemplate{block alerted end}{}{\vskip\addendblockskipamount}
\addtobeamertemplate{block example begin}{\vskip\addbegblockskipamount}{}
\addtobeamertemplate{block example end}{}{\vskip\addendblockskipamount}

% formulas inside blocks
\addtobeamertemplate{frame begin}        {}{
  \setlength{\abovedisplayskip}{2ex plus 1ex minus 1.5ex}
  \setlength{\belowdisplayskip}{2ex plus 1ex minus 2.5ex}
}
\addtobeamertemplate{block begin}        {}{
  \setlength{\abovedisplayskip}{2ex plus 1ex minus 1.5ex}
  \setlength{\belowdisplayskip}{2ex plus 1ex minus 1.5ex}
}
\addtobeamertemplate{block alerted begin}{}{
  \setlength{\abovedisplayskip}{2ex plus 1ex minus 1.5ex}
  \setlength{\belowdisplayskip}{2ex plus 1ex minus 1.5ex}
}
\addtobeamertemplate{block example begin}{}{
  \setlength{\abovedisplayskip}{2ex plus 1ex minus 1.5ex}
  \setlength{\belowdisplayskip}{2ex plus 1ex minus 1.5ex}
}

% set default itemize and enumeration
\setbeamertemplate{itemize item}[square]
\setbeamertemplate{itemize subitem}[circle]
\setbeamertemplate{itemize subsubitem}[triangle]
\setbeamertemplate{enumerate items}[auto]
\setbeamertemplate{enumerate mini template}[blank]
\enumAutoDefault{square}{circle}{bracket}
\AfterEndPreamble{%
  \enumStylesDefault{\arabic}{\alph}{\roman}%
}





%===== abbreviations for often used stuff  =====

% begin/end
  \newcommand{\bblk}{\begin{block}}
  \newcommand{\eblk}{\end{block}}
  \newcommand{\bablk}{\begin{alertblock}}
  \newcommand{\eablk}{\end{alertblock}}
  \newcommand{\beblk}{\begin{exampleblock}}
  \newcommand{\eeblk}{\end{exampleblock}}
  \newcommand{\bnest}{\begin{nesting}}
  \newcommand{\enest}{\end{nesting}}
  \newcommand{\bsblk}{\begin{nesting}\begin{block}}
  \newcommand{\esblk}{\end{block}\end{nesting}}
  \newcommand{\basblk}{\begin{nesting}\begin{alertblock}}
  \newcommand{\easblk}{\end{alertblock}\end{nesting}}
  \newcommand{\besblk}{\begin{nesting}\begin{exampleblock}}
  \newcommand{\eesblk}{\end{exampleblock}\end{nesting}}
  \newcommand{\bit}{\begin{itemize}}
  \newcommand{\eit}{\end{itemize}}
  \newcommand{\ben}{\begin{enumerate}}
  \newcommand{\een}{\end{enumerate}}
  \newcommand{\beq}{\begin{equation}}
  \newcommand{\eeq}{\end{equation}}
  \newcommand{\beqn}{\begin{equation*}}
  \newcommand{\eeqn}{\end{equation*}}
  \newcommand{\beqa}{\begin{eqnarray}}
  \newcommand{\eeqa}{\end{eqnarray}}
  \newcommand{\beqan}{\begin{eqnarray*}}
  \newcommand{\eeqan}{\end{eqnarray*}}
  \newcommand{\bmin}{\begin{minipage}}
  \newcommand{\emin}{\end{minipage}}
  
% general formatting
  \newcommand{\loud}[1]{\STRUC{#1}}

% math formatting
  \newcommand{\R}{\mathbb{R}}
\renewcommand{\Pr}[1]{\mathrm{P}\!\left(#1\right)}
  \newcommand{\EV}[1]{\mathrm{E}\!\left\{\,#1\,\right\}}
  \newcommand{\EVX}[2]{\mathrm{E}_{#1}\!\left\{\,#2\,\right\}}
  \newcommand{\func}[2]{\mathrm{#1}\!\left(#2\right)}
  \newcommand{\set}[1]{{\mathcal{#1}}}
\renewcommand{\implies}{\quad\Longrightarrow\quad}
\renewcommand{\equiv}{\quad\Longleftrightarrow\quad}
  \newcommand{\cond}{\,|\,}
  \newcommand{\trans}{^{\mathrm{T}}}
  \newcommand{\bin}{_{\mathrm{b}}}
  \newcommand{\nonl}{\nonumber\\}
\renewcommand{\d}{\mathrm{d}}
  \newcommand{\ve}[1]{{\boldsymbol{#1}}} %{{\mathbf{#1}}}
  \newcommand{\im}{\mathrm{i}}
  \newcommand{\mnorm}[1]{\left\lVert#1\right\rVert}
  \newcommand{\bigmnorm}[1]{\big\lVert#1\big\rVert}
  \newcommand{\Bigmnorm}[1]{\Big\lVert#1\Big\rVert}
  \newcommand{\sha}{\ensuremath{\mathop{\text{\normalfont\fontencoding{T2A}\selectfont ш}}}}
  \newcommand{\Sha}{\ensuremath{\mathop{\text{\normalfont\fontencoding{T2A}\selectfont Ш}}}}
  
% some colors
\newcommand{\cx}[1]{\textcolor{myred}{#1}}
\newcommand{\ca}[1]{\textcolor{myblue}{#1}}
\newcommand{\cb}[1]{\textcolor{myviolet}{#1}}
\newcommand{\cc}[1]{\textcolor{mygreen}{#1}}
\newcommand{\cd}[1]{\textcolor{myorange}{#1}}
\newcommand{\ce}[1]{\textcolor{cyan(process)}{#1}}
\newcommand{\cf}[1]{\textcolor{saddlebrown}{#1}}
\newcommand{\cg}[1]{\textcolor{cadmiumgreen}{#1}}
\newcommand{\ch}[1]{\textcolor{deepcarmine}{#1}}
\newcommand{\ci}[1]{\textcolor{raspberry}{#1}}

\colorlet{myred}{bostonuniversityred}
\colorlet{mygreen}{officegreen!90!black}
\colorlet{myblue}{blue!80!black}
\colorlet{myorange}{orange(colorwheel)!90!black}
\colorlet{myviolet}{vividviolet}
\colorlet{myvvgray}{gray!10}
\colorlet{myvgray}{gray!20}
\colorlet{mylgray}{gray!40}
\colorlet{myngray}{gray}
\colorlet{mydgray}{gray!90!black}

\newcommand{\p}{\;\%}
\newcommand{\clr}[1]{{\color{myred}#1}}
\newcommand{\clb}[1]{{\color{myblue}#1}}
\newcommand{\clg}[1]{{\color{mygreen}#1}}
\newcommand{\clo}[1]{{\color{myorange}#1}}
\newcommand{\clv}[1]{{\color{myviolet}#1}}
\newcommand{\clz}[1]{{\color{black}#1}}
\newcommand{\clm}[1]{{\color{mylgray}#1}}
\newcommand{\cln}[1]{{\color{myngray}#1}}

% citations
\renewcommand{\cite}[1]{{\,\relsize{-1}\color{mygreen}[\,#1\,]}}




%===== define functions for plotting  =====
\makeatletter
\pgfmathdeclarefunction{erf}{1}{%
  \begingroup
    \pgfmathparse{#1 > 0 ? 1 : -1}%
    \edef\sign{\pgfmathresult}%
    \pgfmathparse{abs(#1)}%
    \edef\x{\pgfmathresult}%
    \pgfmathparse{1/(1+0.3275911*\x)}%
    \edef\t{\pgfmathresult}%
    \pgfmathparse{%
      1 - (((((1.061405429*\t -1.453152027)*\t) + 1.421413741)*\t 
      -0.284496736)*\t + 0.254829592)*\t*exp(-(\x*\x))}%
    \edef\y{\pgfmathresult}%
    \pgfmathparse{(\sign)*\y}%
    \pgfmath@smuggleone\pgfmathresult%
  \endgroup
}
\pgfmathdeclarefunction{geopmf}{2}{%
  \pgfmathparse{#1*(1-#1)^#2}%
}
\pgfmathdeclarefunction{geocmf}{2}{%
  \pgfmathparse{1-(1-#1)^(#2+1)}%
}
\pgfmathdeclarefunction{geopmfx}{2}{%
  \pgfmathparse{#1*(1-#1)^(#2-1)}%
}
\pgfmathdeclarefunction{geocmfx}{2}{%
  \pgfmathparse{1-(1-#1)^(#2)}%
}
\pgfmathdeclarefunction{binompmf}{3}{%
  \pgfmathparse{(#1!)/((#3)!*(#1-#3)!)*#2^#3*(1-#2)^(#1-#3)}%
}
\makeatother





%%===== definitions for syntax highlighting =====
%\makeatletter
%\newcommand\ltiny{\@setfontsize\ltiny\@vipt\@viipt}
%\makeatother
%\lstset{%
%  language=C++,
%  backgroundcolor=\color{myvvgray},frame=single,framerule=0pt,
%  basicstyle=\ttfamily\ltiny,
%  keywordstyle=\color{myblue},
%  stringstyle=\color{myred},
%  commentstyle=\color{mygreen},
%  morecomment=[l][\color{myviolet}]{\#}
%}




%%%%%%%%% preliminary slide(s) %%%%%%%%
\newcommand{\preliminaryInfo}{%
\iftoggle{specialHeiko}{%
  \begin{frame}{Organization}
  \vspace{-1ex}
  \begin{tabular}{ll}
  &\\[-2ex]
  Lecture:    & Monday 16:15-17:45\\
              & Room SR 006 / T9\\
  \\[-.5ex]
  Exercise:   & Monday 14:30-16:00\\
              & Room SR 006 / T9\\
  \\[-.5ex]
  Web page:   & \color{blue}\url{http://iphome.hhi.de/schwarz/DC.htm}\\
  \\[.5ex]
  Literature: & \\
  \end{tabular}
  \bit\relsize{-1}
  \item
  Sayood, K. (2018), ``{\bf Introduction to Data Compression},''
  Morgan Kaufmann, Cambridge, MA.
  \item\smallskip
  Cover, T. M. and Thomas, J. A. (2006), ``{\bf Elements of Information Theory},''
  John Wiley \& Sons, New York.
  \item\smallskip
  Gersho, A. and Gray, R. M. (1992), ``{\bf Vector Quantization and Signal Compression},''\\
  Kluwer Academic Publishers, Boston, Dordrecht, London. 
  \item\smallskip
  Jayant, N. S. and Noll, P. (1994), ``{\bf Digital Coding of Waveforms},''
  Prentice-Hall, Englewood Cliffs, NJ, USA.  
  \item\smallskip
  Wiegand, T. and Schwarz, H. (2010), ``{\bf Source Coding: Part I of Fundamentals of Source and Video Coding},''
  Foundations and Trends in Signal Processing, vol.~4, no.~1-2.~~~(\ital{pdf available on course web page})
  \eit
  \end{frame}
}{}


\iftoggle{specialThomas}{%
%  \begin{frame}{Organization}
%  \begin{tabular}{ll}
%  &\\[-1ex]
%  Lecture: & Thursday 10:15-11:45\\
%              & Zoom \\
%  \\[-.5ex]
% % Material:   & \color{blue}\url{http://www.ic.tu-berlin.de/menue/studium_und_lehre/}\\
% Contact:   & Jonathan Pfaff, jonathan.pfaff@hhi.fraunhofer.de \\
%
%  \\[-.5ex]
%  Literatur:  & \\
%  \end{tabular}
%  \bit
%  \item
%  Sayood, K. (2018), ``Introduction to Data Compression,''
%  Morgan Kaufmann, Cambridge, MA.
%  \item
%  Cover, T. M. and Thomas, J. A. (2006), ``Elements of Information Theory,''
%  John Wiley \& Sons, New York.
%  \item
%  Gersho, A. and Gray, R. M. (1992), ``Vector Quantization and Signal Compression,''
%  Kluwer Academic Publishers, Boston, Dordrecht, London. 
%  \item
%  Jayant, N. S. and Noll, P. (1994), ``Digital Coding of Waveforms,''
%  Prentice-Hall, Englewood Cliffs, NJ, USA.  
%  \item
%  Wiegand, T. and Schwarz, H. (2010), ``Source Coding: Part I of Fundamentals of Source and Video Coding,''
%  Foundations and Trends in Signal Processing, vol.~4, no.~1-2.~~~(\ital{pdf available on course web page})
%  \eit
%  \end{frame}
%}{}
  \begin{frame}{Motivation and Goal of the lecture}
  \textbf{Data compression:}
  \begin{itemize}
  \item Data compression is a core technology for modern communication. 
  \item Demand for transmitting large amounts of data increases.
  \item But: Memory and transmission capabilities are limited $\rightarrow$ Data compression algorithms are needed. 
  \end{itemize}
  
  \textbf{Goal of the lecture}: 
  \begin{itemize}
  \item Explain the fundamental problems and some fundamental principles of data compression.
  \item Explain how practical compression algorithm used in billions of devices worldwide work.
  \item Show that there are strong relations between the field of compression and the field of machine learning. 
%  \item Give interested students the opportunity to start doing research work in the field of (video) compression. 
%  \begin{itemize}
%  \item Opportunities to work as a student researcher in compression at Fraunhofer HHI.  
%  \item Interesting topics for Master Thesis projects. 
%  \end{itemize}
   \end{itemize}
  \end{frame}
}{}
%}

  \begin{frame}{Organization}
 \textbf{Lecture:}
  \begin{itemize}
  \item Thursday 10:15-11:45. 
 % \item Due to current situation, lecture will be held via Zoom. 
  \end{itemize}
  
  \textbf{Contact:} 
  \begin{itemize}
  \item Thomas Wiegand: thomas.wiegand@tu-berlin.de, thomas.wiegand@hhi.fraunhofer.de, Chair of Media Technology, TU Berlin, and Head of Fraunhofer HHI 
  \item Jennifer Rasch: j.rasch@protonmail.com
  \item Information about work of research group: 
  \color{blue}\url{https://www.hhi.fraunhofer.de/en/departments/vca/research-groups/video-coding-technologies.html}
   \end{itemize}
   
  \textbf{Important:}
  \begin{itemize}
  \item Introduce yourselves: tell us your name, semester, field of study and motivation
  \item Suggestions: Ask questions during or after the lecture or send informal email to Jennifer Rasch. Via email, informal  appointments for additional communication can be made. Master thesis are possible here :)
   \end{itemize}
  \end{frame}
%}{}
}



%%%%%%%%% end of style file %%%%%%%%%



\DeclareMathOperator{\cwd}{codeword}
\newtheorem{proposition}{Proposition}
\usepackage{forest}
\begin{document}

\newcommand{\SimpleHuffman}{%
  \begin{tikzpicture}
    [
    snode/.style={text=red,anchor=west,font=\relsize{1}},
    pnode/.style={rounded rectangle,draw=black,inner sep=0pt,text=officegreen,font=\bf,minimum height=4ex,minimum width=6ex,scale=0.9},
    lbla/.style={inner sep=2pt,anchor=south,yshift=2pt,myblue,pos=0.5},
    lblb/.style={lbla,anchor=north,yshift=-4pt},
    hl/.style={draw=myblue,line width=2pt},
    ]  
    \pgfmathsetmacro{\dn}{0.8}
    \pgfmathsetmacro{\dl}{2.8}
    \pgfmathsetmacro{\leftx}{-3.3*\dl}
    \pgfmathsetmacro{\midy}{4.25*\dn}
    \pgfmathsetmacro{\boty}{-0.5*\dn}
    \node[anchor=west] (table) at ({\leftx},{\midy}) {\hfitbox{4.2cm}{%
        \begin{tabular}{>{\color{red}}c<{}>{\color{mygreen}}l<{}>{\color{myblue}}l<{}}
          \color{black}$a_k$ & \color{black}$p_k$ & \color{black}codewords\\
          \hline\rule{0ex}{2.5ex}%
          a     & 0.16 & ~~~{111}\\
          b     & 0.04 & ~~~{0001}\\
          c     & 0.04 & ~~~{0000}\\
          d     & 0.16 & ~~~{110}\\
          e     & 0.23 & ~~~{01}\\
          f     & 0.07 & ~~~{1001}\\
          g     & 0.06 & ~~~{1000}\\
          h     & 0.09 & ~~~{001}\\
          i     & 0.15 & ~~~{101}
        \end{tabular}%
      }};
    \node[pnode,fill=gray!25] (root)  at ({-1*\dl},{4.25*\dn}) {1.00};
    \node[pnode,fill=gray!25] (adifg) at ({0*\dl},{6.375*\dn}) {0.60};
    \node[pnode,fill=gray!25] (ad)    at ({1*\dl},{7.5*\dn})   {0.32};
    \node[pnode,fill=gray!25] (a)     at ({2*\dl},{8*\dn})     {0.16};   \draw (a.east) node[snode] {a};
    \node[pnode,fill=gray!25] (d)     at ({2*\dl},{7*\dn})     {0.16};   \draw (d.east) node[snode] {d};
    \node[pnode,fill=gray!25] (ifg)   at ({1*\dl},{5.25*\dn})  {0.28};
    \node[pnode,fill=gray!25] (i)     at ({2*\dl},{6*\dn})     {0.15};   \draw (i.east) node[snode] {i};
    \node[pnode,fill=gray!25] (fg)    at ({2*\dl},{4.5*\dn})   {0.13};
    \node[pnode,fill=gray!25] (f)     at ({3*\dl},{5*\dn})     {0.07};   \draw (f.east) node[snode] {f};
    \node[pnode,fill=gray!25] (g)     at ({3*\dl},{4*\dn})     {0.06};   \draw (g.east) node[snode] {g};
    \node[pnode,fill=gray!25] (ehbc)  at ({0*\dl},{2.125*\dn}) {0.40};
    \node[pnode,fill=gray!25] (e)     at ({1*\dl},{3*\dn})     {0.23};   \draw (e.east) node[snode] {e};
    \node[pnode,fill=gray!25] (hbc)   at ({1*\dl},{1.25*\dn})  {0.17};
    \node[pnode,fill=gray!25] (h)     at ({2*\dl},{2*\dn})     {0.09};   \draw (h.east) node[snode] {h};
    \node[pnode,fill=gray!25] (bc)    at ({2*\dl},{0.5*\dn})   {0.08};
    \node[pnode,fill=gray!25] (b)     at ({3*\dl},{1*\dn})     {0.04};   \draw (b.east) node[snode] {b};
    \node[pnode,fill=gray!25] (c)     at ({3*\dl},{0*\dn})     {0.04};   \draw (c.east) node[snode] {c};
    \draw   (bc)    -- (b);
    \draw   (bc)    -- (c);
    \draw   (fg)    -- (f);
    \draw   (fg)    -- (g);
    \draw   (hbc)   -- (h);
    \draw   (hbc)   -- (bc);
    \draw   (ifg)   -- (i);
    \draw   (ifg)   -- (fg);
    \draw   (ad) -- (a);
    \draw   (ad) -- (d);
    \draw   (ehbc) -- (e);
    \draw   (ehbc) -- (hbc);
    \draw   (adifg) -- (ad);
    \draw   (adifg) -- (ifg);
    \draw   (root) -- (adifg);
    \draw   (root) -- (ehbc);
    \node [anchor=north] at (root.south) {root};
    \draw (root)--(adifg)                    node [lbla] {1};
    \draw         (adifg)--(ad)              node [lbla] {1};
    \draw                  (ad)--(a)         node [lbla] {1};
    \draw                  (ad)--(d)         node [lblb] {0};
    \draw         (adifg)--(ifg)             node [lblb] {0};
    \draw                  (ifg)--(i)        node [lbla] {1};
    \draw                  (ifg)--(fg)       node [lblb] {0};
    \draw                         (fg)--(f)  node [lbla] {1};
    \draw                         (fg)--(g)  node [lblb] {0};
    \draw (root)--(ehbc)                     node [lblb] {0};
    \draw         (ehbc)--(e)                node [lbla] {1};
    \draw         (ehbc)--(hbc)              node [lblb] {0};
    \draw                 (hbc)--(h)         node [lbla] {1};
    \draw                 (hbc)--(bc)        node [lblb] {0};
    \draw                        (bc)--(b)   node [lbla] {1};
    \draw                        (bc)--(c)   node [lblb] {0};
  \end{tikzpicture}
}

\startnewpart[\SimpleHuffman]{Lossless Coding II}





\section{Huffman Coding}


\begin{frame}
 \vspace{8.0ex}
\begin{center}
\begin{beamercolorbox}[sep=12pt,center]{part title}
\usebeamerfont{section title}\insertsection\par
\end{beamercolorbox}
\end{center}
\end{frame}

\subsection{Introduction}

\begin{frame}{Setup and problem statement}
\loud{Lossless coding task: } 
\bit
\item Given a \loud{finite alphabet} $\mathcal{A}$ of symbols.
\item Each symbol $a\in\mathcal{A}$ has a  \loud{probability mass} $p_a$.
\item \loud{Goal: } Design \loud{uniquely decodable codes} $\gamma$ for $\mathcal{A}$ of \loud{minimal average codeword-length} $\overline{\ell}(\gamma)$. 
\eit

\loud{Entropy gives bounds for minimal average codeword length:}
\bit
\item One always has $H(p)\leq \overline{\ell}(\gamma)$, where $H(p)$ is the \loud{entropy} of the distribution $p$. 
\item There exists a uniquely decodable code $\gamma$ with $\overline{\ell}(\gamma)\leq H(p)+1$. 
\eit


\loud{Open question:}
\bit
\item Can we \loud{explicitly construct} a uniquely decodable code of minimal average codeword length? 
\eit

\ALERT{\iarrow  Huffman Algorithm: Construct prefix code with minimal average codeword length. }
\end{frame}

\subsection{Huffman Algorithm}

\begin{frame}{The Huffman Algorithm}
\begin{minipage}[t]{0.8\linewidth}
  \STRUC{Idea for Construction of Binary Code Tree}
  \bit\small%
  \item Consider optimal prefix codes for which the two least likely symbols correspond to codewords of maximum length that differ only in the final bit
  \item[\iarrow]<3-> Choose the two least likely symbols and create a parent node 
  \item[\iarrow]<6-> Repeat the procedure with the reduced alphabet
 \eit
\end{minipage}%

  \vspace{-2.0ex}
  \begin{center}
    \begin{minipage}{0.8\linewidth}
      \bablk<10->{Huffman Algorithm (via construction of binary code tree)}
      \ben\small
    \item\vspace{.3ex} Select the two letters $a$ and $b$ with the smallest probabilities $p_a$ and $p_b$
    \item\vspace{.3ex} Create a parent node for the two letters $a$ and $b$ in the binary code tree
    \item\vspace{.3ex} Replace the letters $a$ and $b$ with a new letter with probability $p_a+p_b$
    \item\vspace{.3ex} If the resulting new alphabet contains more than a single letter,\\
      repeat all previous steps with this alphabet
    \item\vspace{.3ex} Convert the obtained binary code tree into a prefix code
      \een
      \eablk
    \end{minipage}
  \end{center}
  \vspace{-10ex}
\end{frame}


\subsection{Huffman Algorithm example}



\newcommand{\HuffmanExample}{%
  \begin{tikzpicture}
    [
    snode/.style={text=red,anchor=west,font=\relsize{1}},
    pnode/.style={rounded rectangle,draw=black,inner sep=0pt,text=officegreen,font=\bf,minimum height=4ex,minimum width=6ex,scale=0.9},
    lbla/.style={inner sep=2pt,anchor=south,yshift=2pt,blue,pos=0.5},
    lblb/.style={lbla,anchor=north,yshift=-4pt},
    hl/.style={draw=blue,line width=2pt},
    ]  

    \pgfmathsetmacro{\dn}{0.8}
    \pgfmathsetmacro{\dl}{2.0}
    \pgfmathsetmacro{\leftx}{-4*\dl}
    \pgfmathsetmacro{\topy}{8.5*\dn}
    \pgfmathsetmacro{\boty}{-0.5*\dn}
    
    \begin{scope}[visible on=<1|handout:0>]
      \node[anchor=north west,align=left] (step0) at ({\leftx},{\topy}) 
      {\underline{given:}\\[.5ex]alphabet $\set{A}$ with pmf $\{p_k\}$};
    \end{scope}

    \begin{scope}[visible on=<1->]
      \node[anchor=south west] (table) at ({\leftx},{\boty}) {\hfitbox{3.5cm}{%
          \begin{tabular}{>{\color{red}}c<{}>{\color{officegreen}}c<{}>{\color{blue}}l<{}}
            \color{black}$a_k$ & \color{black}$p_k$ & \color{black}codewords\\
            \hline\rule{0ex}{2.5ex}%
            a     & 0.16 & \action<27->{~~~~111}\\
            b     & 0.04 & \action<27->{~~~~0001}\\
            c     & 0.04 & \action<27->{~~~~0000}\\
            d     & 0.16 & \action<27->{~~~~110}\\
            e     & 0.23 & \action<27->{~~~~01}\\
            f     & 0.07 & \action<27->{~~~~1001}\\
            g     & 0.06 & \action<26->{~~~~1000}\\
            h     & 0.09 & \action<27->{~~~~001}\\
            i     & 0.15 & \action<27->{~~~~101}
          \end{tabular}%
        }};
    \end{scope}

    \begin{scope}[visible on=<2|handout:0>]
      \node[anchor=north west,align=left] at ({\leftx},{\topy}) 
      {\underline{first step:}\\[.5ex]assign symbols and probabilities to terminal nodes};
    \end{scope}

    \begin{scope}[visible on=<3-4|handout:0>]
      \node[pnode]                            (a) at ({0*\dl},{8*\dn}) {0.16};   \draw (a.east) node[snode] {a};
      \node[pnode,alt={<4->{fill=gray!25}{}}] (b) at ({0*\dl},{7*\dn}) {0.04};   \draw (b.east) node[snode] {b};
      \node[pnode,alt={<4->{fill=gray!25}{}}] (c) at ({0*\dl},{6*\dn}) {0.04};   \draw (c.east) node[snode] {c};
      \node[pnode]                            (d) at ({0*\dl},{5*\dn}) {0.16};   \draw (d.east) node[snode] {d};
      \node[pnode]                            (e) at ({0*\dl},{4*\dn}) {0.23};   \draw (e.east) node[snode] {e};
      \node[pnode]                            (f) at ({0*\dl},{3*\dn}) {0.07};   \draw (f.east) node[snode] {f};
      \node[pnode]                            (g) at ({0*\dl},{2*\dn}) {0.06};   \draw (g.east) node[snode] {g};
      \node[pnode]                            (h) at ({0*\dl},{1*\dn}) {0.09};   \draw (h.east) node[snode] {h};
      \node[pnode]                            (i) at ({0*\dl},{0*\dn}) {0.15};   \draw (i.east) node[snode] {i};
      \begin{scope}[visible on=<4->]
        \node[pnode] (bc) at ({-1*\dl},{6.5*\dn}) {0.08};
        \draw   (bc) -- (b);
        \draw   (bc) -- (c);
      \end{scope}
    \end{scope}

    \begin{scope}[visible on=<5-6|handout:0>]
      \node[pnode]                             (a)    at ({0*\dl},{8*\dn})     {0.16};   \draw (a.east) node[snode] {a};
      \node[pnode]                             (bc)   at ({0*\dl},{6.5*\dn})   {0.08};
      \node[pnode,fill=gray!25]                (b)    at ({1*\dl},{7*\dn})     {0.04};   \draw (b.east) node[snode] {b};
      \node[pnode,fill=gray!25]                (c)    at ({1*\dl},{6*\dn})     {0.04};   \draw (c.east) node[snode] {c};
      \node[pnode]                             (d)    at ({0*\dl},{5*\dn})     {0.16};   \draw (d.east) node[snode] {d};
      \node[pnode]                             (e)    at ({0*\dl},{4*\dn})     {0.23};   \draw (e.east) node[snode] {e};
      \node[pnode,alt={<6->{fill=gray!25}{}}]  (f)    at ({0*\dl},{3*\dn})     {0.07};   \draw (f.east) node[snode] {f};
      \node[pnode,alt={<6->{fill=gray!25}{}}]  (g)    at ({0*\dl},{2*\dn})     {0.06};   \draw (g.east) node[snode] {g};
      \node[pnode]                             (h)    at ({0*\dl},{1*\dn})     {0.09};   \draw (h.east) node[snode] {h};
      \node[pnode]                             (i)    at ({0*\dl},{0*\dn})     {0.15};   \draw (i.east) node[snode] {i};
      \draw   (bc)    -- (b);
      \draw   (bc)    -- (c);
      \begin{scope}[visible on=<6->]
        \node[pnode] (fg) at ({-1*\dl},{2.5*\dn}) {0.13};
        \draw   (fg) -- (f);
        \draw   (fg) -- (g);
      \end{scope}
    \end{scope}
    
    \begin{scope}[visible on=<7-8|handout:0>]
      \node[pnode]                             (a)    at ({0*\dl},{8*\dn})     {0.16};   \draw (a.east) node[snode] {a};
      \node[pnode]                             (bc)   at ({0*\dl},{6.5*\dn})   {0.08};
      \node[pnode,fill=gray!25]                (b)    at ({1*\dl},{7*\dn})     {0.04};   \draw (b.east) node[snode] {b};
      \node[pnode,fill=gray!25]                (c)    at ({1*\dl},{6*\dn})     {0.04};   \draw (c.east) node[snode] {c};
      \node[pnode]                             (d)    at ({0*\dl},{5*\dn})     {0.16};   \draw (d.east) node[snode] {d};
      \node[pnode]                             (e)    at ({0*\dl},{4*\dn})     {0.23};   \draw (e.east) node[snode] {e};
      \node[pnode]                             (fg)   at ({0*\dl},{2.5*\dn})   {0.13};
      \node[pnode,fill=gray!25]                (f)    at ({1*\dl},{3*\dn})     {0.07};   \draw (f.east) node[snode] {f};
      \node[pnode,fill=gray!25]                (g)    at ({1*\dl},{2*\dn})     {0.06};   \draw (g.east) node[snode] {g};
      \node[pnode]                             (h)    at ({0*\dl},{1*\dn})     {0.09};   \draw (h.east) node[snode] {h};
      \node[pnode]                             (i)    at ({0*\dl},{0*\dn})     {0.15};   \draw (i.east) node[snode] {i};
      \draw   (bc)    -- (b);
      \draw   (bc)    -- (c);
      \draw   (fg)    -- (f);
      \draw   (fg)    -- (g);
    \end{scope}
    
    \begin{scope}[visible on=<8|handout:0>]
      \node[anchor=north west,align=left] at ({\leftx},{\topy}) 
      {\underline{next step:}\\[.5ex]re-order for better readability};
    \end{scope}

    \begin{scope}[visible on=<9-10|handout:0>]
      \node[pnode]                             (e)    at ({0*\dl},{8*\dn})     {0.23};   \draw (e.east) node[snode] {e};
      \node[pnode]                             (a)    at ({0*\dl},{7*\dn})     {0.16};   \draw (a.east) node[snode] {a};
      \node[pnode]                             (d)    at ({0*\dl},{6*\dn})     {0.16};   \draw (d.east) node[snode] {d};
      \node[pnode]                             (i)    at ({0*\dl},{5*\dn})     {0.15};   \draw (i.east) node[snode] {i};
      \node[pnode]                             (fg)   at ({0*\dl},{3.5*\dn})   {0.13};
      \node[pnode,fill=gray!25]                (f)    at ({1*\dl},{4*\dn})     {0.07};   \draw (f.east) node[snode] {f};
      \node[pnode,fill=gray!25]                (g)    at ({1*\dl},{3*\dn})     {0.06};   \draw (g.east) node[snode] {g};
      \node[pnode,alt={<10->{fill=gray!25}{}}] (h)    at ({0*\dl},{2*\dn})     {0.09};   \draw (h.east) node[snode] {h};
      \node[pnode,alt={<10->{fill=gray!25}{}}] (bc)   at ({0*\dl},{0.5*\dn})   {0.08};
      \node[pnode,fill=gray!25]                (b)    at ({1*\dl},{1*\dn})     {0.04};   \draw (b.east) node[snode] {b};
      \node[pnode,fill=gray!25]                (c)    at ({1*\dl},{0*\dn})     {0.04};   \draw (c.east) node[snode] {c};
      \draw   (bc)    -- (b);
      \draw   (bc)    -- (c);
      \draw   (fg)    -- (f);
      \draw   (fg)    -- (g);
      \begin{scope}[visible on=<10->]
        \node[pnode] (hbc) at ({-1*\dl},{1.25*\dn}) {0.17};
        \draw   (hbc) -- (h);
        \draw   (hbc) -- (bc);
      \end{scope}
    \end{scope}

    \begin{scope}[visible on=<11-12|handout:0>]
      \node[pnode]                             (e)    at ({0*\dl},{8*\dn})     {0.23};   \draw (e.east) node[snode] {e};
      \node[pnode]                             (a)    at ({0*\dl},{7*\dn})     {0.16};   \draw (a.east) node[snode] {a};
      \node[pnode]                             (d)    at ({0*\dl},{6*\dn})     {0.16};   \draw (d.east) node[snode] {d};
      \node[pnode,alt={<12->{fill=gray!25}{}}] (i)    at ({0*\dl},{5*\dn})     {0.15};   \draw (i.east) node[snode] {i};
      \node[pnode,alt={<12->{fill=gray!25}{}}] (fg)   at ({0*\dl},{3.5*\dn})   {0.13};
      \node[pnode,fill=gray!25]                (f)    at ({1*\dl},{4*\dn})     {0.07};   \draw (f.east) node[snode] {f};
      \node[pnode,fill=gray!25]                (g)    at ({1*\dl},{3*\dn})     {0.06};   \draw (g.east) node[snode] {g};
      \node[pnode]                             (hbc)  at ({0*\dl},{1.25*\dn})  {0.17};
      \node[pnode,fill=gray!25]                (h)    at ({1*\dl},{2*\dn})     {0.09};   \draw (h.east) node[snode] {h};
      \node[pnode,fill=gray!25]                (bc)   at ({1*\dl},{0.5*\dn})   {0.08};
      \node[pnode,fill=gray!25]                (b)    at ({2*\dl},{1*\dn})     {0.04};   \draw (b.east) node[snode] {b};
      \node[pnode,fill=gray!25]                (c)    at ({2*\dl},{0*\dn})     {0.04};   \draw (c.east) node[snode] {c};
      \draw   (bc)    -- (b);
      \draw   (bc)    -- (c);
      \draw   (fg)    -- (f);
      \draw   (fg)    -- (g);
      \draw   (hbc)   -- (h);
      \draw   (hbc)   -- (bc);
      \begin{scope}[visible on=<12->]
        \node[pnode] (ifg) at ({-1*\dl},{4.25*\dn})  {0.28};
        \draw   (ifg) -- (i);
        \draw   (ifg) -- (fg);
      \end{scope}
    \end{scope}
    
    \begin{scope}[visible on=<13-14|handout:0>]
      \node[pnode]                             (e)    at ({0*\dl},{8*\dn})     {0.23};   \draw (e.east) node[snode] {e};
      \node[pnode,alt={<14->{fill=gray!25}{}}] (a)    at ({0*\dl},{7*\dn})     {0.16};   \draw (a.east) node[snode] {a};
      \node[pnode,alt={<14->{fill=gray!25}{}}] (d)    at ({0*\dl},{6*\dn})     {0.16};   \draw (d.east) node[snode] {d};
      \node[pnode]                             (ifg)  at ({0*\dl},{4.25*\dn})  {0.28};
      \node[pnode,fill=gray!25]                (i)    at ({1*\dl},{5*\dn})     {0.15};   \draw (i.east) node[snode] {i};
      \node[pnode,fill=gray!25]                (fg)   at ({1*\dl},{3.5*\dn})   {0.13};
      \node[pnode,fill=gray!25]                (f)    at ({2*\dl},{4*\dn})     {0.07};   \draw (f.east) node[snode] {f};
      \node[pnode,fill=gray!25]                (g)    at ({2*\dl},{3*\dn})     {0.06};   \draw (g.east) node[snode] {g};
      \node[pnode]                             (hbc)  at ({0*\dl},{1.25*\dn})  {0.17};
      \node[pnode,fill=gray!25]                (h)    at ({1*\dl},{2*\dn})     {0.09};   \draw (h.east) node[snode] {h};
      \node[pnode,fill=gray!25]                (bc)   at ({1*\dl},{0.5*\dn})   {0.08};
      \node[pnode,fill=gray!25]                (b)    at ({2*\dl},{1*\dn})     {0.04};   \draw (b.east) node[snode] {b};
      \node[pnode,fill=gray!25]                (c)    at ({2*\dl},{0*\dn})     {0.04};   \draw (c.east) node[snode] {c};
      \draw   (bc)    -- (b);
      \draw   (bc)    -- (c);
      \draw   (fg)    -- (f);
      \draw   (fg)    -- (g);
      \draw   (hbc)   -- (h);
      \draw   (hbc)   -- (bc);
      \draw   (ifg)   -- (i);
      \draw   (ifg)   -- (fg);
      \begin{scope}[visible on=<14->]
        \node[pnode] (ad) at ({-1*\dl},{6.5*\dn})  {0.32};
        \draw   (ad) -- (a);
        \draw   (ad) -- (d);
      \end{scope}
    \end{scope}

    \begin{scope}[visible on=<15-16|handout:0>]
      \node[pnode]                             (e)    at ({0*\dl},{8*\dn})     {0.23};   \draw (e.east) node[snode] {e};
      \node[pnode]                             (ad)   at ({0*\dl},{6.5*\dn})   {0.32};
      \node[pnode,fill=gray!25]                (a)    at ({1*\dl},{7*\dn})     {0.16};   \draw (a.east) node[snode] {a};
      \node[pnode,fill=gray!25]                (d)    at ({1*\dl},{6*\dn})     {0.16};   \draw (d.east) node[snode] {d};
      \node[pnode]                             (ifg)  at ({0*\dl},{4.25*\dn})  {0.28};
      \node[pnode,fill=gray!25]                (i)    at ({1*\dl},{5*\dn})     {0.15};   \draw (i.east) node[snode] {i};
      \node[pnode,fill=gray!25]                (fg)   at ({1*\dl},{3.5*\dn})   {0.13};
      \node[pnode,fill=gray!25]                (f)    at ({2*\dl},{4*\dn})     {0.07};   \draw (f.east) node[snode] {f};
      \node[pnode,fill=gray!25]                (g)    at ({2*\dl},{3*\dn})     {0.06};   \draw (g.east) node[snode] {g};
      \node[pnode]                             (hbc)  at ({0*\dl},{1.25*\dn})  {0.17};
      \node[pnode,fill=gray!25]                (h)    at ({1*\dl},{2*\dn})     {0.09};   \draw (h.east) node[snode] {h};
      \node[pnode,fill=gray!25]                (bc)   at ({1*\dl},{0.5*\dn})   {0.08};
      \node[pnode,fill=gray!25]                (b)    at ({2*\dl},{1*\dn})     {0.04};   \draw (b.east) node[snode] {b};
      \node[pnode,fill=gray!25]                (c)    at ({2*\dl},{0*\dn})     {0.04};   \draw (c.east) node[snode] {c};
      \draw   (bc)    -- (b);
      \draw   (bc)    -- (c);
      \draw   (fg)    -- (f);
      \draw   (fg)    -- (g);
      \draw   (hbc)   -- (h);
      \draw   (hbc)   -- (bc);
      \draw   (ifg)   -- (i);
      \draw   (ifg)   -- (fg);
      \draw   (ad) -- (a);
      \draw   (ad) -- (d);
    \end{scope}

    \begin{scope}[visible on=<16|handout:0>]
      \node[anchor=north west,align=left] at ({\leftx},{\topy}) 
      {\underline{next step:}\\[.5ex]re-order for better readability};
    \end{scope}

    \begin{scope}[visible on=<17-18|handout:0>]
      \node[pnode]                             (ad)   at ({0*\dl},{7.5*\dn})   {0.32};
      \node[pnode,fill=gray!25]                (a)    at ({1*\dl},{8*\dn})     {0.16};   \draw (a.east) node[snode] {a};
      \node[pnode,fill=gray!25]                (d)    at ({1*\dl},{7*\dn})     {0.16};   \draw (d.east) node[snode] {d};
      \node[pnode]                             (ifg)  at ({0*\dl},{5.25*\dn})  {0.28};
      \node[pnode,fill=gray!25]                (i)    at ({1*\dl},{6*\dn})     {0.15};   \draw (i.east) node[snode] {i};
      \node[pnode,fill=gray!25]                (fg)   at ({1*\dl},{4.5*\dn})   {0.13};
      \node[pnode,fill=gray!25]                (f)    at ({2*\dl},{5*\dn})     {0.07};    \draw (f.east) node[snode] {f};
      \node[pnode,fill=gray!25]                (g)    at ({2*\dl},{4*\dn})     {0.06};    \draw (g.east) node[snode] {g};
      \node[pnode,alt={<18->{fill=gray!25}{}}] (e)    at ({0*\dl},{3*\dn})     {0.23};   \draw (e.east) node[snode] {e};
      \node[pnode,alt={<18->{fill=gray!25}{}}] (hbc)  at ({0*\dl},{1.25*\dn})  {0.17};
      \node[pnode,fill=gray!25]                (h)    at ({1*\dl},{2*\dn})     {0.09};    \draw (h.east) node[snode] {h};
      \node[pnode,fill=gray!25]                (bc)   at ({1*\dl},{0.5*\dn})   {0.08};
      \node[pnode,fill=gray!25]                (b)    at ({2*\dl},{1*\dn})     {0.04};    \draw (b.east) node[snode] {b};
      \node[pnode,fill=gray!25]                (c)    at ({2*\dl},{0*\dn})     {0.04};    \draw (c.east) node[snode] {c};
      \draw   (bc)    -- (b);
      \draw   (bc)    -- (c);
      \draw   (fg)    -- (f);
      \draw   (fg)    -- (g);
      \draw   (hbc)   -- (h);
      \draw   (hbc)   -- (bc);
      \draw   (ifg)   -- (i);
      \draw   (ifg)   -- (fg);
      \draw   (ad) -- (a);
      \draw   (ad) -- (d);
      \begin{scope}[visible on=<18->]
        \node[pnode] (ehbc) at ({-1*\dl},{2.125*\dn})  {0.40};
        \draw   (ehbc) -- (e);
        \draw   (ehbc) -- (hbc);
      \end{scope}
    \end{scope}

    \begin{scope}[visible on=<19-20|handout:0>]
      \node[pnode,alt={<20->{fill=gray!25}{}}] (ad)   at ({0*\dl},{7.5*\dn})   {0.32};
      \node[pnode,fill=gray!25]                (a)    at ({1*\dl},{8*\dn})     {0.16};   \draw (a.east) node[snode] {a};
      \node[pnode,fill=gray!25]                (d)    at ({1*\dl},{7*\dn})     {0.16};   \draw (d.east) node[snode] {d};
      \node[pnode,alt={<20->{fill=gray!25}{}}] (ifg)  at ({0*\dl},{5.25*\dn})  {0.28};
      \node[pnode,fill=gray!25]                (i)    at ({1*\dl},{6*\dn})     {0.15};   \draw (i.east) node[snode] {i};
      \node[pnode,fill=gray!25]                (fg)   at ({1*\dl},{4.5*\dn})   {0.13};
      \node[pnode,fill=gray!25]                (f)    at ({2*\dl},{5*\dn})     {0.07};   \draw (f.east) node[snode] {f};
      \node[pnode,fill=gray!25]                (g)    at ({2*\dl},{4*\dn})     {0.06};   \draw (g.east) node[snode] {g};
      \node[pnode]                             (ehbc) at ({0*\dl},{2.125*\dn}) {0.40};
      \node[pnode,fill=gray!25]                (e)    at ({1*\dl},{3*\dn})     {0.23};   \draw (e.east) node[snode] {e};
      \node[pnode,fill=gray!25]                (hbc)  at ({1*\dl},{1.25*\dn})  {0.17};
      \node[pnode,fill=gray!25]                (h)    at ({2*\dl},{2*\dn})     {0.09};   \draw (h.east) node[snode] {h};
      \node[pnode,fill=gray!25]                (bc)   at ({2*\dl},{0.5*\dn})   {0.08};
      \node[pnode,fill=gray!25]                (b)    at ({3*\dl},{1*\dn})     {0.04};   \draw (b.east) node[snode] {b};
      \node[pnode,fill=gray!25]                (c)    at ({3*\dl},{0*\dn})     {0.04};   \draw (c.east) node[snode] {c};
      \draw   (bc)    -- (b);
      \draw   (bc)    -- (c);
      \draw   (fg)    -- (f);
      \draw   (fg)    -- (g);
      \draw   (hbc)   -- (h);
      \draw   (hbc)   -- (bc);
      \draw   (ifg)   -- (i);
      \draw   (ifg)   -- (fg);
      \draw   (ad) -- (a);
      \draw   (ad) -- (d);
      \draw   (ehbc) -- (e);
      \draw   (ehbc) -- (hbc);
      \begin{scope}[visible on=<20->]
        \node[pnode] (adifg) at ({-1*\dl},{6.375*\dn})  {0.60};
        \draw   (adifg) -- (ad);
        \draw   (adifg) -- (ifg);
      \end{scope}
    \end{scope}

    \begin{scope}[visible on=<21->]
      \node[pnode,alt={<22->{fill=gray!25}{}}] (adifg) at ({0*\dl},{6.375*\dn}) {0.60};
      \node[pnode,fill=gray!25]                (ad)    at ({1*\dl},{7.5*\dn})   {0.32};
      \node[pnode,fill=gray!25]                (a)     at ({2*\dl},{8*\dn})     {0.16};   \draw (a.east) node[snode] {a};
      \node[pnode,fill=gray!25]                (d)     at ({2*\dl},{7*\dn})     {0.16};   \draw (d.east) node[snode] {d};
      \node[pnode,fill=gray!25]                (ifg)   at ({1*\dl},{5.25*\dn})  {0.28};
      \node[pnode,fill=gray!25]                (i)     at ({2*\dl},{6*\dn})     {0.15};   \draw (i.east) node[snode] {i};
      \node[pnode,fill=gray!25]                (fg)    at ({2*\dl},{4.5*\dn})   {0.13};
      \node[pnode,fill=gray!25]                (f)     at ({3*\dl},{5*\dn})     {0.07};   \draw (f.east) node[snode] {f};
      \node[pnode,fill=gray!25]                (g)     at ({3*\dl},{4*\dn})     {0.06};   \draw (g.east) node[snode] {g};
      \node[pnode,alt={<22->{fill=gray!25}{}}] (ehbc)  at ({0*\dl},{2.125*\dn}) {0.40};
      \node[pnode,fill=gray!25]                (e)     at ({1*\dl},{3*\dn})     {0.23};   \draw (e.east) node[snode] {e};
      \node[pnode,fill=gray!25]                (hbc)   at ({1*\dl},{1.25*\dn})  {0.17};
      \node[pnode,fill=gray!25]                (h)     at ({2*\dl},{2*\dn})     {0.09};   \draw (h.east) node[snode] {h};
      \node[pnode,fill=gray!25]                (bc)    at ({2*\dl},{0.5*\dn})   {0.08};
      \node[pnode,fill=gray!25]                (b)     at ({3*\dl},{1*\dn})     {0.04};   \draw (b.east) node[snode] {b};
      \node[pnode,fill=gray!25]                (c)     at ({3*\dl},{0*\dn})     {0.04};   \draw (c.east) node[snode] {c};
      \draw   (bc)    -- (b);
      \draw   (bc)    -- (c);
      \draw   (fg)    -- (f);
      \draw   (fg)    -- (g);
      \draw   (hbc)   -- (h);
      \draw   (hbc)   -- (bc);
      \draw   (ifg)   -- (i);
      \draw   (ifg)   -- (fg);
      \draw   (ad) -- (a);
      \draw   (ad) -- (d);
      \draw   (ehbc) -- (e);
      \draw   (ehbc) -- (hbc);
      \draw   (adifg) -- (ad);
      \draw   (adifg) -- (ifg);
      \begin{scope}[visible on=<22->]
        \node[pnode,fill=gray!25] (root) at ({-1*\dl},{4.25*\dn})  {1.00};
        \draw   (root) -- (adifg);
        \draw   (root) -- (ehbc);
        \node [anchor=north] at (root.south) {root};
      \end{scope}
    \end{scope}

    \begin{scope}[visible on=<23-24|handout:0>]
      \node[anchor=north west,align=left] at ({\leftx},{\topy}) 
      {\underline{next step:}\\[.5ex]label branches with {\color{blue}0} and {\color{blue}1}};
    \end{scope}

    \begin{scope}[visible on=<24->]
      \draw (root)--(adifg)                    node [lbla] {1};
      \draw         (adifg)--(ad)              node [lbla] {1};
      \draw                  (ad)--(a)         node [lbla] {1};
      \draw                  (ad)--(d)         node [lblb] {0};
      \draw         (adifg)--(ifg)             node [lblb] {0};
      \draw                  (ifg)--(i)        node [lbla] {1};
      \draw                  (ifg)--(fg)       node [lblb] {0};
      \draw                         (fg)--(f)  node [lbla] {1};
      \draw                         (fg)--(g)  node [lblb] {0};
      \draw (root)--(ehbc)                     node [lblb] {0};
      \draw         (ehbc)--(e)                node [lbla] {1};
      \draw         (ehbc)--(hbc)              node [lblb] {0};
      \draw                 (hbc)--(h)         node [lbla] {1};
      \draw                 (hbc)--(bc)        node [lblb] {0};
      \draw                        (bc)--(b)   node [lbla] {1};
      \draw                        (bc)--(c)   node [lblb] {0};
    \end{scope}

    \begin{scope}[visible on=<25-26|handout:0>]
      \node[anchor=north west,align=left] at ({\leftx},{\topy}) 
      {\underline{next step:}\\[.5ex]assign codewords\\(follow branches from root\\to terminal nodes)};
    \end{scope}

    \begin{scope}[visible on=<26|handout:0>]
      \draw[hl] (root)--(adifg);
      \draw[hl]         (adifg)--(ifg);
      \draw[hl]                  (ifg)--(fg);
      \draw[hl]                         (fg)--(g);
    \end{scope}

    \begin{scope}[visible on=<28>]
      \node[anchor=north west,align=left] at ({\leftx},{\topy}) 
      {$\begin{array}{r@{\;\;}c@{\;\;}l}
          \bar{\ell} &=& 2.98\\[.5ex]
          H(p)&\approx&2.9405\\[.5ex]
          \varrho&\approx&0.0395~\,(\,1.34\,\%\,)
        \end{array}$
      };
    \end{scope}
  \end{tikzpicture}
}





%\newcommand{\HuffmanExampleOld}{%
%  \begin{tikzpicture}
%    [
%    snode/.style={text=red,anchor=west,font=\relsize{1}},
%    pnode/.style={circle,draw=black,inner sep=0pt,text=officegreen,font=\bf,minimum size=4ex},
%    lbla/.style={inner sep=2pt,anchor=south,yshift=2pt,blue,pos=0.5},
%    lblb/.style={lbla,anchor=north,yshift=-4pt},
%    hl/.style={draw=blue,line width=2pt},
%    ]  
%
%    \pgfmathsetmacro{\dn}{0.8}
%    \pgfmathsetmacro{\dl}{2.0}
%    \pgfmathsetmacro{\leftx}{-4*\dl}
%    \pgfmathsetmacro{\topy}{8.5*\dn}
%    \pgfmathsetmacro{\boty}{-0.5*\dn}
%    
%    \begin{scope}[visible on=<1|handout:0>]
%      \node[anchor=north west,align=left] (step0) at ({\leftx},{\topy}) 
%      {\underline{given:}\\[.5ex]alphabet $\set{A}$ with pmf $\{p_k\}$};
%    \end{scope}
%
%    \begin{scope}[visible on=<1->]
%      \node[anchor=south west] (table) at ({\leftx},{\boty}) {\hfitbox{3cm}{%
%          \begin{tabular}{>{\color{red}}c<{}>{\color{officegreen}}l<{}>{\color{blue}}l<{}}
%            \color{black}$a_k$ & \color{black}$p_k$ & \color{black}$\ve{b}_k$\\
%            \hline\rule{0ex}{2.5ex}%
%            a     & 0.16 & \action<27->{111}\\
%            b     & 0.04 & \action<27->{0001}\\
%            c     & 0.04 & \action<27->{0000}\\
%            d     & 0.16 & \action<27->{110}\\
%            e     & 0.23 & \action<27->{01}\\
%            f     & 0.07 & \action<27->{1001}\\
%            g     & 0.06 & \action<26->{1000}\\
%            h     & 0.09 & \action<27->{001}\\
%            i     & 0.15 & \action<27->{101}
%          \end{tabular}%
%        }};
%    \end{scope}
%
%    \begin{scope}[visible on=<2|handout:0>]
%      \node[anchor=north west,align=left] at ({\leftx},{\topy}) 
%      {\underline{first step:}\\[.5ex]assign symbols and probabilities to terminal nodes\\
%        (use {\color{officegreen}$100\,p_k$} for better readability)};
%    \end{scope}
%
%    \begin{scope}[visible on=<3-4|handout:0>]
%      \node[pnode]                            (a) at ({0*\dl},{8*\dn}) {16};   \draw (a.east) node[snode] {a};
%      \node[pnode,alt={<4->{fill=gray!25}{}}] (b) at ({0*\dl},{7*\dn}) {4};    \draw (b.east) node[snode] {b};
%      \node[pnode,alt={<4->{fill=gray!25}{}}] (c) at ({0*\dl},{6*\dn}) {4};    \draw (c.east) node[snode] {c};
%      \node[pnode]                            (d) at ({0*\dl},{5*\dn}) {16};   \draw (d.east) node[snode] {d};
%      \node[pnode]                            (e) at ({0*\dl},{4*\dn}) {23};   \draw (e.east) node[snode] {e};
%      \node[pnode]                            (f) at ({0*\dl},{3*\dn}) {7};    \draw (f.east) node[snode] {f};
%      \node[pnode]                            (g) at ({0*\dl},{2*\dn}) {6};    \draw (g.east) node[snode] {g};
%      \node[pnode]                            (h) at ({0*\dl},{1*\dn}) {9};    \draw (h.east) node[snode] {h};
%      \node[pnode]                            (i) at ({0*\dl},{0*\dn}) {15};   \draw (i.east) node[snode] {i};
%      \begin{scope}[visible on=<4->]
%        \node[pnode] (bc) at ({-1*\dl},{6.5*\dn}) {8};
%        \draw   (bc) -- (b);
%        \draw   (bc) -- (c);
%      \end{scope}
%    \end{scope}
%
%    \begin{scope}[visible on=<5-6|handout:0>]
%      \node[pnode]                             (a)    at ({0*\dl},{8*\dn})     {16};   \draw (a.east) node[snode] {a};
%      \node[pnode]                             (bc)   at ({0*\dl},{6.5*\dn})   {8};
%      \node[pnode,fill=gray!25]                (b)    at ({1*\dl},{7*\dn})     {4};    \draw (b.east) node[snode] {b};
%      \node[pnode,fill=gray!25]                (c)    at ({1*\dl},{6*\dn})     {4};    \draw (c.east) node[snode] {c};
%      \node[pnode]                             (d)    at ({0*\dl},{5*\dn})     {16};   \draw (d.east) node[snode] {d};
%      \node[pnode]                             (e)    at ({0*\dl},{4*\dn})     {23};   \draw (e.east) node[snode] {e};
%      \node[pnode,alt={<6->{fill=gray!25}{}}]  (f)    at ({0*\dl},{3*\dn})     {7};    \draw (f.east) node[snode] {f};
%      \node[pnode,alt={<6->{fill=gray!25}{}}]  (g)    at ({0*\dl},{2*\dn})     {6};    \draw (g.east) node[snode] {g};
%      \node[pnode]                             (h)    at ({0*\dl},{1*\dn})     {9};    \draw (h.east) node[snode] {h};
%      \node[pnode]                             (i)    at ({0*\dl},{0*\dn})     {15};   \draw (i.east) node[snode] {i};
%      \draw   (bc)    -- (b);
%      \draw   (bc)    -- (c);
%      \begin{scope}[visible on=<6->]
%        \node[pnode] (fg) at ({-1*\dl},{2.5*\dn}) {13};
%        \draw   (fg) -- (f);
%        \draw   (fg) -- (g);
%      \end{scope}
%    \end{scope}
%    
%    \begin{scope}[visible on=<7-8|handout:0>]
%      \node[pnode]                             (a)    at ({0*\dl},{8*\dn})     {16};   \draw (a.east) node[snode] {a};
%      \node[pnode]                             (bc)   at ({0*\dl},{6.5*\dn})   {8};
%      \node[pnode,fill=gray!25]                (b)    at ({1*\dl},{7*\dn})     {4};    \draw (b.east) node[snode] {b};
%      \node[pnode,fill=gray!25]                (c)    at ({1*\dl},{6*\dn})     {4};    \draw (c.east) node[snode] {c};
%      \node[pnode]                             (d)    at ({0*\dl},{5*\dn})     {16};   \draw (d.east) node[snode] {d};
%      \node[pnode]                             (e)    at ({0*\dl},{4*\dn})     {23};   \draw (e.east) node[snode] {e};
%      \node[pnode]                             (fg)   at ({0*\dl},{2.5*\dn})   {13};
%      \node[pnode,fill=gray!25]                (f)    at ({1*\dl},{3*\dn})     {7};    \draw (f.east) node[snode] {f};
%      \node[pnode,fill=gray!25]                (g)    at ({1*\dl},{2*\dn})     {6};    \draw (g.east) node[snode] {g};
%      \node[pnode]                             (h)    at ({0*\dl},{1*\dn})     {9};    \draw (h.east) node[snode] {h};
%      \node[pnode]                             (i)    at ({0*\dl},{0*\dn})     {15};   \draw (i.east) node[snode] {i};
%      \draw   (bc)    -- (b);
%      \draw   (bc)    -- (c);
%      \draw   (fg)    -- (f);
%      \draw   (fg)    -- (g);
%    \end{scope}
%    
%    \begin{scope}[visible on=<8|handout:0>]
%      \node[anchor=north west,align=left] at ({\leftx},{\topy}) 
%      {\underline{next step:}\\[.5ex]re-order for better readability};
%    \end{scope}
%
%    \begin{scope}[visible on=<9-10|handout:0>]
%      \node[pnode]                             (e)    at ({0*\dl},{8*\dn})     {23};   \draw (e.east) node[snode] {e};
%      \node[pnode]                             (a)    at ({0*\dl},{7*\dn})     {16};   \draw (a.east) node[snode] {a};
%      \node[pnode]                             (d)    at ({0*\dl},{6*\dn})     {16};   \draw (d.east) node[snode] {d};
%      \node[pnode]                             (i)    at ({0*\dl},{5*\dn})     {15};   \draw (i.east) node[snode] {i};
%      \node[pnode]                             (fg)   at ({0*\dl},{3.5*\dn})   {13};
%      \node[pnode,fill=gray!25]                (f)    at ({1*\dl},{4*\dn})     {7};    \draw (f.east) node[snode] {f};
%      \node[pnode,fill=gray!25]                (g)    at ({1*\dl},{3*\dn})     {6};    \draw (g.east) node[snode] {g};
%      \node[pnode,alt={<10->{fill=gray!25}{}}] (h)    at ({0*\dl},{2*\dn})     {9};    \draw (h.east) node[snode] {h};
%      \node[pnode,alt={<10->{fill=gray!25}{}}] (bc)   at ({0*\dl},{0.5*\dn})   {8};
%      \node[pnode,fill=gray!25]                (b)    at ({1*\dl},{1*\dn})     {4};    \draw (b.east) node[snode] {b};
%      \node[pnode,fill=gray!25]                (c)    at ({1*\dl},{0*\dn})     {4};    \draw (c.east) node[snode] {c};
%      \draw   (bc)    -- (b);
%      \draw   (bc)    -- (c);
%      \draw   (fg)    -- (f);
%      \draw   (fg)    -- (g);
%      \begin{scope}[visible on=<10->]
%        \node[pnode] (hbc) at ({-1*\dl},{1.25*\dn}) {17};
%        \draw   (hbc) -- (h);
%        \draw   (hbc) -- (bc);
%      \end{scope}
%    \end{scope}
%
%    \begin{scope}[visible on=<11-12|handout:0>]
%      \node[pnode]                             (e)    at ({0*\dl},{8*\dn})     {23};   \draw (e.east) node[snode] {e};
%      \node[pnode]                             (a)    at ({0*\dl},{7*\dn})     {16};   \draw (a.east) node[snode] {a};
%      \node[pnode]                             (d)    at ({0*\dl},{6*\dn})     {16};   \draw (d.east) node[snode] {d};
%      \node[pnode,alt={<12->{fill=gray!25}{}}] (i)    at ({0*\dl},{5*\dn})     {15};   \draw (i.east) node[snode] {i};
%      \node[pnode,alt={<12->{fill=gray!25}{}}] (fg)   at ({0*\dl},{3.5*\dn})   {13};
%      \node[pnode,fill=gray!25]                (f)    at ({1*\dl},{4*\dn})     {7};    \draw (f.east) node[snode] {f};
%      \node[pnode,fill=gray!25]                (g)    at ({1*\dl},{3*\dn})     {6};    \draw (g.east) node[snode] {g};
%      \node[pnode]                             (hbc)  at ({0*\dl},{1.25*\dn})  {17};
%      \node[pnode,fill=gray!25]                (h)    at ({1*\dl},{2*\dn})     {9};    \draw (h.east) node[snode] {h};
%      \node[pnode,fill=gray!25]                (bc)   at ({1*\dl},{0.5*\dn})   {8};
%      \node[pnode,fill=gray!25]                (b)    at ({2*\dl},{1*\dn})     {4};    \draw (b.east) node[snode] {b};
%      \node[pnode,fill=gray!25]                (c)    at ({2*\dl},{0*\dn})     {4};    \draw (c.east) node[snode] {c};
%      \draw   (bc)    -- (b);
%      \draw   (bc)    -- (c);
%      \draw   (fg)    -- (f);
%      \draw   (fg)    -- (g);
%      \draw   (hbc)   -- (h);
%      \draw   (hbc)   -- (bc);
%      \begin{scope}[visible on=<12->]
%        \node[pnode] (ifg) at ({-1*\dl},{4.25*\dn})  {28};
%        \draw   (ifg) -- (i);
%        \draw   (ifg) -- (fg);
%      \end{scope}
%    \end{scope}
%    
%    \begin{scope}[visible on=<13-14|handout:0>]
%      \node[pnode]                             (e)    at ({0*\dl},{8*\dn})     {23};   \draw (e.east) node[snode] {e};
%      \node[pnode,alt={<14->{fill=gray!25}{}}] (a)    at ({0*\dl},{7*\dn})     {16};   \draw (a.east) node[snode] {a};
%      \node[pnode,alt={<14->{fill=gray!25}{}}] (d)    at ({0*\dl},{6*\dn})     {16};   \draw (d.east) node[snode] {d};
%      \node[pnode]                             (ifg)  at ({0*\dl},{4.25*\dn}) {28};
%      \node[pnode,fill=gray!25]                (i)    at ({1*\dl},{5*\dn})     {15};   \draw (i.east) node[snode] {i};
%      \node[pnode,fill=gray!25]                (fg)   at ({1*\dl},{3.5*\dn})   {13};
%      \node[pnode,fill=gray!25]                (f)    at ({2*\dl},{4*\dn})     {7};    \draw (f.east) node[snode] {f};
%      \node[pnode,fill=gray!25]                (g)    at ({2*\dl},{3*\dn})     {6};    \draw (g.east) node[snode] {g};
%      \node[pnode]                             (hbc)  at ({0*\dl},{1.25*\dn})  {17};
%      \node[pnode,fill=gray!25]                (h)    at ({1*\dl},{2*\dn})     {9};    \draw (h.east) node[snode] {h};
%      \node[pnode,fill=gray!25]                (bc)   at ({1*\dl},{0.5*\dn})   {8};
%      \node[pnode,fill=gray!25]                (b)    at ({2*\dl},{1*\dn})     {4};    \draw (b.east) node[snode] {b};
%      \node[pnode,fill=gray!25]                (c)    at ({2*\dl},{0*\dn})     {4};    \draw (c.east) node[snode] {c};
%      \draw   (bc)    -- (b);
%      \draw   (bc)    -- (c);
%      \draw   (fg)    -- (f);
%      \draw   (fg)    -- (g);
%      \draw   (hbc)   -- (h);
%      \draw   (hbc)   -- (bc);
%      \draw   (ifg)   -- (i);
%      \draw   (ifg)   -- (fg);
%      \begin{scope}[visible on=<14->]
%        \node[pnode] (ad) at ({-1*\dl},{6.5*\dn})  {32};
%        \draw   (ad) -- (a);
%        \draw   (ad) -- (d);
%      \end{scope}
%    \end{scope}
%
%    \begin{scope}[visible on=<15-16|handout:0>]
%      \node[pnode]                             (e)    at ({0*\dl},{8*\dn})     {23};   \draw (e.east) node[snode] {e};
%      \node[pnode]                             (ad)   at ({0*\dl},{6.5*\dn})   {32};
%      \node[pnode,fill=gray!25]                (a)    at ({1*\dl},{7*\dn})     {16};   \draw (a.east) node[snode] {a};
%      \node[pnode,fill=gray!25]                (d)    at ({1*\dl},{6*\dn})     {16};   \draw (d.east) node[snode] {d};
%      \node[pnode]                             (ifg)  at ({0*\dl},{4.25*\dn})  {28};
%      \node[pnode,fill=gray!25]                (i)    at ({1*\dl},{5*\dn})     {15};   \draw (i.east) node[snode] {i};
%      \node[pnode,fill=gray!25]                (fg)   at ({1*\dl},{3.5*\dn})   {13};
%      \node[pnode,fill=gray!25]                (f)    at ({2*\dl},{4*\dn})     {7};    \draw (f.east) node[snode] {f};
%      \node[pnode,fill=gray!25]                (g)    at ({2*\dl},{3*\dn})     {6};    \draw (g.east) node[snode] {g};
%      \node[pnode]                             (hbc)  at ({0*\dl},{1.25*\dn})  {17};
%      \node[pnode,fill=gray!25]                (h)    at ({1*\dl},{2*\dn})     {9};    \draw (h.east) node[snode] {h};
%      \node[pnode,fill=gray!25]                (bc)   at ({1*\dl},{0.5*\dn})   {8};
%      \node[pnode,fill=gray!25]                (b)    at ({2*\dl},{1*\dn})     {4};    \draw (b.east) node[snode] {b};
%      \node[pnode,fill=gray!25]                (c)    at ({2*\dl},{0*\dn})     {4};    \draw (c.east) node[snode] {c};
%      \draw   (bc)    -- (b);
%      \draw   (bc)    -- (c);
%      \draw   (fg)    -- (f);
%      \draw   (fg)    -- (g);
%      \draw   (hbc)   -- (h);
%      \draw   (hbc)   -- (bc);
%      \draw   (ifg)   -- (i);
%      \draw   (ifg)   -- (fg);
%      \draw   (ad) -- (a);
%      \draw   (ad) -- (d);
%    \end{scope}
%
%    \begin{scope}[visible on=<16|handout:0>]
%      \node[anchor=north west,align=left] at ({\leftx},{\topy}) 
%      {\underline{next step:}\\[.5ex]re-order for better readability};
%    \end{scope}
%
%    \begin{scope}[visible on=<17-18|handout:0>]
%      \node[pnode]                             (ad)   at ({0*\dl},{7.5*\dn})   {32};
%      \node[pnode,fill=gray!25]                (a)    at ({1*\dl},{8*\dn})     {16};   \draw (a.east) node[snode] {a};
%      \node[pnode,fill=gray!25]                (d)    at ({1*\dl},{7*\dn})     {16};   \draw (d.east) node[snode] {d};
%      \node[pnode]                             (ifg)  at ({0*\dl},{5.25*\dn})  {28};
%      \node[pnode,fill=gray!25]                (i)    at ({1*\dl},{6*\dn})     {15};   \draw (i.east) node[snode] {i};
%      \node[pnode,fill=gray!25]                (fg)   at ({1*\dl},{4.5*\dn})   {13};
%      \node[pnode,fill=gray!25]                (f)    at ({2*\dl},{5*\dn})     {7};    \draw (f.east) node[snode] {f};
%      \node[pnode,fill=gray!25]                (g)    at ({2*\dl},{4*\dn})     {6};    \draw (g.east) node[snode] {g};
%      \node[pnode,alt={<18->{fill=gray!25}{}}] (e)    at ({0*\dl},{3*\dn})     {23};   \draw (e.east) node[snode] {e};
%      \node[pnode,alt={<18->{fill=gray!25}{}}] (hbc)  at ({0*\dl},{1.25*\dn})  {17};
%      \node[pnode,fill=gray!25]                (h)    at ({1*\dl},{2*\dn})     {9};    \draw (h.east) node[snode] {h};
%      \node[pnode,fill=gray!25]                (bc)   at ({1*\dl},{0.5*\dn})   {8};
%      \node[pnode,fill=gray!25]                (b)    at ({2*\dl},{1*\dn})     {4};    \draw (b.east) node[snode] {b};
%      \node[pnode,fill=gray!25]                (c)    at ({2*\dl},{0*\dn})     {4};    \draw (c.east) node[snode] {c};
%      \draw   (bc)    -- (b);
%      \draw   (bc)    -- (c);
%      \draw   (fg)    -- (f);
%      \draw   (fg)    -- (g);
%      \draw   (hbc)   -- (h);
%      \draw   (hbc)   -- (bc);
%      \draw   (ifg)   -- (i);
%      \draw   (ifg)   -- (fg);
%      \draw   (ad) -- (a);
%      \draw   (ad) -- (d);
%      \begin{scope}[visible on=<18->]
%        \node[pnode] (ehbc) at ({-1*\dl},{2.125*\dn})  {40};
%        \draw   (ehbc) -- (e);
%        \draw   (ehbc) -- (hbc);
%      \end{scope}
%    \end{scope}
%
%    \begin{scope}[visible on=<19-20|handout:0>]
%      \node[pnode,alt={<20->{fill=gray!25}{}}] (ad)   at ({0*\dl},{7.5*\dn})   {32};
%      \node[pnode,fill=gray!25]                (a)    at ({1*\dl},{8*\dn})     {16};   \draw (a.east) node[snode] {a};
%      \node[pnode,fill=gray!25]                (d)    at ({1*\dl},{7*\dn})     {16};   \draw (d.east) node[snode] {d};
%      \node[pnode,alt={<20->{fill=gray!25}{}}] (ifg)  at ({0*\dl},{5.25*\dn})  {28};
%      \node[pnode,fill=gray!25]                (i)    at ({1*\dl},{6*\dn})     {15};   \draw (i.east) node[snode] {i};
%      \node[pnode,fill=gray!25]                (fg)   at ({1*\dl},{4.5*\dn})   {13};
%      \node[pnode,fill=gray!25]                (f)    at ({2*\dl},{5*\dn})     {7};    \draw (f.east) node[snode] {f};
%      \node[pnode,fill=gray!25]                (g)    at ({2*\dl},{4*\dn})     {6};    \draw (g.east) node[snode] {g};
%      \node[pnode]                             (ehbc) at ({0*\dl},{2.125*\dn}) {40};
%      \node[pnode,fill=gray!25]                (e)    at ({1*\dl},{3*\dn})     {23};   \draw (e.east) node[snode] {e};
%      \node[pnode,fill=gray!25]                (hbc)  at ({1*\dl},{1.25*\dn})  {17};
%      \node[pnode,fill=gray!25]                (h)    at ({2*\dl},{2*\dn})     {9};    \draw (h.east) node[snode] {h};
%      \node[pnode,fill=gray!25]                (bc)   at ({2*\dl},{0.5*\dn})   {8};
%      \node[pnode,fill=gray!25]                (b)    at ({3*\dl},{1*\dn})     {4};    \draw (b.east) node[snode] {b};
%      \node[pnode,fill=gray!25]                (c)    at ({3*\dl},{0*\dn})     {4};    \draw (c.east) node[snode] {c};
%      \draw   (bc)    -- (b);
%      \draw   (bc)    -- (c);
%      \draw   (fg)    -- (f);
%      \draw   (fg)    -- (g);
%      \draw   (hbc)   -- (h);
%      \draw   (hbc)   -- (bc);
%      \draw   (ifg)   -- (i);
%      \draw   (ifg)   -- (fg);
%      \draw   (ad) -- (a);
%      \draw   (ad) -- (d);
%      \draw   (ehbc) -- (e);
%      \draw   (ehbc) -- (hbc);
%      \begin{scope}[visible on=<20->]
%        \node[pnode] (adifg) at ({-1*\dl},{6.375*\dn})  {60};
%        \draw   (adifg) -- (ad);
%        \draw   (adifg) -- (ifg);
%      \end{scope}
%    \end{scope}
%
%    \begin{scope}[visible on=<21->]
%      \node[pnode,alt={<22->{fill=gray!25}{}}] (adifg) at ({0*\dl},{6.375*\dn}) {60};
%      \node[pnode,fill=gray!25]                (ad)    at ({1*\dl},{7.5*\dn})   {32};
%      \node[pnode,fill=gray!25]                (a)     at ({2*\dl},{8*\dn})     {16};   \draw (a.east) node[snode] {a};
%      \node[pnode,fill=gray!25]                (d)     at ({2*\dl},{7*\dn})     {16};   \draw (d.east) node[snode] {d};
%      \node[pnode,fill=gray!25]                (ifg)   at ({1*\dl},{5.25*\dn})  {28};
%      \node[pnode,fill=gray!25]                (i)     at ({2*\dl},{6*\dn})     {15};   \draw (i.east) node[snode] {i};
%      \node[pnode,fill=gray!25]                (fg)    at ({2*\dl},{4.5*\dn})   {13};
%      \node[pnode,fill=gray!25]                (f)     at ({3*\dl},{5*\dn})     {7};    \draw (f.east) node[snode] {f};
%      \node[pnode,fill=gray!25]                (g)     at ({3*\dl},{4*\dn})     {6};    \draw (g.east) node[snode] {g};
%      \node[pnode,alt={<22->{fill=gray!25}{}}] (ehbc)  at ({0*\dl},{2.125*\dn}) {40};
%      \node[pnode,fill=gray!25]                (e)     at ({1*\dl},{3*\dn})     {23};   \draw (e.east) node[snode] {e};
%      \node[pnode,fill=gray!25]                (hbc)   at ({1*\dl},{1.25*\dn})  {17};
%      \node[pnode,fill=gray!25]                (h)     at ({2*\dl},{2*\dn})     {9};    \draw (h.east) node[snode] {h};
%      \node[pnode,fill=gray!25]                (bc)    at ({2*\dl},{0.5*\dn})   {8};
%      \node[pnode,fill=gray!25]                (b)     at ({3*\dl},{1*\dn})     {4};    \draw (b.east) node[snode] {b};
%      \node[pnode,fill=gray!25]                (c)     at ({3*\dl},{0*\dn})     {4};    \draw (c.east) node[snode] {c};
%      \draw   (bc)    -- (b);
%      \draw   (bc)    -- (c);
%      \draw   (fg)    -- (f);
%      \draw   (fg)    -- (g);
%      \draw   (hbc)   -- (h);
%      \draw   (hbc)   -- (bc);
%      \draw   (ifg)   -- (i);
%      \draw   (ifg)   -- (fg);
%      \draw   (ad) -- (a);
%      \draw   (ad) -- (d);
%      \draw   (ehbc) -- (e);
%      \draw   (ehbc) -- (hbc);
%      \draw   (adifg) -- (ad);
%      \draw   (adifg) -- (ifg);
%      \begin{scope}[visible on=<22->]
%        \node[pnode,fill=gray!25] (root) at ({-1*\dl},{4.25*\dn})  {\relsize{-1}100};
%        \draw   (root) -- (adifg);
%        \draw   (root) -- (ehbc);
%        \node [anchor=north] at (root.south) {root};
%      \end{scope}
%    \end{scope}
%
%    \begin{scope}[visible on=<23-24|handout:0>]
%      \node[anchor=north west,align=left] at ({\leftx},{\topy}) 
%      {\underline{next step:}\\[.5ex]label branches with {\color{blue}0} and {\color{blue}1}};
%    \end{scope}
%
%    \begin{scope}[visible on=<24->]
%      \draw (root)--(adifg)                    node [lbla] {1};
%      \draw         (adifg)--(ad)              node [lbla] {1};
%      \draw                  (ad)--(a)         node [lbla] {1};
%      \draw                  (ad)--(d)         node [lblb] {0};
%      \draw         (adifg)--(ifg)             node [lblb] {0};
%      \draw                  (ifg)--(i)        node [lbla] {1};
%      \draw                  (ifg)--(fg)       node [lblb] {0};
%      \draw                         (fg)--(f)  node [lbla] {1};
%      \draw                         (fg)--(g)  node [lblb] {0};
%      \draw (root)--(ehbc)                     node [lblb] {0};
%      \draw         (ehbc)--(e)                node [lbla] {1};
%      \draw         (ehbc)--(hbc)              node [lblb] {0};
%      \draw                 (hbc)--(h)         node [lbla] {1};
%      \draw                 (hbc)--(bc)        node [lblb] {0};
%      \draw                        (bc)--(b)   node [lbla] {1};
%      \draw                        (bc)--(c)   node [lblb] {0};
%    \end{scope}
%
%    \begin{scope}[visible on=<25-26|handout:0>]
%      \node[anchor=north west,align=left] at ({\leftx},{\topy}) 
%      {\underline{next step:}\\[.5ex]assign codewords\\(follow branches from root\\to terminal nodes)};
%    \end{scope}
%
%    \begin{scope}[visible on=<26|handout:0>]
%      \draw[hl] (root)--(adifg);
%      \draw[hl]         (adifg)--(ifg);
%      \draw[hl]                  (ifg)--(fg);
%      \draw[hl]                         (fg)--(g);
%    \end{scope}
%
%    \begin{scope}[visible on=<28>]
%      \node[anchor=north west,align=left] at ({\leftx},{\topy}) 
%      {$\begin{array}{rcl}
%          \bar{\ell} &=& 2.98\\[.5ex]
%          H(p)&\approx&2.9405\\[.5ex]
%          \varrho&\approx&0.0395~(1.34\%)
%        \end{array}$
%      };
%    \end{scope}
%
%    \begin{scope}[visible on=<0|handout:1>]
%      \node[anchor=north west,align=left] at ({-1.8*\dl},{0*\dn}) 
%      {\underline{note:} nodes are labeled with {\color{officegreen}$100\,p_k$}};
%    \end{scope}
%  \end{tikzpicture}
%}


\begin{frame}{Example: Construction of a Huffman Code}
  \slidefitbox{\HuffmanExample}
\end{frame}


 

%\begin{frame}{Optimality of Huffmann codes}
%
%\begin{lemma}
%Let $\gamma$ be an optimal code. If $a, b\in\mathcal{A}$ satisfy 
%$p_a<p_b$, then $\ell(\gamma(a))\geq \ell(\gamma(b))$.
%\end{lemma}
%\loud{Proof: } Otherwise, exchanging codewords of $a$ and $b$ would give a code with strictly smaller average codeword length.
%
%\begin{lemma}
%Let $\gamma$ be an optimal prefix code. Let $\ell_{max}(\gamma)$ be the maximal length of all codewords of $\gamma$. 
%\bit
%\item There are two codewords $\gamma(a)$ and $\gamma(b)$ of length $\ell_{max}(\gamma)$ that differ only in the final bit and such that 
%$p_a$ is smallest symbol probability. 
%\item There exists a prefix code with the same average codeword length as $\gamma$ for which there are two codewords $\gamma(a_{i_1})$ and $\gamma(a_{i_2})$ of length $\ell_{max}(\gamma)$ that differ only in the final bit and for which $p_{i_1}$ and $p_{i_2}$ correspond to the two smallest symbol 
%probabilities. 
%\eit 
%\end{lemma}
%\loud{Proof of second statemen:} Let $a$ and $b$ be as in the first statement. Let $c$ be a smybol such that $p_a$ and $p_c$ are 
%the two smallest symbol probabilites. Exchange codewords for $b$ and $c$. Since $p_c\leq p_b$ and since $\ell(\gamma(b))\geq \ell(\gamma(c))$, resulting codes can not have larger average codeword length. Codeword lengths are the same since $\gamma$ was assumed to be optimal. 
%\end{frame}

\subsection{Optimality of Huffman codes}

\begin{frame}{Optimality of Huffman codes: First lemma}
\begin{lemma}[First Lemma for optimal prefix codes]
Let $\gamma$ be an optimal prefix code. Let $\ell_{max}(\gamma)$ be the maximal length of all codewords of $\gamma$. 
There are two codewords $\gamma(a)$ and $\gamma(b)$ of length $\ell_{max}(\gamma)$ that differ only in the final bit and such that 
$p_a$ is the smallest symbol probability.
\end{lemma}
\loud{Proof:} 
\bit
\item Let $\mathcal{A}_{max}$ be the set of all symbols whose codewords have length $\ell_{max}(\gamma)$.
\only<2->{\item Let $a\in\mathcal{A}_{max}$ have smallest probability $p_a$ among all symbols of $\mathcal{A}_{max}$. }
\only<3->{\item If there exists
symbol $c\notin\mathcal{A}_{max}$  with $p_c<p_a$, exchanging codewords of $a$ and $c$ would 
give code with strictly smaller average codeword length than $\gamma$. }
\only<4->{\item [\iarrow] $p_a$ has to be smallest probability among all
symbols of $\mathcal{A}$. }
\only<5->{\item[\iarrow]Last lecture: Since $\gamma$ is optimal, tree representing $\gamma$ has to be proper.
Thus, parent node of $a$ has a child $b$ different from $a$. Then $b$ satisfies required properties.}
\eit
\end{frame}

\begin{frame}{Optimality of Huffman codes: Second lemma}
\begin{lemma}[Second Lemma for optimal prefix codes]
There exists an optimal prefix code $\gamma$ with the following property: 
There are two codewords $\gamma(a)$ and $\gamma(b)$ which are of maximal length, differ only in the final bit and such that 
$p_a$ and $p_b$ are the two smallest symbol probabilities 
\end{lemma}
\loud{Proof:} 
\bit
\item Let $\gamma$ be an optimal prefix code.
\only<2->{\item  Let $a$ and $b$ be as in previous lemma.}
\only<3->{\item Let $c$  be symbol such that $p_a$ and $p_c$ are the smallest probabilities. Then $p_c\leq p_b$ and $\ell(\gamma(c))\geq \ell(\gamma(b))$. }
\only<4->{\item If $p_c=p_b$, Lemma is proved.}
\only<5->{\item Assume that $p_c<p_b$. Since $b$ has maximal codeword length, exchanging codewords of $b$ and $c$ gives code $\gamma'$ with $\overline{\ell}(\gamma')\leq \overline{\ell}(\gamma)$.}
\only<6->{\item[\iarrow] Since $\gamma$ is optimal:  Exchanging $b$ and $c$ gives code with same average codeword length and desired property. }
\eit 
\end{frame}

\begin{frame}{Optimality of Huffman codes}
\begin{theorem}
The Huffman algorithm yields a prefix code that minimizes the average codeword length among
all uniquely decodable codes. 
\end{theorem}
\loud{Proof:}
\bit
\item Proof by induction on alphabet-size $N$. For $N=2$, statement obviously true. 
\only<2->{\item Let $\gamma$ be an optimal prefix code  as in second lemma, symbols $a, b$ as in second lemma.}
\only<3->{\item Represent $\gamma$ by tree. Merge $a$ and $b$ to their single parent node, get tree $\mathcal{T}^{\#}$.}
\only<4->{\item Let $\mathcal{A}^{\#}$ be alphabet arising from $\mathcal{A}$ by merging $a$ and $b$ to a single symbol
with probability $p_a+p_b$. Keep other symbol probabilities. }
\only<5->{\item Let $\gamma_1$ be prefix code for $\mathcal{A}^{\#}$ represented by $\mathcal{T}^{\#}$ and let $\gamma_{2}$ be Huffman code for $\mathcal{A}^{\#}$.}
%\item One has $\overline{\ell}(\gamma)=\overline{\ell}(\gamma_1)+p_a+p_b$ by construction. 
\only<6->{\item Induction hypothesis: $\overline{\ell}(\gamma_1)\geq \overline{\ell}(\gamma_2)$.}
\only<7->{\item Thus $\overline{\ell}(\gamma)=\overline{\ell}(\gamma_1)+ p_a \ell_a+ p_b \ell_b \geq \overline{\ell}(\gamma_2)+p_a \ell_a+ p_b \ell_b = \overline{\ell}(\gamma_2)+\ell_{a} (p_a + p_b)$}
\only<8->{\item Right hand side is average codeword-length of a Huffman code for $\mathcal{A}$.}
\qed 
\eit
\end{frame}



\section{Coding of multiple sources}


\begin{frame}
 \vspace{8.0ex}
\begin{center}
\begin{beamercolorbox}[sep=12pt,center]{part title}
\usebeamerfont{section title}\insertsection\par
\end{beamercolorbox}
\end{center}
\end{frame}

\subsection{Motivation}

\begin{frame}{Motivational example}
\ALERT{How to model the sources that one wants to compress?}
\begin{figure}
\centering
\includegraphics[width=0.33\textwidth]{Lossless_II/Shannon_Bio.png}
%\captionsetup{labelformat=empty}
%\caption{Example for handwritten digits from MNIST.}
\end{figure}
\bit
\item Assume one wants to design a lossless compression of texts written in German language.
\item Inspecting the letter $h$ as an example: 
\bit 
\item Frequency of $h$ varies depending on previous letters. 
\item Letters after $h$ have different distribution than letters after i.g. letter $a$. 
\eit 
\eit
\end{frame}



\begin{frame}{Motivational example}

\begin{minipage}[t]{0.8\linewidth}
\loud{First observation:} For a given position in the text,  probability of a given letter depends heavily 
on previous letters.
\bit\small
\item Probability of letter $h$ is very large if previous letter was a $c$ but very small if previous letter was e.g. a $k$. 
\item The more previous letters are considered, the better the probability for a given letter can be estimated. If previous
two letters are $sc$, probability of $h$ is even larger.  
\item Probability should never be one for any letter: Text may always contain typos. Process not deterministic.
\eit
 \ALERT{\iarrow $\:$ Considering conditional probabilities between letters should be very useful for compression}. 
\end{minipage}

\begin{minipage}[t]{0.8\linewidth}
\loud{Second observation:} The actual position within a text is not relevant for the probability of a given letter 
or of a given block of letters.

\smallskip
\ALERT{\iarrow $\:$ Sequence of letters should be modelled as a stationary process}.
\end{minipage}

\end{frame}


\begin{frame}{Motivational example}
\begin{minipage}[t]{0.9\linewidth}
\loud{Third observation:} 
\bit\small
\item The probability distribution on the ``set of all texts'' completely determines the probability 
of a given letter or a block of letters. 
\item \textbf{However: } It seems only realistic to model the probability of a given letter or of some finite small sequence of letters. 
\item Underlying probability on the ``set of all texts'' remains unknown.
\eit 
\smallskip
\ALERT{\iarrow $\:$ Full sample space with its probability only used formally, not explicitly. Rather: Random variables on the sample space 
and their distributions are considered}.
\end{minipage}

\begin{minipage}[t]{0.9\linewidth}
\loud{Case of images or videos:} Similar situation.
\bit\small
\item First example: Can think of each sample position and sample value as analogon for letters in texts. Conditional dependencies 
more complicated. 
\item In typical codecs: Consider quantized transform coefficients or side information as analogon. 
\item Again modeled as a stationary random process with strong conditional dependencies. 
%\item Underlying distribution on the full sample space, ``set of all natural images'' unknonw.   
\eit
\smallskip
\ALERT{\iarrow $\:$ Make above observations formal in the sequel!} 
\end{minipage}


\end{frame}



\subsection{Discrete probability spaces}

\begin{frame}{Discrete probability spaces}
\loud{Discrete probability space $(P,\Omega)$:} 
\bit
\item 
\loud{Sample space:} Discrete set  $\Omega$. 
\item \loud{Set of events:} Power set 
\[
\mathfrak{P}(\Omega)=\{A|A\subset\Omega\}
\]
of $\Omega$. 
\item \loud{Probability:} Function $P:\mathfrak{P}(\Omega)\to[0,1]$, such that
\begin{equation*}
P(\Omega)=1
\end{equation*}
and such that
\begin{align*}
P(\bigsqcup_{i\in I}A_i)=\sum_{i\in I}P(A_i)
\end{align*}
for each countable index set $I$ an each family $(A_i)_{i\in I}$ of events with $A_i\cap A_j=\emptyset$ for $i\neq j$. 
\eit 
\end{frame}

\begin{frame}{Discrete random variables} 

\loud{Discrete random variable:}
\bit
\item 
Function $X:\Omega\to\mathbb{R}$. 
\item The \loud{alphabet} $\mathcal{A}_X$ of 
$X$ is defined as $\mathcal{A}_X=\{X(\omega)\colon \omega\in\Omega\}$. 
\eit
\smallskip
\loud{Probability mass function:} 
\bit
\item Function $p_X:\mathbb{R}\to [0,1]$, given by 
\[
p_X(x)=P(X=x)=P(\{\omega\in\Omega\colon X(\omega)=x\}).
\]
\item One has $0\leq p_X(a)\leq 1$ for all $a\in\mathcal{A}_X$ and $\sum_{a\in\mathcal{A}_X}p_X(a)=1$.
\eit
\smallskip
\loud{Probability distribution:} 
\bit
\item Function $p_X:\mathfrak{P}(\mathbb{R})\to \mathbb{R}$ defined as 
\[
p_X(B):=P(X\in B)=P(\{\omega\in\Omega\colon X(\omega)\in B\}),\quad B\subseteq\mathbb{R}.
\]
\item One has 
\[
p_X(B)=\sum_{a \in \mathcal{A}_X\cap B}p_X(a).
\]
\eit 
\end{frame}

\begin{frame}{Joint distributions}
 Let $X$, $Y$ be random variables on $\Omega$.

\loud{Joint probability mass function:}
\bit
\item Function $p_{X,Y}:\mathbb{R}^2\to\mathbb{R}$, defined as 
\begin{align*}
p_{X,Y}(x,y)=P((X=x) \cap (Y=y))=P(\{\omega\in\Omega\colon X(\omega)=x\text{ and }Y(\omega)=y\}).
\end{align*}
\eit
\loud{Joint distribution:}
\bit
\item 
Function $p_{X,Y}:\mathcal{P}(\mathbb{R}^2)\to\mathbb{R}$ defined as 
\begin{align*}
p_{X,Y}(A)=P(\{\omega\in\Omega\colon (X(\omega),Y(\omega))\in A\}),\quad A\subseteq\mathbb{R}^2.
\end{align*}
\eit
\loud{Marginal distributions: }
\bit
\item Distributions $p_X$ and $p_Y$.
\item One has 
\begin{align}\label{MarginalJoint}
p_X(x)=\sum_{y\in\mathcal{A}_Y}p(x,y);\quad p_Y(y)=\sum_{x\in\mathcal{A}_X}p(x,y). 
\end{align}
\eit 
\end{frame} 


\begin{frame}{Conditional probability} 
\loud{Conditional probability given an event}
\bit
\item For an event $B\in\mathfrak{P}(\Omega)$ with $P(B)>0$ and any event $A\in\mathfrak{P}(\Omega)$, define 
\[
P(A|B):=\frac{P(A\cap B)}{P(B)}
\]
as the conditional probability of $A$ given $B$ .
\eit
\loud{Conditional probability for random variables:}
\bit
\item  Let $X$, $Y$ be random variables on $\Omega$ with alphabets $\mathcal{A}_X$ and $\mathcal{A}_Y$. 
\item For $x\in\mathcal{A}_X$ with $p_X(x)>0$ and for $y\in\mathcal{A}_Y$, the conditional probability $p_{Y|X}(y|x)$ of $y$ given $x$ is 
\begin{align*}
p_{Y|X}(y|x)=\frac{p_{X,Y}(x,y)}{p_X(x)}.
\end{align*}
\item $X$ and $Y$ are \loud{independent} if $p_{X,Y}=p_X\cdot p_Y$, i.e. $p_{Y|X}(y|x)=p_Y(y)$ for all $y$, all $x$ with $p_X(x)>0$.
\eit
\end{frame}


\begin{frame}{Discrete random process} 
\loud{Joint, marginal and conditional probabilities of $N$ random variables:}
\bit
\item Above definitions for two random variables $X$ and $Y$ extend directly to the case of $N$ random variables $X_1,\dots,X_N$.
\item Joint probability mass functions and distributions $p_{X_1,\dots,X_N}$ are defined as above.
\item Conditional distribution $p_{X_N|X_1,\dots,X_{N-1}}(x_N|x_1,\dots,x_{N-1})$ defined as above. 
\eit
\loud{Discrete random process:}
\bit
\item A sequence $\mathcal{X}=(X_n)_{n=1}^\infty$ of discrete random variables on $\Omega$.
\eit
\loud{Stationary random process $\mathcal{X}$:}
\bit
\item All consecutive joint distributions are invariant under time. One has
\[
p_{X_1,\dots,X_N}=p_{X_{1+T},\dots,X_{N+T}}
\]
for all $T\in\mathbb{N}$ and all $N\in\mathbb{N}$. 
\eit
\end{frame}


\subsection{Setup}


\begin{frame}{Coding of combined sources}
\loud{Goal:} 
\bit
\item Consider sources $X$ and $Y$ with alphabets $\mathcal{A}_X,  \mathcal{A}_Y$.
\item Assume that joint distribution $p_{X,Y}$ is given.
\item Want to code symbols of the combined alphabet $\mathcal{A}_X\times \mathcal{A}_Y$. Write symbols as $xy$, $x\in\mathcal{A}_X$, $y\in\mathcal{A}_Y$. 
\eit

\loud{Joint coding:}
\bit
\item Can treat joint distribution $p_{X,Y}$, alphabet $\mathcal{A}_X\times\mathcal{A}_Y$ as single distribution and single alphabet. 
\item Then: Proceed as before, Shannon Code or Huffman Code.
\item Joint entropy of  $p_{X,Y}$ is lower bound for optimal average codeword length. 
\eit
\loud{Problems of joint coding:}
\bit
\item Alphabet $\mathcal{A}_X\times\mathcal{A}_Y$ can be very large compared to individual alphabets $\mathcal{A}_X$, $\mathcal{A}_Y$. 
\item Instantaneous decoding of individual symbols $x$ and $y$ from $xy$ not guaranteed.
\item Even for joint prefix code: May need to wait for whole codeword for $xy$ before $x$ can be decoded. 
\item Problems become worse if more than two sources are considered. 
\eit 
\end{frame}

\subsection{Marginal and conditional coding}

\begin{frame}{Marginal and conditional coding}
\loud{Marginal coding}
\bit
\item Let $\gamma_X$ prefix code for $\mathcal{A}_X$ and $\gamma_Y$ prefix code for $\mathcal{A}_Y$. 
\item Define a uniquely decodable code (\textbf{exercise!}) on symbols $xy$, with $x\in\mathcal{A}_X, \:y\in\mathcal{A}_Y$ by :
\bit
\item Code $x$ using $\gamma_X$.
\item Code $y$ using $\gamma_Y$. 
\eit
\item This seems to solve above problems of joint coding. 
\item \loud{But: } Possible dependencies between symbols $x$ and $y$ are not reflected. Efficiency may suffer. 
\eit
\loud{Conditional coding} 
\bit
\item Let $\gamma_X$ prefix code for $\mathcal{A}_X$. 
\item For each $x_i$ in $\mathcal{A}_X$, let $\gamma_Y^{i}$ prefix code. Can use different codes for different $x_i$. 
\item Define a uniquely decodable code on symbols $xy$, with $x\in\mathcal{A}_X, \:y\in\mathcal{A}_Y$ by :
\bit
\item Code $x$ using $\gamma_X$.
\item If $x=x_i$, code $y$ using $\gamma_Y^i$. 
\eit
\item Exploit dependencies between $x$ and $y$ by optimizing $\gamma_Y^i$ for conditional probability $p(\cdot|x_i)$. 
\eit 
\end{frame}


\begin{frame}{Average codeword lengths of marginal and conditional coding}
\bit
\item Average codeword length $\overline{\ell}_m$ of marginal code is (\textbf{exercise!})
\begin{align}\label{CWMarg}
\overline{\ell}_m= \overline{\ell}(\gamma_X)+\overline{\ell}(\gamma_Y) 
\end{align}
\item Average codeword length $\overline{\ell}_c$ of conditional code:
\begin{align}
\overline{\ell}_c=&\sum_{x_i\in\mathcal{A}_X}\sum_{y\in\mathcal{A}_Y}p(x_i,y)(\ell(\gamma_X(x_i))+\ell(\gamma_Y^i(y)))\nonumber\\
=&\sum_{x_i\in\mathcal{A}_X}p(x_i)\ell(\gamma_X(x_i))+\sum_{x_i\in\mathcal{A}_X}p(x_i)\sum_{y\in\mathcal{A}_Y}p(y|x_i)\ell(\gamma_Y^i(y))\label{EqCondL}\\
=&\overline{\ell}(\gamma_X)+\sum_{x_i\in\mathcal{A}_X}p(x_i)\overline\ell(\gamma_Y^i)\nonumber,
\end{align}
where $\overline\ell(\gamma_Y^i)$ taken with respect to $p(\cdot|x_i)$.
\item Choose optimal codes $\gamma_Y^{i}$. Then by optimality
\[
\sum_{y\in\mathcal{A}_Y}p(y|x_i)\ell(\gamma_Y^i(y))\leq \sum_{y\in\mathcal{A}_Y}p(y|x_i)\ell(\gamma_Y(y)).
\] 
\eit
\end{frame}

\begin{frame}{Comparison of optimal average codeword lengths for marginal and conditional coding}
\bit
\item Using the optimal codes $\gamma_Y^{i}$, one has
\begin{align*}
\sum_{x_i\in\mathcal{A}_X}p(x_i)\sum_{y\in\mathcal{A}_Y}p(y|x_i)\ell(\gamma_Y^i(y))\leq& \sum_{x_i\in\mathcal{A}_X}p(x_i)\sum_{y\in\mathcal{A}_Y}p(y|x_i)\ell(\gamma_Y(y))
\\=&\sum_{y\in\mathcal{A}_Y}\sum_{x_i\in\mathcal{A}_X}p(x_i,y)\ell(\gamma_Y(y))
\\=& \sum_{y\in\mathcal{A}_Y}p(y)\ell(\gamma_Y(y))
\\=&\overline{\ell}(\gamma_Y).
\end{align*}
\item Applying \eqref{CWMarg} and \eqref{EqCondL} implies: 
\item [\iarrow] \loud{Optimal average codeword length of conditional coding is less or equal than optimal average codeword length of marginal coding.}
\eit
\end{frame} 


\subsection{Examples for conditional coding}

\begin{frame}{Example for benefit of conditional coding}
\loud{Given distribution: }
\bit
\item Consider sources $X$ and $Y$ with alphabets $\mathcal{A}_X=\{0,1\}$ and $\mathcal{A}_Y=\{0,1,2\}$. 
\item Assume that the joint probability $p_{X,Y}$ is given as
\begin{center} 
 \begin{tabular}[hbt!]{ |c|c|c| } 
 \hline
 x & y & $p_{X,Y}(x,y)$ \\
 \hline 
0 & 0 & 1/4\\  
0 & 1 & 1/8\\
0 & 2 &  1/8\\
1 & 0 & 1/8\\
1 & 1 &  1/8\\
1& 2& 1/4\\
 \hline
\end{tabular}.
\end{center}
\eit
\smallskip
%\item Joint entropy: 
%\begin{align*}
%H(X,Y)=-2\cdot 1/4\log_2(1/4)-4\cdot 1/8\log_2(1/8)=1+12/8=5/2.
%\end{align*}
\loud{Optimal average codeword length is 5/2=20/8:}
\bit
\item Each $p_{X,Y}(x,y)$ is negative 
integer power of two, thus entropy equals average codeword length. Compute entropy, \textbf{exercise!}
\item Alternative: Write down Shannon-Code or Huffman-Code explicitly, \textbf{exercise!}
\eit
\end{frame}

\begin{frame}{Example for benefit of conditional coding}
\loud{Marginal probabilities:}
\bit 
\item $p_X(0)=p_X(1)=1/2$
\item $p_Y(0)=3/8, \:p_Y(1)=1/4, \:p_Y(2)=3/8.$
\eit
%\item Marginal entropies: 
%\bit
%\item $H(X)=1$
%\item $H(Y)=-6/8\log_2(3/8)-1/4\log_2(1/4)$.
%\eit
%\eit
\loud{Average codeword length using optimal marginal code is 21/8:}
\bit
\item Optimal codes for marginals $X$ and $Y$:
\bit
\item Code $\gamma_X$ with $\gamma_X(0)=0,\:\gamma_X(1)=1$ is optimal code for $p_X$ with average 
codeword length $\bar{\ell}_{\gamma_X} = 1$.
\item Code $\gamma_Y$ with $\gamma_Y(0)=0,\: \gamma_Y(1)=10,\: \gamma_Y(2)= 11$ is an optimal code for $p_Y$ with average 
codeword length $\bar{\ell}_{\gamma_Y}=3/8+2\cdot 1/4+2\cdot 3/8=13/8$. 
\eit
\item Marginal code: Code symbols $xy$ by coding $x$ with $\gamma_X$ and then coding $y$ with $\gamma_Y$. 
\item Marginal code has average codeword-length $1+13/8=21/8$, redundancy $R=1/8$.
\item Marginal code not optimal.
\eit


\end{frame}

\begin{frame}{Example for benefit of conditional coding}
\loud{ Conditional probabilites: }
\bit
\item $p_{Y|X}(y=0|x=0)=1/2,\:p_{Y|X}(y=1|x=0)=1/4,\:p_{Y|X}(y=2|x=0)=1/4$ 
\item $p_{Y|X}(y=0|x=1)=1/4,\:p_{Y|X}(y=1|x=1)=1/4,\:p_{Y|X}(y=2|x=1)=1/2$.
\eit
%\item Conditional entropy: $H(Y|X)=3/2$, exercise!  
%\item Optimal codes: 
%\bit
%\item Code $0\mapsto 0$, $1\mapsto 10$, $2\mapsto 11$  is an optimal code for $p_{Y|X}(-|x=0)$,  
%average codeword length $3/2$.
%\item Code $0\mapsto 00$, $1\mapsto 01$, $2\mapsto 1$  is an optimal code for $p_{Y|X}(-|x=1)$,  
%average codeword length $3/2$.
%\eit
\smallskip
\loud{Average codeword length using conditional code is 5/4=20/8}
\bit
\item Code $\gamma^{0}_{Y}$, with $\gamma^{0}_Y(0)=0, \gamma^{0}_Y(1)=10, \gamma^{0}_Y(2)=11$ is an optimal code for $p_{Y|X}(\cdot|x=0)$,  
average codeword length $3/2$.
\item Code $\gamma^{1}_Y$, with $\gamma^{1}_Y(0)=00, \gamma^{1}_Y(1)=01, \gamma^{1}_Y(2)=1$ is an optimal code for $p_{Y|X}(\cdot|x=1)$,  
average codeword length $3/2$.
\item Code combined symbols $xy$ by conditional code: 
\bit
\item Code $x$ using $\gamma_X$.
\item If $x=0$, code $y$ using $\gamma_Y^0$. If $x=1$, code $y$ using $\gamma_Y^1$. 
\eit
\item Average codeword length of this conditional code is $5/2=20/8$, \textbf{exercise! } 
\item Conditional code is optimal. 
\eit
\smallskip
\loud{\iarrow $\:$ Conditional code  has better average codeword length than marginal code. }
\end{frame}

\begin{frame}{Conditional coding not always beneficial}
\loud{Given distribution: }
\bit
\item Consider sources $X$ and $Y$ with alphabets $\mathcal{A}_X=\{0,1\}$ and $\mathcal{A}_Y=\{0,1,2\}$. 
\item Assume that the joint probability $p_{X,Y}$ is given as
\begin{center} 
 \begin{tabular}[hbt!]{ |c|c|c| } 
 \hline
 x & y & $p_{X,Y}(x,y)$ \\
 \hline 
0 & 0 & 1/2\\  
0 & 1 & 1/8\\
0 & 2 &  1/8\\
1 & 0 & 1/8\\
1 & 1 &  1/16\\
1& 2& 1/16\\
 \hline
\end{tabular}.
\end{center}
\eit
\loud{Exercise: }
\bit
\item Optimal average code length is $17/8$.
\item Optimal average code length for conditional coding is $19/8$.
\item Optimal average code length for marginal coding is $19/8$.
\eit 
\end{frame}

%\section{Fundamental Lossless Source Coding Theorem}
%\begin{frame}
% \vspace{8.0ex}
%\begin{center}
%\begin{beamercolorbox}[sep=12pt,center]{part title}
%\usebeamerfont{section title}\insertsection\par
%\end{beamercolorbox}
%\end{center}
%\end{frame}


%\begin{frame}
%\begin{Lemma}{Conditioning reduces entropy for more than one condition}
%Let $X, Y, Z$ be random variables on a discrete probability space $\Omega$.
%Then one has 
%\begin{align*}
%H(Z|X,Y)\leq H(Z|Y). 
%\end{align*}
%\end{Lemma}
%
%\loud{Proof: } 
%
%\bit\small
%\item 
%One has 
%\[
%p(z|x,y)=\frac{p(z,y|x)}{p(y|x)}.
%\]
%\item This implies (\textbf{exercise!}):
%\begin{align*} 
%H(Z|X,Y)=H(Z,Y|X)-H(Y|X).
%\end{align*}
%\textbf{exercise!}
%\item[\iarrow] \textbf{Conditioning does not increase entropy and chain rule for entropy: }
%\begin{align*}
%H(Z,Y|X)-H(Y|X)\leq &H(Z,Y)-H(Y|X)\\
%=&H(Z|Y)+H(Y)-H(Y|X).
%\end{align*}
%\item[\iarrow] \textbf{Chain rule for entropy: }
%\begin{align*}
%H(Z,Y)-H(Y|X)=H(Z|Y)+H(Y)-H(Y|X).
%\end{align*}
%\eit
%
%
%\begin{align*}
%H(Z|X)-H(Z|X,Y)=\sum_{x\in\mathcal{A}_X}\sum_{y\in\mathcal{A}_Y}\sum_{z\in\mathcal{A}_Z}p(x,y,z)\log_2(\frac{p(z|x)}{p(z|(x,y))})
%\end{align*}
%\end{frame}


\subsection{Conditional and joint entropy}

\begin{frame}{Conditional and joint entropy}

\ALERT{Goal: Derive relation between entropy and optimal average codeword length for a random process similar as for 
the case of a single random variable}.

\iarrow$\:$ Need to introduce concepts for the entropy of a random process and derive some properties. 

\vskip 4pt 
Let $X$ and $Y$ be random variables on a discrete probability space $(\Omega,P)$ with alphabets $\mathcal{A}_X,\:\mathcal{A}_Y$. 
\vskip 1pt
\loud{Conditional entropy of $Y$ given $X$:} 
\begin{align*}
H(Y|X):=\sum_{x\in\mathcal{A}_X}p_X(x)H(p(\cdot|x))=-\sum_{x\in\mathcal{A}_X }\sum_{y\in\mathcal{A}_Y}p_{X,Y}(x,y)\log_2(p_{Y|X}(y|x)).
\end{align*}
\textbf{Convention} Here and in the sequel, sums are meant as ommitting summands with probability zero. 
%\bit
%\item \textbf{Convention} Here and in the sequel, sums are meant as ommitting summands with $p_X(x)=0$. 
%\item One has 
%\begin{align}\label{MarginalJointEntropy}
%H(Y|X)
%=-&-\sum_{x\in\mathcal{A}_X }\sum_{y\in\mathcal{A}_Y}p_{X,Y}(x,y)\log_2(p_{Y|X}(y|x)).
%\end{align}
%\eit

\loud{Joint entropy of $X$ and $Y$:} 
\begin{align*}
H(X,Y)=\sum_{x\in\mathcal{A}_X}\sum_{y\in\mathcal{A}_Y}-p_{X,Y}(x,y)\log_2(p_{X,Y}(x,y)),
\end{align*}
entropy of joint distribution $p_{X,Y}$.
\end{frame}


\begin{frame}{Chain rule of entropies}
\begin{proposition}[Chain rule for entropies]
For any two random variables $X,Y$ on a discrete probability space $(\Omega,P)$, one has
\begin{align*}
H(X,Y)=H(Y|X)+H(X).
\end{align*}
\end{proposition}
\loud{Proof of chain rule:}
\begin{align*}
H(X,Y)=&-\sum_{x\in \mathcal{A}_X}\sum_{y\in\mathcal{A}_Y}p_{X,Y}(x,y)\log_2\left(p_{X,Y}(x,y)\right)\\
=&-\sum_{x\in \mathcal{A}_X}\sum_{y\in\mathcal{A}_Y}p_{X,Y}(x,y)\log_2\left(\frac{p_{X,Y}(x,y)p_X(x)}{p_X(x)}\right)\\
=&-\sum_{x\in \mathcal{A}_X}\sum_{y\in\mathcal{A}_Y}p_{X,Y}(x,y)\log_2\left(p_{Y|X}(y|x)\right)-\sum_{x\in \mathcal{A}_X}\sum_{y\in\mathcal{A}_Y}p_{X,Y}(x,y)\log_2(p_X(x)).
\end{align*}
\smallskip
\bit 
\item First summand equals $H(Y|X)$.
\item Second summand equals $H(X)$ by \eqref{MarginalJoint}. \qed
\eit 
\end{frame}







\begin{frame}{Conditioning does not increase entropy}
\begin{proposition}
For any two random variables $X,Y$ on a discrete probability space $(\Omega,P)$, one has
\[
H(Y)\geq H(Y|X)
\]
with equality if and only if $X$ and $Y$ are independent. 
\end{proposition}
%Proof: 
%
\loud{Proof:} Define a probability mass function $q:\mathcal{A}_X\times\mathcal{A}_Y\to\mathbb{R}$ by $q(x,y):=p_X(x)p_Y(y)$. 
Then 
\begin{align*}
H(Y|X)=&-\sum_{x\in\mathcal{A}_X}\sum_{y\in\mathcal{A}_Y}p_{X,Y}(x,y)\log_2\left(\frac{p_{X,Y}(x,y)}{p_X(x)}\cdot \frac{p_Y(y)}{p_Y(y)}\right)\\
=&-\sum_{x\in\mathcal{A}_X}\sum_{y\in\mathcal{A}_Y}p_{X,Y}(x,y)\left(\log_2\left(\frac{p_{X,Y}(x,y)}{q(x,y)}\right)+\log_2(p_Y(y))\right)\\
=&-D(p_{X,Y}||q)+H(Y). 
\end{align*}
\loud{Divergence inequality: }$D(p_{X,Y}||q)\geq 0$, equality only if $p_{X,Y}=q$, i.e. $X$ and $Y$ independent. \qed
\end{frame}


\begin{frame}{Conditional entropy for $N$ random variables}
Let $X_1,\dots,X_N$ be random variables on discrete probability space $\Omega$.
\bit
\item \loud{Conditional entropy} $H(X_N|X_{N-1},\dots,X_1)$ of $X_N$ given  $X_1,\dots,X_{N-1}$:  

Defined as for two random variables $X, Y$. 
\item Properties from two variables extend to $N$ variables by induction: 
\eit
\begin{proposition}[Properties of conditional entropy for random process]
\bit
\item \loud{Chain rule of entropies}: One has
\[
H(X_1,\dots,X_N)=\sum_{i=1}^NH(X_i|X_1,\dots,X_{i-1})
\]
\item \loud{Conditioning does not increase entropy:} One has
\[
H(X_N|X_1,\dots,X_{N-1})\leq H(X_N),
\]
equality if and only if $X_N$ is independent from $(X_1,\dots,X_{N-1})$. 
\eit
\end{proposition}
\end{frame}




\subsection{Entropy rate}
\begin{frame}{Entropy rate and conditional entropy rate}
Let $\mathcal{X}=\left(X_i\right)_{i=1}^\infty$ be a discrete random process.
\bit
\item The \loud{entropy-rate}  of $\mathcal{X}$ is 
defined as 
\begin{align}\label{LimEntrRate}
H(\mathcal{X}):=\lim_{N\to\infty}\frac{H\left(X_1,\dots,X_N\right)}{N}
\end{align}
if the limit exists.
\item The \loud{conditional entropy-rate} of $\mathcal{X}$ is 
defined as 
\begin{align}\label{LimCondEntrRate}
H'(\mathcal{X}):=\lim_{N\to\infty}H\left(X_N|X_1,\dots,X_{N-1}\right),
\end{align}
if the limit exists.
\eit
\end{frame}

\begin{frame}{Entropy rate exists for stationary process}
\begin{theorem}[Entropy rate and conditional rate exist and are equal for stationary process]
If the process $\mathcal{X}$ is stationary, the limits \eqref{LimEntrRate} and \eqref{LimCondEntrRate} exist. Moreover, one has
\begin{align*}
H(\mathcal{X})=H'(\mathcal{X}),
\end{align*}
i.e. the entropy rate is equal to the conditional entropy rate.
\end{theorem}
\loud{Proof that conditional entropy rate exists:}
\bit
\item Using divergence inequality and \eqref{MarginalJoint}, one can show that: 
\begin{align*}
H(X_{n+1}|X_1,\dots,X_n)\leq &H(X_{n+1}|X_2,\dots,X_{n}).\\
\end{align*}
\item Process is stationary: 
\begin{align*}
H(X_{n+1}|X_2,\dots,X_{n})=H(X_{n}|X_1,\dots,X_{n-1}).
\end{align*}
\item [\iarrow] Sequence \eqref{LimCondEntrRate} is monotonically decreasing. Since it is non-negative: Conditional entropy rate exists. 
\eit
\end{frame}


\begin{frame}
\loud{Proof that entropy rate exists and equals conditional entropy rate:}
\bit
\item \loud{Chain rule for entropies}: Sequence of averaged joint entropies is sequence of averages of conditional entropies.
\begin{align*}
\frac{1}{N}H(X_1,\dots,X_N)=\frac{1}{N}\sum_{i=1}^NH(X_i|X_{1},\dots,X_{i-1}).
\end{align*}
\item \loud{Already proved:} Sequence $H(X_i|X_1,\dots,X_{i-1})$ converges to $H'(\mathsf{X})$.
\item \loud{Cauchy limit theorem}: If a sequence of complex number converges to a value, the sequence of averages converges to the same value. See next slide.
\item [\iarrow] Sequence of averaged joint entropies converges to conditional entropy rate. 
\qed
\eit
\end{frame}



\begin{frame}{Cauchy limit theorem}
\begin{lemma}[Cauchy limit theorem]
Let $(a_n)_{n=1}^{\infty}$ be a sequence of complex numbers that converges to $a$. Then, also the sequence of averages 
\[
b_n=\frac{1}{n}\sum_{i=1}^na_i
\]
converges to a. 
\end{lemma}
\loud{Proof:} Let $\epsilon>0$. There exists an $n_0\in\mathbb{N}$ such that $|a_n-a|<\epsilon/2$ for all $n\geq n_0$.
\bit
%\item Let $\epsilon>0$. There exists an $n_0\in\mathbb{N}$ such that $|a_n-a|<\epsilon/2$ for all $n\geq n_0$.
\item Let $c:=\max\{|a_n-a|\colon n\in\mathbb{N}\}$. There exists an $n_1\in\mathbb{N}$ such that $cn_0/n<\epsilon/2$ for all $n\geq n_1$.
 \item[\iarrow] For all $n\geq \max\{n_0,n_1\}$, one has 
\begin{align*}
|b_n-a|=\left|\sum_{i=1}^n\frac{a_i-a}{n}\right|&\leq \sum_{i=1}^n\frac{|a_i-a|}{n}=
\sum_{i=1}^{n_0}\frac{|a_i-a|}{n}+\sum_{i=n_0}^n\frac{|a_i-a|}{n}
 \\ &\leq c\frac{n_0}{n}+\frac{n-n_0}{n}\epsilon/2 \leq \epsilon/2+\epsilon/2=\epsilon. \qed
\end{align*}
\eit 
\end{frame}


\subsection{Block codes}

\begin{frame}{Block codes}
Let $\mathcal{X}=(X_n)_{n=1}^\infty$ be a \loud{stationary random process} with alphabet $\mathcal{A}_X$.  
\bit
\item \loud{Block code of order $N$:} Uniquely decodable code for coding $N$ successive symbols $(x_i,x_{i+1},\dots,x_{i+N})$ of $\mathcal{A}_X$ jointly. 
\eit
\bit
\item \loud{Optimal average codeword length $\overline{\ell}_N$:} Average codeword length for optimal block-code of order $N$. Independent of $i$ because 
process stationary. 
\eit
\bit
\item \loud{Optimal average codeword length per symbol:}
\begin{align*}
\overline{\ell}:=\lim_{N\to\infty}\frac{\overline{\ell}_N}{N},
\end{align*} 
if the limit exists. 
\eit
\bit
\item \loud{Block entropy $H_N(\mathcal{X})$:} Joint entropy $H(X_{i},\dots,X_{i+N})$. Independent of $i$ because process stationary. 
\eit
\end{frame}

\subsection{Main result}
\begin{frame}{Summary: Fundamental theorem of lossless source coding}
\begin{theorem}[Fundamental theorem of lossless source coding]
Let $\mathcal{X}=(X_n)_{n=1}^\infty$ be a \loud{stationary random process}. 
\bit
\item For every $N$, optimal average codeword length and block entropy satisfy:
\begin{align*}
H_N(\mathcal{X})\leq\overline{\ell}_N\leq H_N(\mathcal{X})+1
\end{align*}
\item The limit 
\begin{align*}
\overline{\ell}:=\lim_{N\to\infty}\frac{\overline{\ell}_N}{N},
\end{align*} 
defining the optimal average codeword length per symbol exists.
\item Optimal average codeword length $\overline{\ell}$ per symbol asymptotically achieves entropy rate:
\begin{align*}
\lim_{N\to\infty}\frac{\overline{\ell}_N}{N}=H(\mathcal{X}).
\end{align*}
\eit
\end{theorem}

\end{frame}



\end{document}
